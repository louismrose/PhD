%!TEX root = /Users/louis/Documents/PhD/Deliverables/Thesis/thesis.tex

\cleardoublepage
\begin{abstract}
	Software changes over time. During the lifetime of a software system, unintended behaviour must be corrected and new requirements satisfied. Because software changes are costly, tools for automatically managing change are commonplace. Contemporary software development environments can automatically perform change management tasks such as impact analysis, refactoring and background compilation.

Increasingly, models and modelling languages are first-class citizens in software development. Model-Driven Engineering (MDE), a state-of-the-art approach to software engineering, prescribes the use of models throughout the software engineering process. In MDE, modelling tools and task-specific language are used to generate an ultimate artefact, such as simulation models or working code. 

Contemporary MDE environments provide little support for managing a type of evolution termed \emph{model-metamodel co-evolution}, in which changes to a modelling language are propagated to models. This thesis demonstrates that model-metamodel co-evolution occurs often in MDE projects, and that dedicated structures and processes for its management can increase developer productivity. Structures and processes for managing model-metamodel co-evolution are proposed, developed, and then evaluated by comparison to existing structures and processes with quantitative and qualitative techniques.
\end{abstract} 