%!TEX root = /Users/louis/Documents/PhD/Deliverables/Thesis/thesis.tex

\section{Research Hypothesis}
\label{sec:hypothesis}

The research presented in this thesis explores the hypothesis below. The emboldened terms are potentially ambiguous, and their definition follows the hypothesis.

\begin{quote}
\emph{In existing MDE projects, the evolution of \textbf{MDE development artefacts} is typically managed in an ad-hoc manner with little regard for re-use. Dedicated structures and processes for \textbf{managing evolutionary change} can be designed by analysing evolution in existing MDE projects. Furthermore, supporting those dedicated structures and processes in contemporary MDE environments is beneficial in terms of increased \textbf{productivity} and \textbf{understandability} of software development.}
\end{quote}

In this thesis, the terms below have the following definitions:

\paragraph{MDE development artefacts.} Compared to traditional approaches to software engineering, MDE uses additional development artefacts as first-class citizens in the development process. The additional development artefacts particular to MDE include models and modelling languages, as well as model management operations (such as model transformations). Chapter~\ref{Background} describes models, modelling languages and model management operations in more detail.

\paragraph{Managing evolutionary change.} Contemporary computer systems are constructed by combining numerous interdependent artefacts. Evolutionary changes to one artefact can affect other artefacts. For example, changing a database schema might cause data to become invalid with respect to the database integrity constraints, and changing source code may require recompilation of object code to ensure the latter is an accurate representation of the former. Managing evolutionary change typically comprises three related activities: \emph{identifying} when a change has occurred, \emph{reporting} the effects of a change, and \emph{reconciling} affected artefacts in response to a change.

\paragraph{Productivity} is a measure of the amount of work required to complete a development activity. For example, the productivity of data entry might be increased by using an Optical Character Recognition (OCR) system rather than a typist. In this scenario, the extent to which productivity might be increased is affected by at least the following: the accuracy and capabilities of the OCR system, the speed and accuracy of the typist, and the legibility and consistency of the data. In general, productivity is affected by many factors.


\paragraph{Understandability} is a measure of the ease with which the requirements, implementation, dependencies, and other qualities of a system can be identified from the representation of that system. For example, the function of a circuit is arguably easier to understand when examining a schematic of the circuit rather than a Printed Circuit Board (PCB) that implements the circuit, because components are often arranged to save space on a PCB. Understanding why, how and when a system has evolved in the past is useful for anticipating future changes and improving the development process. Clearly, understandability is somewhat subjective. Due to differences in knowledge and experience, a representation that is easy to understand for one person may be difficult for another. 


\subsection{Thesis Objectives}
The objectives of the thesis are to:

\begin{enumerate}
	\item Identify and analyse the evolution of MDE development artefacts in existing projects.
	\item Investigate the extent to which existing structures and processes can be used to manage the evolution of MDE development artefacts. 
	\item Propose and develop new structures and processes for managing the evolution of MDE development artefacts, and integrate those structures and processes with a contemporary MDE development environment.
	\item Evaluate and assess the proposed structures and processes for managing evolutionary change, particularly with respect to the productivity and understandability of software development.
\end{enumerate}
