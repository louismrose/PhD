%!TEX root = /Users/louis/Documents/PhD/Deliverables/Thesis/thesis.tex

\section{Research Hypothesis}
\label{sec:hypothesis}

The research presented in this thesis explores the hypothesis below. The emboldened terms are potentially ambiguous, and their definition follows the hypothesis.

\begin{quote}
\emph{In existing MDE projects, the evolution of \textbf{MDE development artefacts} is typically managed in an ad-hoc manner with little regard for re-use. Dedicated structures and processes for \textbf{managing evolutionary change} can be designed by analysing evolution in existing MDE projects. Furthermore, supporting those dedicated structures and processes in contemporary MDE environments is beneficial in terms of increased \textbf{productivity} for software development activities pertaining to the management of evolutionary change.}
\end{quote}

In this thesis, the terms below have the following definitions:

\paragraph{MDE development artefacts.} Compared to traditional approaches to software engineering, MDE uses additional development artefacts as first-class citizens in the development process. The additional development artefacts particular to MDE include models and modelling languages, as well as model management operations (such as model transformations). Chapter~\ref{Background} describes models, modelling languages and model management operations in more detail.

\paragraph{Managing evolutionary change.} Contemporary computer systems are constructed by combining numerous interdependent artefacts. Evolutionary changes to one artefact can affect other artefacts. For example, changing a database schema might cause data to become invalid with respect to the database integrity constraints, and changing source code may require recompilation of object code to ensure the latter is an accurate representation of the former. Managing evolutionary change typically comprises three related activities: \emph{identifying} when a change has occurred, \emph{reporting} the effects of a change, and \emph{reconciling} affected artefacts in response to a change. Chapter~\ref{LiteratureReview} reviews existing approaches to managing evolutionary change.

\paragraph{Productivity} is a measure of the output from a process, per unit of input \cite{beattie07economics}. For example, the productivity of data entry might be measured by counting the number of characters produced per typist per hour. An Optical Character Recognition (OCR) system might increase data entry productivity, but this is likely to be dependent on many factors, including: the accuracy and capabilities of the OCR system, the speed and accuracy of each typist, and the legibility and consistency of the data. Managing and measuring the productivity of software engineering is challenging. Division of labour, for example, can decrease productivity in software engineering as evidenced by Brooks's eponymous law (``adding manpower to a late software project makes it later'') \cite{brooks95mythical}. This thesis investigates the productivity of small, well-defined software development activities, and not the productivity of software engineering projects.

\subsection{Thesis Objectives}
The objectives of the thesis are to:

\begin{enumerate}
	\item Identify and analyse the evolution of MDE development artefacts in existing projects.
	\item Investigate the extent to which existing structures and processes can be used to manage the evolution of MDE development artefacts. 
	\item Propose and develop new structures and processes for managing the evolution of MDE development artefacts, and integrate those structures and processes with a contemporary MDE development environment.
	\item Evaluate the proposed structures and processes for managing evolutionary change, particularly with respect to productivity.
\end{enumerate}
