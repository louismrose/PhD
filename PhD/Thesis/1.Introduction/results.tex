%!TEX root = /Users/louis/Documents/PhD/Deliverables/Thesis/thesis.tex

\section{Results of the Thesis Research}
\label{sec:research_results}

This thesis proposes novel structures and processes for managing model-met\-amo\-del co-evolution. Prototypical reference implementations of the proposed structures have been constructed, including \emph{Epsilon HUTN} (a textual modelling notation) and \emph{Epsilon Flock} (a model migration language). The reference implementations have extended and reused \changed{``have been built atop'' changed to ``have extended and reused''} Epsilon \cite{kolovos09thesis}, an extensible platform for specifying MDE languages and tools, and are interoperable with the Eclipse Modelling Framework \cite{steinberg09emf}, arguably the most widely used MDE modelling framework.

Additionally, this thesis identifies a novel process for managing model-metamodel co-evolution and proposes a theoretical categorisation of existing process for managing model-metamodel co-evolution. The novel process, termed \emph{user-driven co-evolution}, is demonstrated by application to a MDE development process for a real-world project.

The research hypothesis has been validated by comparing the prototypes of the proposed structures and processes with existing structures and processes using examples of evolution from real-world MDE projects. Evaluation has been performed using several approaches, including a collaborative comparison of model migration tools carried out with three MDE experts, comparing quantitive measurements of the proposed and existing migration languages, and application of the proposed structures and processes to two examples of evolution, including an example from a widely used modelling language, the Unified Modelling Language (UML) \cite{uml212}. The evaluation also explored areas in which the prototypical implementations of the proposed structures and processes might be usefully improved to be fit for industrial use.


% Productivity
%  HUTN + MMIS for user-driven co-evolution
%  A choice of user-driven vs developer-driven co-evolution
%  Dedicated migration lang rather than M2M lang for model migration

% Understandability
%  Flock for expressing model migration
%  Use of design patterns / refactorign terminology to describe metamodel evolution
%  HUTN rather than XMI for representating non-conformant models

% Might also mention portability:
%  Migrating between modelling technologies
%  Higher-order migration (migrating between transformation [and other model management?] languages)
