%!TEX root = /Users/louis/Documents/PhD/Deliverables/Thesis/thesis.tex

\section{Motivation: Software Evolution in MDE}
Proponents of MDE suggest that, compared to traditional approaches to software engineering, application of MDE leads to systems that better support evolutionary change \cite{kleppe03mda}. \cite{frankel02mda} suggests that large-scale systems, developed with traditional approaches to software engineering, are examples of a  modern-day Sisyphus\footnote{In Greek mythology, Sisyphus was condemned to an eternity of repeatedly rolling a boulder to the top of a mountain, only to see it return to the mountain's base.}, whose developers must constantly perform evolution to support conformance to changing standards and interoperability with external systems, and that MDE can be used to reduce the cost of software evolution. However, \cite{Mens07} report that MDE introduces additional challenges for managing evolution.

In particular, the evolution of models, modelling languages and other MDE development artefacts must be managed in MDE. Contemporary development environments provide some assistance for performing software evolution activities (by, for example, providing transformations that automatically restructure code). However, there is little support for software evolution activities that involve models and modelling languages. Chapters~\ref{LiteratureReview} and~\ref{Analysis} review and analyse the support for software evolution available in contemporary MDE development environments.

% There is an urgent need for development environments that support the evolution of models, modelling languages and other MDE development artefacts. Without this support, software development with MDE is less productive and harder to reason about. Resources that could be used to improve the software system are wasted, used instead to manually identify and reconcile the problems caused by evolution. The way in which the system changes over time is not recorded and cannot be analysed to anticipate future changes or to improve the development process.

This thesis explores the extent to which the productivity of managing evolutionary change can be increased by enhancing contemporary MDE development environments with dedicated structures and processes.
