%!TEX root = /Users/louis/Documents/PhD/Deliverables/Thesis/thesis.tex

\section{Thesis Structure}
Chapter~\ref{Background} gives an overview of MDE by defining terminology; describing associated engineering principles, practices and tools; and reviewing related areas of computer science. Section~\ref{sec:mde_benefits_and_challenges} synthesises some of the benefits of, and challenges for, contemporary MDE.

Chapter~\ref{LiteratureReview} reviews theoretical and practical software evolution research. Areas of research that underpin software evolution are described, including refactoring, design patterns, and traceability. The review then discusses work that approaches particular categories of evolution problem, such as programming language, schema and grammar evolution. Section~\ref{subsec:mde_evo} surveys work that considers evolution in the context of MDE. Section~\ref{sec:literature_review_summary} identifies three types of evolution that occur in MDE projects and highlights challenges for their management.

Chapter~\ref{Analysis} surveys existing MDE projects and categorises the evolution of MDE development artefacts in those projects. From this survey, the context for the thesis research is narrowed, and the remainder of the thesis focuses on one type of evolution occurring in MDE projects, termed \emph{model-metamodel co-evolution} or simply \emph{co-evolution}. Examples of co-evolution are used to identify the strengths and weaknesses of existing structures and processes for managing co-evolution. From this, Section~\ref{subsec:user-driven_co-evolution} identifies a process for managing co-evolution which has not been recognised previously in the literature, Section~\ref{subsec:co-evolution_categorisation} derives a categorisation of existing processes for managing co-evolution, and Section~\ref{sec:requirements_identification} synthesises requirements for novel structures for managing co-evolution.

Chapter~\ref{Implementation} describes novel structures for managing co-evolution, including a metamodel-independent syntax, which is used to identify, report and to facilitate the reconciliation of problems caused by metamodel evolution. The textual modelling notation described in Section~\ref{sec:notation} and the model migration language described in Section~\ref{sec:flock} are used for reconciliation of models in response to metamodel evolution. The latter provides a means for performing reconciliation in a repeatable manner.

Chapter~\ref{Evaluation} assesses the structures and processes proposed in this thesis by comparison to existing structures and processes. To explore the research hypothesis, several different types of comparison were performed, including an experiment in which quantitive measurements were derived, a collaborative comparison of model migration tools with three MDE experts, and application to a large, independent example of evolution taken from a real-world MDE project.

Chapter~\ref{Conclusion} summarises the achievements of the research, and discusses results in the context of the research hypothesis. Limitations of the thesis research and areas of future work are also outlined.