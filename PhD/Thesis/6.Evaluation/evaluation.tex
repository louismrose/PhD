%!TEX root = /Users/louis/Documents/PhD/Deliverables/Thesis/thesis.tex

\chapter{Evaluation}
\label{Evaluation}

%The evaluation chapter will outline our evaluation method and results, including the impact and limitations of our research; and discuss the extent to which the requirements identified in the analysis chapter have been fulfilled. Evaluation will be conducted in three ways: application of our structures and processes in a case study; publication of our research in academic journals, international conferences and workshops; and assessing the contribution made when delivering our work through an Eclipse research incubation project.

\section{Evaluation Measures}

%!TEX root = /Users/louis/Documents/PhD/Deliverables/Thesis/thesis.tex

\section{Evaluating User-Driven Co-Evolution}
\label{sec:exemplar_user-driven_co-evo}
This section explores the extent to which developer productivity increases when dedicated structures are used for performing \emph{user-driven co-evolution} (in which a model migration strategy is not specified in an executable format and the metamodel user performs migration on their models). Chapter~\ref{Analysis} described several real-world MDE projects in which user-driven co-evolution has been observed, and noted that no tool support for user-driven co-evolution has been reported in the literature. Chapter~\ref{Implementation} proposed two structures to support user-driven co-evolution, a metamodel-independent syntax (Section~\ref{sec:mmi_syntax}) and a textual modelling notation (Section~\ref{sec:notation}). This section explores the extent to which the structures increase the productivity of user-driven co-evolution, supporting the research hypothesis which stated that \emph{integrating dedicated structures and processes with contemporary MDE environments is beneficial in terms of increased productivity}.

To explore this claim, several approaches to evaluation could have been used. The metamodel-independent syntax and textual modelling notation are freely available as part of Epsilon, a component of the Eclipse Modeling Project. The productivity benefits of the structures might have been explored by gathering and analysing the opinion of users. However, this approach was discounted because drawing meaningful conclusions would have likely required understanding the domain, context and background of each user. Alternatively, evaluation might have been performed with a comprehensive user study that measured the time taken for developers to perform model migration with and without the dedicated structures for user-driven co-evolution. However, locating developers and co-evolution examples for this study was not possible given the time available to perform the evaluation. Instead, evaluation was conducted by comparing two approaches to user-driven co-evolution using an example of user-driven co-evolution from a real-world MDE project. The first approach uses only those tools available in the Eclipse Modeling Framework (EMF), arguably the most widely-used contemporary MDE development environment; while the second approach uses EMF together with the metamodel-independent syntax and textual modelling notation introduced in Chapter~\ref{Implementation}.

The remainder of this section first summarises Section~\ref{subsec:user-driven_co-evolution}, which described the challenges to productivity faced by developers while performing user-driven co-evolution with EMF. Section~\ref{subsec:user-driven_co-evolution_example} introduces the example of user-driven co-evolution used to perform the evaluation. In Sections~\ref{subsec:user-driven_co-evolution_with_emf} and~\ref{subsec:user-driven_co-evolution_with_dedicated_structures}, the two approaches to user-driven co-evolution are demonstrated. The section concludes by comparing the two approaches and highlighting ways in which the metamodel-independent syntax and textual modelling notation increase developer productivity in the context of user-driven co-evolution.

\subsection{Challenges for Performing User-Driven Co-Evolution}
Two productivity challenges for performing user-driven co-evolution in contemporary MDE environments were identified in Section~\ref{subsec:user-driven_co-evolution} and are now summarised. Firstly, model storage representations have not been optimised for use by humans, and hence user-driven co-evolution -- which typically involves changing models by hand -- can be error-prone and time consuming. Secondly, the multi-pass parsers used to load models in contemporary MDE environments cause user-driven co-evolution to be an iterative process, because not all conformance errors are reported at once. The identification of these productivity challenges led to the derivation of the following research requirement in Section~\ref{sec:requirements_identification}: \emph{This thesis must demonstrate a user-driven co-evolution process that enables the editing of non-conformant models without directly manipulating the underlying storage representation and provides a conformance report for the original model and evolved metamodel.}

Two of the structures presented in Chapter~\ref{Implementation} provide the foundation for fulfilling the above research requirement. The first, a metamodel-independent syntax, facilitates the conformance checking of a model against any metamodel. The second structure, the textual modelling notation \emph{Epsilon HUTN}, allows models to be managed in a format that is reputedly easier for humans to use than XMI, the canonical model storage format \cite{hutn}.

To fulfil the above research requirement, this section uses the metamodel-independent syntax and the textual modelling notation to demonstrate that user-driven co-evolution can be performed without encountering the challenges to productivity described above. To this end, an example of co-evolution is used to show the way in which user-driven co-evolution might be achieved with and without the metamodel-independent syntax and Epsilon HUTN.

\subsection{Co-Evolution Example}
\label{subsec:user-driven_co-evolution_example}
The remainder of this section uses a co-evolution example taken from collaborative work with Adam Sampson, then a Research Associate at the University of Kent. The purpose of the collaboration was to build a prototypical editor for graphical models of programs written in process-oriented programming languages, such as occam-$\pi$ \cite{occam_pi}. The graphical models would provide a standard notation for describing process-oriented programs.

The collaboration with Sampson was selected for the evaluation presented here for several reasons. Firstly, the work involved constructing a graphical model editor, which is a common MDE development activity as evidenced by the numerous graphical modelling frameworks surveyed in \cite{amyot06evaluation}. Secondly, the editor was developed in an incremental and iterative manner, and involved several different types of change to the metamodel, some of which affected conformance. Finally, a relatively small number of models were constructed during the collaboration, and hence a user-driven approach to managing co-evolution was more suitable than a developer-driven approach for this example.

The graphical model editor was developed using a MDE approach. A metamodel was used to capture the abstract syntax of process-oriented programming languages, and code for a graphical model editor was automatically generated from the metamodel. 

The final version of the graphical model editor is shown in Figure~\ref{fig:po_final_graphical_editor}. The editor captures the three primary concepts used to specify process-oriented programs: processes, connection points and channels. Processes, represented as boxes in the graphical notation, are the fundamental building blocks of a process-oriented program. Channels, represented as lines in the graphical notation, are the mechanism by which processes communicate, and are unidirectional. Connection points, represented as circles in the graphical notation, define the channels on which a process can communicate. Because channels are unidirectional, connection points are either reading (consume messages from the channel) or writing (generate messages on the channel). Reading (writing) connection points are represented as white (black) circles in the graphical notation.  

\begin{figure}[htbp]
	\centering
	\includegraphics[width=13.5cm]{6.Evaluation/images/user_driven/po_final_editor.png}
	\caption{Final version of the prototypical graphical model editor.}
	\label{fig:po_final_graphical_editor}
\end{figure}

The graphical model editor was implemented using EMF. The metamodel was specified in Ecore, the metamodelling language of EMF, and the editor was generated from the metamodel using the Graphical Modeling Framework (GMF), an extension to EMF for graphical modelling. Section~\ref{sec:mde_tools} describes in more detail the way in which EMF and GMF can be used to specify metamodels and to generate graphical model editors.

The process-oriented metamodel was developed iteratively, and the six iterations are described in Appendix~\ref{ProcessOriented}. During each iteration, the metamodel was changed. The remainder of this section uses an example of metamodel changes from the fifth iteration of the project. The way in which development proceeded during that iteration is described in Section~\ref{sec:po_it5} and summarised below.

\subsubsection{Aim of Iteration 5}
The purpose of the iteration was to refine the way in which connection points were represented. At the start of the iteration, the graphical model editor could be used to draw processes, channels and connection points. However, no distinction was made between reading and writing connection points.

Figure~\ref{fig:po_original_editor} shows an exemplar model represented in the graphical model editor before the iteration began. The model contains two processes (depicted as boxes), \texttt{P1} and \texttt{P2}, one channel (depicted as a line), \texttt{a}, and two connection points (depicted as circles), \texttt{a!} and \texttt{a?}.

\begin{figure}[htbp]
	\centering
	\includegraphics[width=13.5cm]{6.Evaluation/images/user_driven/po_original_editor.png}
	\caption{The graphical editor at the start of the iteration.}
	\label{fig:po_original_editor}
\end{figure}


The aim of the iteration was to distinguish between reading and writing connection points in the graphical notation. The former are used to receive messages, and the latter to send messages. In Figure~\ref{fig:po_original_editor}, \texttt{a?} is intended to represent a reading connection point, and \texttt{a!} a writing connection point. Sampson and the author decided that the editor should be changed so that black circles would be used to represent writing connection points, and white circles to represent reading connection points. At the end of the iteration the model shown in Figure~\ref{fig:po_original_editor} would be represented as shown in Figure~\ref{fig:po_evolved_editor}. Furthermore, the editor would ensure that \texttt{a?} was used only as the reader of a channel, and \texttt{a!} only as the writer of a channel. Before the iteration started, the editor did not enforce this constraint.

\begin{figure}[htbp]
	\centering
	\includegraphics[width=13.5cm]{6.Evaluation/images/user_driven/po_evolved_editor.png}
	\caption{The graphical editor at the end of the iteration.}
	\label{fig:po_evolved_editor}
\end{figure}


\subsubsection{Implementation during Iteration 5}
Before the iteration began, the metamodel, shown in Figure~\ref{fig:po_original_mm}, did not distinguish between reading and writing \texttt{Co\-nn\-ec\-ti\-o\-nPo\-i\-nt}s. A \texttt{Co\-nn\-ec\-ti\-o\-nPo\-i\-nt} could be associated with a \texttt{Ch\-an\-nel} via the \texttt{re\-ad\-er} or \texttt{wr\-it\-er} reference of \texttt{Ch\-an\-nel}, but the type of a \texttt{Co\-nn\-ec\-ti\-o\-nPo\-i\-nt} was not specified explicitly.

The way in which connection points were modelled was changed, resulting in the metaclasses shown in Figure~\ref{fig:po_evolved_mm}. \texttt{Co\-nn\-ec\-ti\-o\-nPo\-i\-nt} was made abstract, and two subtypes, \texttt{Re\-ad\-i\-ngCo\-nn\-ec\-ti\-o\-nPo\-i\-nt} and \texttt{Wr\-i\-ti\-ngCo\-nn\-ec\-ti\-o\-nPo\-i\-nt}, were introduced. The \texttt{re\-ad\-er} and \texttt{wr\-it\-er} references of \texttt{Ch\-an\-n\-el} were changed to refer to the new subtypes. The evolved metamodel correctly prevented the use of a \texttt{Co\-nn\-ec\-ti\-o\-nPo\-i\-nt} as both a \texttt{re\-ad\-er} and a \texttt{wr\-it\-er}.

\begin{figure}[htbp]
	\centering
	\subfigure[Part of the original metamodel.]
	{
	    \label{fig:po_original_mm}
	    \includegraphics[width=7.5cm]{6.Evaluation/images/user_driven/po_before.pdf}
	}
	\subfigure[Part of the evolved metamodel.]
	{
	    \label{fig:po_evolved_mm}
	    \includegraphics[width=11cm]{6.Evaluation/images/user_driven/po_after.pdf}
	}
	\caption{Process-oriented metamodel evolution.}
\label{fig:po_mms}
\end{figure}

Following the metamodel changes, a new version of the graphical editor was generated automatically from the metamodel using GMF. An annotation -- not shown in Figure~\ref{fig:po_evolved_mm} -- on the \texttt{Wr\-i\-ti\-ngCo\-nn\-ec\-ti\-o\-nPo\-i\-nt} class was used to indicate to GMF that black circles were to be used to represent writing connection points in the graphical notation.

\subsubsection{Testing during Iteration 5}
Testing the new version of the graphical editor highlighted the need for model migration. Attempting to load existing models, such as the one shown in Figure~\ref{fig:po_original_editor}, caused an error because \texttt{Co\-nn\-ec\-ti\-onP\-oi\-nt} was now an abstract class. Any model specifying at least one connection point no longer conformed to the metamodel. Model migration was performed to re-establish conformance and to allow the models to be loaded. 

Several models, presented in Appendix~\ref{ProcessOriented}, had been constructed when testing previous versions of the graphical editor. The models were used during each iteration to ensure that any changes had not introduced regressions. After the metamodel changes described above, the test models could no longer be loaded and required migration. A user-driven rather than a developer-driven co-evolution approach was preferred throughout the development of process-oriented editor because only a few small models required migration in each iteration.

The sequel describes the way in which migration was performed during the development of the process-oriented metamodel, without dedicated structures for performing user-driven co-evolution. Section~\ref{subsec:user-driven_co-evolution_with_dedicated_structures} describes the way in which migration could have been performed using two of the structures presented in Chapter~\ref{Implementation}. The section concludes by comparing the two approaches.

\subsection{User-Driven Co-Evolution with EMF}
\label{subsec:user-driven_co-evolution_with_emf}
During the development of the process-oriented metamodel, no structures for performing user-driven co-evolution were available. Instead, migration was performed using only those tools available in EMF, as described below.

Migration with EMF involved identifying and fixing conformance errors, using the workflow shown in Figure~\ref{fig:emf_process}. When the user attempts to load a model in the graphical editor, EMF automatically checks the conformance of the model. If the model does not conform to the process-oriented metamodel, conformance errors are reported, loading fails and the model is not displayed in the graphical editor. To re-establish conformance, the user must edit by hand the underlying storage representation of the model, XMI. After saving the reconciled XMI to disk, the user attempts to load the model in the graphical editor again. If the user makes a mistake in reconciling the XMI, loading will fail again and further conformance errors will be reported. Even if the user makes no mistakes in reconciling the XMI, further conformance errors might be reported because EMF uses a multi-pass XMI parser and cannot report all categories of conformance problem in one pass of the XMI. If further conformance problems are reported, the user continues to reconcile the XMI by hand. Otherwise, migration is complete and the model is displayed in the graphical editor. 

One of the test models, shown in Figure~\ref{fig:po_original_editor}, is now used to illustrate the way in which user-driven co-evolution was performed using the workflow shown in Figure~\ref{fig:emf_process}. For the test model shown in Figure~\ref{fig:po_original_editor}, the conformance problems shown in the bottom pane (and by the error markers in the left-hand margin of the top pane) of Figure~\ref{fig:po_original_xmi} were reported by EMF. For example, the first conformance problem reported is shown in the tooltip in Figure~\ref{fig:po_original_xmi}, and states that a \texttt{ClassNotFoundException} was encountered because the ``Class `ConnectionPoint' is not found or is abstract.''

\begin{figure}[htbp]
	\centering
		\includegraphics*[viewport=80 280 600 550,height=5.75cm]{6.Evaluation/images/user_driven/emf_process.pdf}
	\caption{User-driven co-evolution with EMF}
	\label{fig:emf_process}
\end{figure}

\begin{figure}[htbp]
	\centering
	\includegraphics[width=13.5cm]{6.Evaluation/images/user_driven/po_original_xmi.png}
	\caption{XMI prior to migration}
	\label{fig:po_original_xmi}
\end{figure}

The conformance problems were fixed by editing the XMI shown in Figure~\ref{fig:po_original_xmi}, producing the XMI shown in Figure~\ref{fig:po_migrated_xmi}. The type of each connection point element was changed to either \texttt{Re\-ad\-i\-ngCo\-nn\-ec\-ti\-o\-nPo\-i\-nt} or \texttt{Wr\-i\-ti\-ngCo\-nn\-ec\-ti\-o\-nPo\-i\-nt}. The former was used when the connection point was referenced via the \texttt{reader} reference of \texttt{Channel}, and the latter otherwise. The reconciled XMI is shown in Figure~\ref{fig:po_migrated_xmi}. On lines 4 and 7, the connection point model elements have been changed to include \texttt{xsi:type} attributes, which specify whether the connection point should instantiate \texttt{Re\-ad\-i\-ngCo\-nn\-ec\-ti\-o\-nPo\-i\-nt} or \texttt{Wr\-i\-ti\-ngCo\-nn\-ec\-ti\-o\-nPo\-i\-nt}.

\begin{figure}[htbp]
	\centering
	\includegraphics[width=13.5cm]{6.Evaluation/images/user_driven/po_migrated_xmi.png}
	\caption{XMI after migration}
	\label{fig:po_migrated_xmi}
\end{figure}

Reconciling the conformance problems by editing the XMI required considerable knowledge of the XMI specification. For example, the \texttt{xsi:type} attribute is used to specify the type of the connection point model elements. In fact, it must be included for those model elements. However, for the other model elements in Figure~\ref{fig:po_migrated_xmi} the \texttt{xsi:type} attribute is not necessary, and is omitted. When and how to use the \texttt{xsi:type} attribute is discussed further in the sidebar, in the XMI specification \cite{xmi}, and in \cite{steinberg09emf}. EMF abstracts away from XMI, and typically users do not interact directly with XMI. Therefore, it may be reasonable to assume that EMF users might not be familiar with XMI, and implementation details such as the \texttt{xsi:type} attribute.

\begin{framed}
\textbf{The \texttt{xsi:type} attribute} \\
In XMI, each model element must indicate the metaclass that it instantiates. Typically, the \texttt{xsi:type} attribute is used for this purpose. For example, the model element on line 4 of Figure~\ref{fig:po_migrated_xmi} instantiates the metaclass named \texttt{Wr\-i\-ti\-ngCo\-nn\-ec\-ti\-o\-nPo\-i\-nt}. To reduce the size of models on disk, the XMI specification allows type information to be omitted when it can be inferred. For example, line 9 of Figure~\ref{fig:po_migrated_xmi} defines a model element that is contained in the \texttt{ch\-an\-ne\-ls} reference of a \texttt{Pr\-o\-ce\-ss}. Because the \texttt{ch\-an\-ne\-ls} reference can contain only one type of model element (\texttt{Ch\-an\-n\-el}), the \texttt{xsi:type} attribute can be omitted, and the type information is inferred from the metamodel.
\end{framed}

% For every instance of \texttt{Ch\-an\-n\-el} that referenced a \texttt{Co\-nn\-ec\-ti\-o\-nPo\-i\-nt}, the following message was produced: ``Unresolved reference `\texttt{<ID>}' '' where \texttt{<ID>} was the identifier of the referenced \texttt{Co\-nn\-ec\-ti\-o\-nPo\-i\-nt}.

% Notice that conformance problem markers are still present in Figure~\ref{fig:po_original_editor}

During the development of the process-oriented editor, some mistakes were made when the XMI of the test models was edited by hand. For example, the wrong subtype of \texttt{Co\-nn\-ec\-ti\-onPo\-in\-t} was used as the type of several connection point model elements. The mistake occurred because XMI identifies model elements using an offset from the root of the document. For example, consider the XMI shown in Figure~\ref{fig:po_migrated_xmi}. The channel on line 9 specifies the value ``//@processes.1/@connectionPoints.0'' for its \texttt{re\-ad\-er} attribute. The value is an XMI path referencing the first connection point (``@connectionPoints.0'') contained in the second process (``@processes.1'') of this document (``//''); in other words the connection point on line 7. One of Sampson's models contained many channels and connection points and incorrectly counting the connection points in the model led to several mistakes during the manual editing of the XMI. Each time a mistake was made when reconciling the XMI by hand, another loop around the workflow shown in Figure~\ref{fig:emf_process} was required.

As demonstrated above, migration using only the tools provided by EMF can be iterative and error-prone. The sequel demonstrates that, by using the dedicated structures described in Chapter~\ref{Implementation}, migration can be performed in one iteration, without requiring the developer to switch between conformance reporting and model migration tools. In addition, the sequel suggests how the mistake described above might be avoided by using Epsilon HUTN rather than XMI for manually migrating models.

\subsection{User-Driven Co-Evolution with Dedicated Structures}
\label{subsec:user-driven_co-evolution_with_dedicated_structures}

Chapter~\ref{Implementation} describes two structures that can be used to perform user-driven co-evolution. Here, the functionality of the two structures, a metamodel-independent syntax and a textual modelling notation, is summarised. Subsequently, an approach that uses the metamodel-independent syntax and the textual modelling notation for migrating the model from the process-oriented example is presented. The model migration example presented in this section was performed retrospectively by the author after the process-oriented editor was completed, and demonstrates how migration might have been achieved with dedicated structures for user-driven co-evolution. The sequel compares the user-driven co-evolution approach presented in this section with the approach presented in Section~\ref{subsec:user-driven_co-evolution_with_emf}.

The metamodel-independent syntax presented in Section~\ref{sec:mmi_syntax} allows non-conformant models to be loaded with EMF, and for the conformance of models to be checked against any metamodel. Epsilon HUTN, the textual modelling notation presented in Section~\ref{sec:notation} is built atop the metamodel-independent syntax and is an alternative to XMI for representing models in a textual format. Together, the two structures can be used for performing user-driven co-evolution using the workflow shown in Figure~\ref{fig:hutn_process}. First, the user attempts to load a model in the graphical editor. If the model is non-conformant and cannot be loaded, the user clicks the ``Generate HUTN'' menu item, and the model is loaded with the metamodel-independent syntax and then a HUTN representation of the model is generated by Epsilon HUTN. The generated HUTN is presented in an editor that automatically reports conformance problems using the metamodel-independent syntax. The user edits the HUTN to reconcile conformance problems, and the conformance report is automatically updated as the user edits the model. When the conformance problems are fixed, XMI for the conformant model is automatically generated, and migration is complete. The model can then be loaded in the graphical editor.

\begin{figure}[htbp]
	\centering
	\includegraphics*[viewport=80 290 760 550,height=4.75cm]{6.Evaluation/images/user_driven/hutn_process.pdf}
	\caption{User-driven co-evolution with dedicated structures}
	\label{fig:hutn_process}
\end{figure}

The way in which the workflow shown in Figure~\ref{fig:hutn_process} was used to perform user-driven co-evolution for the process-oriented metamodel is now demonstrated. For the model shown in Figure~\ref{fig:po_original_editor}, the HUTN shown in Figure~\ref{fig:po_hutn} was generated by invoking the automatic XMI-to-HUTN transformation. The HUTN development tools automatically present any conformance problems, as shown in the bottom pane (and the left-hand margin of the top pane) in Figure~\ref{fig:po_hutn}.

\begin{figure}[htbp]
  \centering
  \includegraphics[width=13.5cm]{6.Evaluation/images/user_driven/po_hutn.png}
  \caption{HUTN source prior to migration}
  \label{fig:po_hutn}
\end{figure}

Conformance problems are reconciled manually by the user, who edits the HUTN source. Conformance is automatically checked whenever the HUTN is changed. For example, Figure~\ref{fig:po_hutn_partial} shows the HUTN editor when migration is partially complete. Some of the conformance problems have been reconciled, and the associated error-markers are no longer displayed in the left-hand margin.

\begin{figure}[htbp]
  \centering
  \includegraphics[width=13.5cm]{6.Evaluation/images/user_driven/po_hutn_partial.png}
  \caption{HUTN source part way through migration}
  \label{fig:po_hutn_partial}
\end{figure}

When no conformance errors remain, Epsilon HUTN automatically generates XMI for reconciled model, and the user can now successfully load the migrated model with the graphical editor.

\subsection{Comparison}
\label{subsec:user_driven_example_comparison}
To suggest ways in which dedicated structures for user-driven co-evolution might increase developer productivity, the two user-driven co-evolution approaches demonstrated above are now compared. The first approach, described in Section~\ref{subsec:user-driven_co-evolution_with_emf}, uses only those tools available in EMF for performing user-driven co-evolution, while the second approach, described in Section~\ref{subsec:user-driven_co-evolution_with_dedicated_structures} uses two of the structures introduced in Chapter~\ref{Implementation}. Applying the approaches to the process-oriented example highlighted differences between the modelling notations used, and the way in which conformance problems were reported.

\subsubsection{Differences in modelling notation}
For reconciling conformance problems, the two approaches used different modelling notations, XMI and Epsilon HUTN. Differences in notation that might influence developer productivity during user-driven co-evolution are now discussed. However, further work is required to more rigorously explore the extent to which developer productivity is affected by the modelling notation, as discussed in Section~\ref{subsec:user_driven_further_work}.

The way in which the type of a model element is specified varies between XMI and HUTN. In XMI, type information can be omitted in some circumstances, but must be included in others. In HUTN, type information is mandatory for every model element. Consequently, every HUTN document contains examples of how type information should be specified, whereas XMI documents may not. 

Reference values are specified using paths in XMI (such as \texttt{//@pro\-ces\-ses.1/@con\-nec\-ti\-onPo\-in\-ts.0}) and by name (such as \texttt{a?}) in HUTN. XMI paths are constructed in terms of a document's structure and, as such, rely on implementation details. The name of a model element, on the other hand, is specified in the model, and does not rely on any implementation details. Consequently, it is conceivable that fewer mistakes will be made during user-driven co-evolution when reference values are specified by name rather than with the structural details of a model.

\subsubsection{Differences in conformance reports}
The two approaches varied in the way in which conformance problems were reported, and, as a consequence, the first approach was iterative and the second was not. The way in which these differences might influence developer productivity during user-driven co-evolution are now discussed. Again, further work is required to more rigorously explore the extent to which developer productivity is affected by the differences in conformance reporting, as discussed in Section~\ref{subsec:user_driven_further_work}.

With EMF, user-driven co-evolution is an iterative process. Conformance errors are fixed by the user, who then reloads the reconciled model (with, for example, a graphical editor). Each time the model is loaded, further conformance problems might be reported when, for example, the user makes a mistake when reconciling the model. By contrast, the implementation of HUTN described in Section~\ref{sec:notation} uses a background compiler that checks conformance while the user edits the HUTN source. When the user makes a mistake reconciling the HUTN source, the error is reported immediately, and does not require the model to be loaded in the graphical editor.

Although not demonstrated in the example considered in this section, user-driven co-evolution would, for some types of metamodel changes, remain an iterative process even if EMF performed conformance checking in the background. Because EMF uses a multi-pass parser, some types of conformance problem are reported before other types. For example, conformance problems relating to multiplicity constraints (e.g. a process does not specify a name, but name is a mandatory attribute) are reported after all other types of conformance problem. When several types of conformance problem have been affected by metamodel changes, user-driven co-evolution with EMF would remain an iterative process. Single-pass, background parsing is required to display all conformance problems while the user migrates a model.

\subsection{Towards a more thorough comparison}
\label{subsec:user_driven_further_work}
Although the above comparison suggests that dedicated structures for performing user-driven co-evolution might increase developer productivity, further research is required to more rigorously evaluate this claim. The ways in which this evaluation might be extended in the future are now discussed.

A comprehensive user study, involving hundreds of users, is one means for exploring the extent to which productivity varies when dedicated structures are used to perform user-driven co-evolution. Ideally, participants for the study would constitute a large and representative sample of the users of EMF. Productivity might be measured by the time taken to perform co-evolution. To remove a potential source of bias, several examples of co-evolution might be used.

% Autocompletion
% Iterative problem reporting might be good - perhaps it could be used to show "likely" root cause problems first, and problems less likely to be a root cause later?s
% Other types of XMI ID: UUID, derived


Locating a reasonable number of participants and co-evolution examples for a comprehensive user study was not feasible in the context of this thesis. Nevertheless, the comparison presented in Section~\ref{subsec:user_driven_example_comparison} suggests that productivity might be increased when using dedicated structures for user-driven co-evolution. By demonstrating an approach to user-driven co-evolution that uses dedicated structures, this thesis provides a foundation for further, more rigorous evaluation. For example, the HUTN specification \cite{hutn} makes claims about the human-usability of the notation, but the usability of HUTN has not been studied or compared with other modelling notations. Epsilon HUTN (Section~\ref{sec:notation}) is a reference implementation of HUTN and, as demonstrated by the evaluation presented here, facilitates the evaluation of HUTN and the comparison of HUTN to other modelling notations, such as XMI.


\subsection{Summary}
\label{subsec:user_driven_example_summary}
This section has demonstrated two approaches to user-driven co-evolution using a co-evolution example from a project in which a graphical model editor was created for process-oriented programs. The first approach used the structures available in EMF alone, while the second approach used two of the structures described in Chapter~\ref{Implementation}. Comparing the two approaches highlighted differences between the way in which conformance problems were reported and between the modelling notations used to reconcile conformance problems. The comparison described in Section~\ref{subsec:user_driven_example_comparison} suggests that developer productivity might be increased by using the second approach, but, as discussed in Section~\ref{subsec:user_driven_further_work}, further work is required to more rigorously evaluate this claim.


%!TEX root = /Users/louis/Documents/PhD/Deliverables/Thesis/thesis.tex

\subsection{Collaborative Case Study}
%!TEX root = /Users/louis/Documents/PhD/Deliverables/Thesis/thesis.tex

\section{Transformation Tools Contest}
\label{sec:ttc}
The Transformation Tools Contest (TTC) is a workshop series that seeks to compare and contrast tools for performing model and graph transformation. At TTC 2010, two rounds of submissions were invited: cases (transformation problems, three of which are selected by the workshop organisers) and solutions to the selected cases. In addition, TTC 2010 include a \emph{live contest}: during the workshop a further transformation problem was announced and solutions submitted.

Participation in TTC 2010 facilitated further evaluation of Flock. Flock and 8 other transformation tools were assessed for a model migration problem based on a real-world example of metamodel evolution from the UML \cite{uml22}. As part of the live contest, Flock was also assessed along with 13 transformation tools for a model transformation problem. Compared to the evaluation described in Section~\ref{sec:collaborative_comparison}, the evaluation in this section compares Flock to a wider range of tools (model and graph transformation tools, and not just model migration tools), and investigates the suitability of Flock for specifying model transformation.

The remainder of this section describes the model migration problem (Section~\ref{subsec:ttc_case}) and Flock solution (Section~\ref{subsec:ttc_solution}), and the use of Flock for specifying a model transformation in the live contest (Section~\ref{subsec:ttc_live_contest}).

\subsection{Model Migration Case}
\label{subsec:ttc_case}
To compare Flock with other transformation tools for specifying model migration, the thesis author submitted a case to TTC based on the evolution of the UML. The way in which activity diagrams are modelled in the UML changed significantly between versions 1.4 and 2.1 of the specification. In the former, activities were defined as a special case of state machines, while in the latter they are defined atop a more general semantic base\footnote{A variant of generalised coloured Petri nets.} \cite{selic05uml2}.

The remainder of this section briefly introduces UML activity diagrams, describes their evolution, and discusses the way in which solutions were assessed. The work presented in this section is based on the case submitted to TTC 2010 \cite{rose10ttc_case}. 

\subsubsection{Activity Diagrams in UML}
Activity diagrams are used for modelling lower-level behaviours, emphasising sequencing and co-ordination conditions. They are used to model business processes and logic \cite{uml22}. Figure~\ref{fig:activity} shows an activity diagram for filling orders. The diagrams is partitioned into three \emph{swimlanes}, representing different organisational units. \emph{Activities} are represented with rounded rectangles and \emph{transitions} with directed arrows. \emph{Fork} and \emph{join} nodes are specified using a solid black rectangle. \emph{Decision} nodes are represented with a diamond. Guards on transitions are specified using square brackets. For example, in Figure~\ref{fig:activity} the transition to the restock activity is guarded by the condition \texttt{[not in stock]}. Text on transitions that is not enclosed in square brackets represents a trigger event. In Figure~\ref{fig:activity}, the transition from the restock activity occurs on receipt of the asynchronous signal called \texttt{receive stock}. Finally, the transitions between activities might involve interaction with objects. In Figure~\ref{fig:activity}, the Fill Order activity leads to an interaction with an object called \texttt{Filled Object}. 

\begin{figure}[htbp]
  \centering
  \includegraphics*[viewport=75 230 585 800,width=13cm]{6.Evaluation/images/activity.pdf}
  \caption{Activity model to be migrated.}
  \label{fig:activity}
\end{figure}

Between versions 1.4 and 2.2 of the UML specification, the metamodel for activity diagrams has changed significantly. The sequel summarises most of the changes, and details can be found in \cite{uml14} and \cite{uml22}.

\subsubsection{Evolution of Activity Diagrams}
Figures~\ref{fig:uml14} and \ref{fig:uml22} are simplifications of the activity diagram metamodels from versions 1.4 and 2.2 of the UML specification, respectively. In the interest of clarity, some features and abstract classes have been removed from Figures~\ref{fig:uml14} and \ref{fig:uml22}.

Some differences between Figures~\ref{fig:uml14} and \ref{fig:uml22} are: activities have been changed such that they comprise nodes and edges, actions replace states in UML 2.2, and the subtypes of control node replace pseudostates.

\begin{figure}[htbp]
  \centering
  \includegraphics[width=12cm]{6.Evaluation/images/uml14.pdf}
  \caption{UML 1.4 Activity Graphs (based on \cite{uml14}).}
  \label{fig:uml14}
\end{figure}
 

\begin{figure}[htbp]
  \centering
  \includegraphics[width=12cm]{6.Evaluation/images/uml22.pdf}
  \caption{UML 2.2 Activity Diagrams (based on \cite{uml22}).}
  \label{fig:uml22}
\end{figure}

To facilitate the comparison of solutions, the exemplar model shown in Figure~\ref{fig:activity} was used. Figure~\ref{fig:activity} is based on \cite[pg3-165]{uml14}. Solutions migrated the activity diagram shown in Figure~\ref{fig:activity} -- which conforms to UML 1.4 -- to conform to UML 2.2. The UML 1.4 model, the migrated UML 2.2 model, and the UML 1.4 and 2.2 metamodels are available from\footnote{\url{http://www.cs.york.ac.uk/~louis/ttc/}}.

Submissions were evaluated using the following four criteria, which were decided by the thesis author and the workshop organisers:

\begin{itemize}
	\item \textbf{Correctness}: Does the transformation produce a model equivalent to the migrated UML 2.2. model included in the case resources?
	\item \textbf{Conciseness}: How much code is required to specify the transformation? (In \cite{sprinkle04domain} et al. propose that the amount of effort required to codify migration should be directly proportional to the number of changes between original and evolved metamodel).
		\item \textbf{Clarity}: How easy is it to read and understand the transformation? (For example, is a well-known or standardised language?)
		\item \textbf{Extensions}: Which of the case extensions (described below) were implemented in the solution?
\end{itemize}

To further distinguish between solutions, three extensions to the core task were proposed. The first extension was added after the case was submitted, and was proposed by the workshop organisers and the solution authors. The second and third extension were included in the case by the thesis author. 

\subsubsection{Extension 1: Alternative Object Flow State Migration Semantics}
\label{sub:object_flow_states}
Following the submission of the case, discussion on the TTC forums\footnote{\url{http://planet-research20.org/ttc2010/index.php?option=com_community&view=groups&task=viewgroup&groupid=4&Itemid=150} (registration required)} revealed an ambiguity in the UML 2.2 specification indicating that the migration semantics for the ObjectFlowState UML 1.4 concept are not clear from the UML 2.2 specification.

\textbf{In the core task} described above, instances of ObjectFlowState were migrated to instances of ObjectNode. Any instances of Transition that had an ObjectFlowState as their source or target were migrated to instances of ObjectFlow. Listing~\ref{lst:ofs_to_node} shows an example application of this migration semantics. The top line of Listing~\ref{lst:ofs_to_node} shows instances of UML 1.4 metaclasses, include an instance of ObjectFlowState. The bottom line of Listing~\ref{lst:ofs_to_node} shows the equivalent UML 2.2 instances according to this migration semantics. Note that the Transitions, t1 and t2, are migrated to an instance of ObjectFlow. Likewise, the instance of ObjectFlowState, s2, is migrated to an instance of ObjectFlow.

\begin{lstlisting}[caption=Migrating Actions, label=lst:ofs_to_node]
s1:State <- t1:Transition -> s2:ObjectFlowState <- t2:Transition -> s3:State

s1:ActivityNode <- t1:ObjectFlow -> s2:ObjectNode <- t2:ObjectFlow -> s3:ActivityNode
\end{lstlisting}

\textbf{This extension} considered an alternative migration semantics for ObjectFlowState. For this extension, instances of ObjectFlowState (and any connected Transitions) were migrated to instances ObjectFlow, as shown by the example in Listing~\ref{lst:ofs_to_flow} in which the UML 2.2 ObjectFlow, f1, replaces t1, t2 and s2.

\begin{lstlisting}[caption=Migrating Actions, label=lst:ofs_to_flow]
s1:State <- t1:Transition -> s2:ObjectFlowState <- t2:Transition -> s3:State

s1:ActivityNode <- f1:ObjectFlow -> s3:ActivityNode
\end{lstlisting}

The alternative semantics were proposed on the TTC 2010 forums, and agreed as an extension to the core task by consensus between the solution authors and the workshop organisers. 

\subsubsection{Extension 2: Concrete Syntax}
\label{sub:concrete_syntax}
The second extension relates to the appearance of activity diagrams. The UML specifications provide no formally defined metamodel for the concrete syntax of UML diagrams. However, some UML tools store diagrammatic information in a structured manner using XML or a modelling tool. For example, the Eclipse UML 2 tools \cite{mdt_uml2} store diagrams as GMF \cite{gronback09emp} diagram models.

As such, submissions were invited to explore the feasibility of migrating the concrete syntax of the activity diagram shown in Figure~\ref{fig:activity} to the concrete syntax in their chosen UML 2 tool. To facilitate this, the case resources included an ArgoUML project\footnote{\url{http://argouml.tigris.org/}} containing the activity diagram shown in Figure~\ref{fig:activity}.

\subsubsection{Extension 3: XMI}
\label{sub:xmi}
The UML specifications indicate that UML models should be stored using XMI. However, because XMI has evolved at the same time as UML, UML 1.4 tools most likely produce XMI of a different version to UML 2.2 tools. For instance, ArgoUML produces XMI 1.2 for UML 1.4 models, while the Eclipse UML2 tools produce XMI 2.1 for UML 2.2.

As an extension to the core task, submissions were invited to consider how to migrate a UML 1.4 model represented in XMI 1.x to a UML 2.1 model represented in XMI 2.x. To facilitate this, the UML 1.4 model shown in Figure~\ref{fig:activity} was made available in XMI 1.2 as part of the case resources.

Following the submission of the case, solutions were encouraged to solve this extension by Tom Morris, the project leader for ArgoEclipse and a committer on ArgoUML. On the TTC forums, Morris stated that ``We have nothing available to fill this hole currently, so any contributions would be hugely valuable.  Not only would achieve academic fame and glory from the contest, but you'd get to see your code benefit users of one of the oldest (10+ yrs) open source UML modeling tools.'' \footnote{\url{http://www.planet-research20.org/ttc2010/index.php?option=com_community&view=groups&task=viewdiscussion&groupid=4&topicid=20&Itemid=150} (registration required)}

\subsection{Model Migration Solution in Epsilon Flock}
\label{subsec:ttc_solution}
This section discusses the Flock solution to the TTC case described above (the evolution of UML activity diagrams).  

The solution was developed in an iterative and incremental manner, using the following process:

\begin{enumerate}
	\item Change the Flock migration strategy.
	\item Execute Flock on the original model, producing a migrated model.
	\item Compare the migrated model with the reference model provided in the case resources.
	\item Repeat until the migrated and reference models were the same.
\end{enumerate}

The remainder of this section presents the Flock solution in an incremental manner. The code listings in this section show only those rules relevant to the iteration being discussed.

\subsubsection{Actions, Transitions and Final States}
Development of the migration strategy began by executing an empty Flock migration strategy on the original model. Because Flock automatically copies model elements that have not been affected by evolution, the resulting model contained \texttt{Pseudostates}s and \texttt{Transition}s, but none of the \texttt{ActionState}s from the original model. In UML 2.2 activities, \texttt{OpaqueAction}s replace \texttt{ActionState}s. Listing~\ref{lst:actions} shows the Flock code for changing \texttt{ActionState}s to corresponding \texttt{OpaqueAction}s.

\begin{lstlisting}[caption=Migrating Actions, label=lst:actions, language=Flock]
migrate ActionState to OpaqueAction
\end{lstlisting}

Next, similar rules were added to migrate instances of \texttt{FinalState} to instances of \texttt{ActivityFinalNode} and to migrate instances of \texttt{Transition} to \texttt{ControlFlow}, as shown in Listing~\ref{lst:final_states}.

\begin{lstlisting}[caption=Migrating FinalStates and Transitions, label=lst:final_states, language=Flock]
migrate FinalState to ActivityFinalNode
migrate Transition to ControlFlow
\end{lstlisting}

\subsubsection{Pseudostates}
Development continued by selected further types of state that were not present in the migrated model, such as \texttt{Pseudostates}s, which are not used in UML 2.2 activities. Instead, UML 2.2 activities use specialised \texttt{Node}s, such as \texttt{InitialNode}. Listing~\ref{lst:pseudostates} shows the Flock code used to change \texttt{Pseudostate}s to corresponding \texttt{Node}s.

\begin{lstlisting}[caption=Migrating Pseudostates, label=lst:pseudostates, language=Flock]
migrate Pseudostate to InitialNode  when: original.kind = Original!PseudostateKind#initial
migrate Pseudostate to DecisionNode when: original.kind = Original!PseudostateKind#junction
migrate Pseudostate to ForkNode     when: original.kind = Original!PseudostateKind#fork
migrate Pseudostate to JoinNode     when: original.kind = Original!PseudostateKind#join
\end{lstlisting}

\subsubsection{Activities}
In UML 2.2, \texttt{Activity}s no longer inherit from state machines. As such, some of the features defined by \texttt{Activity} have been renamed. Specifically, \texttt{tr\-an\-si\-ti\-o\-ns} has become \texttt{edges} and \texttt{paritions} has become \texttt{group}. Furthermore, the states (or nodes in UML 2.2 parlance) of an \texttt{Activity} are now contained in a feature called \texttt{nodes}, rather than in the \texttt{subvertex} feature of a composite state accessed via the \texttt{top} feature of \texttt{Activity}. The Flock migration rule shown in Listing~\ref{lst:activities} captured these changes.

\begin{lstlisting}[caption=Migrating ActivityGraphs, label=lst:activities, language=Flock]
migrate ActivityGraph to Activity {
	migrated.edge  = original.transitions.equivalent();
	migrated.group = original.partition.equivalent();
	migrated.node  = original.top.subvertex.equivalent();
}
\end{lstlisting}

Note that the rule in Listing~\ref{lst:activities} used the built-in \texttt{equivalent} operation to find migrated model elements from original model elements. As discussed in Section~\ref{sec:flock}, the \texttt{equivalent} operation invokes other migration rules where necessary and caches results to improve performance.

Next, a similar rule for migrating \texttt{Guard}s was added. In UML 1.4, the the \texttt{guard} feature of \texttt{Transition} references a \texttt{Guard}, which in turn references an \texttt{Ex\-pr\-es\-si\-on} via its \texttt{expression} feature. In UML 2.2, the \texttt{guard} feature of \texttt{Transition} references an \texttt{OpaqueExpression} directly. Listing~\ref{lst:guards} captures this in Flock.

\begin{lstlisting}[caption=Migrating Guards, label=lst:guards, language=Flock]
migrate Guard to OpaqueExpression {
	migrated.body.add(original.expression.body);
}

\end{lstlisting}


\subsubsection{Partitions}
In UML 1.4 activity diagrams, \texttt{Partition} specifies a single containment reference for its \texttt{contents}. In UML 2.2 activity diagrams, partitions have been renamed to \texttt{ActivityPartition}s and specify two containment features for their contents, \texttt{edges} and \texttt{nodes}. Listing~\ref{lst:partitions} shows the rule used to migrate \texttt{Partition}s to \texttt{ActivityPartition}s in Flock. The body of the rule shown in Listing~\ref{lst:partitions} uses the \emph{collect} operation to segregate the \texttt{contents} feature of the original model element into two parts.

\begin{lstlisting}[caption=Migrating Partitions, label=lst:partitions, language=Flock]
migrate Partition to ActivityPartition {
	migrated.edges = original.contents.collect(e:Transition | e.equivalent());
	migrated.nodes = original.contents.collect(n:StateVertex | n.equivalent());	
}
\end{lstlisting}


\subsubsection{ObjectFlows}
Finally, two rules were written for migrating model elements relating to object flows. In UML 1.4 activity diagrams, object flows are specified using \texttt{ObjectFlowState}, a subtype of \texttt{St\-at\-eVe\-rt\-ex}. In UML 2.2 activity diagrams, object flows are modelled using a subtype of \texttt{ObjectNode}. In UML 2.2 flows that connect to and from \texttt{ObjectNode}s must be represented with \texttt{ObjectFlow}s rather than \texttt{ControlFlow}s.

Listing~\ref{lst:objectflows} shows the Flock rule used to migrate \texttt{Transiton}s to \texttt{ObjectFlow}s. The rule applies for \texttt{Transition}s whose source or target \texttt{St\-at\-eVe\-rt\-ex} is of type \texttt{ObjectFlowState}.

\begin{lstlisting}[caption=Migrating ObjectFlows, label=lst:objectflows, language=Flock]
migrate ObjectFlowState to ActivityParameterNode

migrate Transition to ObjectFlow when: original.source.isTypeOf(ObjectFlowState) or original.target.isTypeOf(ObjectFlowState)
\end{lstlisting}

In addition to the core task, the Flock solution also approached two of the three extensions described in the case (Section~\ref{subsec:ttc_case}). The solutions to the extensions are now discussed.

\subsubsection{Alternative ObjectFlowState Migration Semantics}
The first extension required submissions to consider an alternative migration semantics for \texttt{ObjectFlowState}, in which a single \texttt{ObjectFlow} replaces each \texttt{ObjectFlowState} and any connected \texttt{Transition}s.

Listing~\ref{lst:objectflows2} shows the Flock source code used to migrate \texttt{ObjectFlowStates} (and connecting \texttt{Transitions}) to a single \texttt{ObjectFlow}. This rule was used instead of the two rules defined in Listing~\ref{lst:objectflows}. In the body of the rule shown in Listing~\ref{lst:objectflows2}, the \texttt{source} of the \texttt{Transition} is copied directly to the \texttt{source} of the \texttt{ObjectFlow}. The \texttt{target} of the \texttt{ObjectFlow} is set to the target of the first outgoing \texttt{Transition} from the \texttt{ObjectFlowState}. 

\begin{lstlisting}[caption=Migrating ObjectFlowStates to a single ObjectFlow, label=lst:objectflows2, language=Flock]
migrate Transition to ObjectFlow when: original.target.isTypeOf(ObjectFlowState) {
	migrated.source = original.source.equivalent();
	migrated.target = original.target.outgoing.first.target.equivalent();
}
\end{lstlisting}

Because, in this alternative semantics, \texttt{ObjectFlowState}s are represented as edges rather than nodes, the partition migration rule was changed such that \texttt{ObjectFlowState}s were not copied to the \texttt{nodes} feature of \texttt{Partition}s. To filter out the \texttt{ObjectFlowState}s, line 3 of Listing~\ref{lst:partitions} was changed to include a reject statement, as shown on line 3 of Listing~\ref{lst:partitions2}.

\begin{lstlisting}[caption=Migrating Partitions without ObjectFlowStates, label=lst:partitions2, language=Flock]
migrate Partition to ActivityPartition {
	migrated.edges = original.contents.collect(e:Transition | e.equivalent());
	migrated.nodes = original.contents.reject(ofs:ObjectFlowState | true).collect(n:Original!StateVertex | n.equivalent());	
}
\end{lstlisting}


\subsubsection{XMI}
\label{sec:xmi}
The second extension required submissions to migrate an activity graph conforming to UML 1.4 and encoded in XMI 1.2 to an equivalent activity graph conforming to UML 2.2 and encoded in XMI 2.1. The core task did not require submissions to consider changes to XMI (the model storage representation), but, in practice, this is a challenge to migration, as noted by Tom Morris on the TTC forums\footnote{\footnote{\url{http://www.planet-research20.org/ttc2010/index.php?option=com_community&view=groups&task=viewdiscussion&groupid=4&topicid=20&Itemid=150} (registration required)}}.

As discussed in Section~\ref{sec:flock}, Flock is built atop Epsilon, which includes a model connectivity layer (EMC). EMC provides a common interface for accessing and persisting models. Currently, EMC supports EMF (XMI 2.x), MDR (XMI 1.x), and plain XML models. To support migration between metamodels defined in heterogenous modelling frameworks, EMC was extended during the development of Flock to provide a conformance checking service.

Consequently, the migration strategy developed for the core task works for all of the types of model supported by EMC. To migrate a model encoded in XMI 1.2 rather than in XMI 2.1, the user must select a different option when executing the Flock migration strategy. Otherwise, no other changes are required.

\subsubsection{Results}
At the workshop, solutions to the migration case described in Section~\ref{subsec:ttc_case} were presented. Each solution was allocated two opponents who highlighted weaknesses of each approach. Following the solution presentations and opposition statements, each solution was scored using the four criteria described above, correctness, clarity, conciseness and number of extensions solved. Every workshop participants scored each solution on clarity and conciseness. The workshop organisers scored each solution on correctness and number of extensions solved, as these criteria could be measured objectively. Epsilon Flock was awarded first position for the migration case.

The opposition statements highlighted two weaknesses of Flock. Firstly, there is some duplicated code in Listing~\ref{lst:pseudostates}: the \texttt{migrate Pseudostate to X} statement appears several times. The duplication exists because Flock only allows one-to-one mappings between original and evolved metamodel types. The conservative copy algorithm would need to be extended to allow one-to-many mappings to remove this kind of duplication.

Secondly, the body of Flock rules are specified in an imperative manner. Consequently, reasoning about the correctness of the a migration strategy is more difficult than in languages that use a purely declarative syntax. \footnote{TODO: Need to expand on this. I probably need to explain why migration need not preserve semantics, but I don't really want to open that can of worms here.}

\subsection{Epsilon Flock in the Live Contest}
\label{subsec:ttc_live_contest}
TTC 2010 also invited the workshop participants to take part in a live contest. A problem was announced at the start of the workshop, and participants developed their solutions during the first day. The solutions were presented in the workshop and assessed in four categories. Flock was awarded first position for the \emph{exogenous transformation} category. The remainder of this section discusses the parts of the problem that relate to the exogenous transformation category and the Flock solution. 

The live contest problem required several model management operations be combined to perform beta-reduction of a simplified lambda calculus. Flock was used to specify one of the model management operations: model transformation between two similar (but not identical) metamodels. Flock was chosen rather than a new-target transformation language because the metamodels shared several classes and features. Using Flock allowed automatic copying of the model elements conforming to the classes common to both source and target metamodel. In other words, transformation rules were specified only for those parts of the metamodels that differed.

Flock was awarded first position by the workshop participants and organisers for the category in which it was entered. Participation in the live contest highlighted that, in addition to model migration, Flock can be used for specifying model transformation. In particular, Flock was appropriate because the source and target metamodels were similar (having several classes and features in common) and the conservative copy strategy reduced the number of rules required to specify the transformation.


% \begin{lstlisting}[caption=Flock migration strategy for the TTC 2010 live contest, label=lst:live_contest_flock, language=Flock]
% migrate Abstr {
% 	migrated.boundVar = new BoundVar;
% 	migrated.boundVar.name = original.bound;
% }
% 
% migrate Var to Ref {
% 	var migrated_var : Two!Var;
% 	
% 	if (original.isBound()) {
% 		migrated_var = original.bindingContext().equivalent().boundVar;
% 	} else {
% 		migrated_var = new Two!FreeVar;
% 		migrated_var.name = original.name;
% 	}
% 	
% 	migrated.refersTo = migrated_var;
% }
% 
% operation One!Var isBound() : Boolean {
% 	return self.bindingContext().isDefined();
% }
% 
% operation One!Var bindingContext() : One!Abstr {
% 	return One!Abstr.all.selectOne(a|a.bound == self.name);
% }
% \end{lstlisting}


\subsection{Summary}
This section has discussed the way in which Flock was evaluated by participating in the 2010 edition of the Transformation Tools Contest (TTC). Flock was assessed by application to an example of migration from the UML and comparison with eight other model and graph transformation tools. Flock was awarded first prize by the workshop participants and organisers. Additionally, Flock was used as part of a solution to a live contest developed during the workshop. The live contest highlighted that Flock is suitable for specifying some types of model transformation (in particular, those in which the source and target metamodel have common classes and features), as Flock was awarded first prize in the exogenous transformation category.

In addition to evaluating Flock, the work described in this section provides three further contributions. Firstly, the migration case submitted to TTC 2010, described in Section~\ref{subsec:ttc_case} provides a real-world example of co-evolution for use in future comparisons of model migration tools. The case is based on the evolution of UML, between versions 1.4 and 2.2. The migration strategy was devised by analysis of the UML specification, and by discussion between workshop participants.

Secondly, the Flock solution to the migration case (Section~\ref{subsec:ttc_solution}) demonstrates the way in which a migration strategy can be constructed using Flock. In particular, Section~\ref{subsec:ttc_solution} describes an iterative and incremental development process and inicates that an empty Flock migration strategy can provide a useful starting point for development.

Finally, Section~\ref{sec:flock} claims that Flock support several modelling technologies. The solution described in Section~\ref{subsec:ttc_solution} demonstrates the way in which Flock can be used to migrate models over two modelling technologies: MDR (XMI 1.x) and EMF (XMI 2.x). 


%!TEX root = /Users/louis/Documents/PhD/Deliverables/Thesis/thesis.tex

\subsection{Quantitive Comparison of Model Migration Languages}
In Section~\ref{sec:requirements_identification}, the following research requirement was identified: \emph{This thesis must implement and evaluate a domain-specific language for specifying and executing model migration strategies, comparing it to existing languages for specifying model migration strategies.} As discussed in Section~\ref{subsec:flock_implementation}, this thesis contributes Epsilon Flock, a domain-specific language for model migration. This section fulfils the second part of the above research requirement, comparing Flock with languages that are used in contemporary migration tools. 

In developer-driven migration, a programming language codifies the migration strategy. Because migration involves deriving the migrated model from the original, migration strategies typically access information from the original model and, based on that information, update the migrated model in some way. As such, migration is written in a language with constructs for accessing and updating the original and migrated models. Here, those language constructs are termed \textit{model operations}. Using examples of co-evolution, this section explores the variation in frequency of \emph{model operation} over different model migration languages, and discusses to what extent the results of this comparison can be used to assess the suitability of the languages considered for model migration.

As discussed in Chapter~\ref{Implementation}, the languages currently used for model migration vary. Model-to-model transformation languages are used in some migration tools (e.g. \cite{cicchetti08automating,garces09managing}); general-purpose languages in others (e.g. \cite{herrmannsdoerfer09cope,hussey06advanced}). Irrespective of the language used for migration, the way in which a migration tool relates original and migrated model elements falls into one of two categories: new- or existing-target, which were first introduced in Section~\ref{subsec:existing_migration_languages}. In the former, the migrated model is created afresh by the execution of the migration strategy. In the latter, the migrated model is initialised as a copy of the original model and then the migration strategy is executed.

Flock contributes a novel approach for relating original and migrated model elements, termed conservative copy. Conservative copy is a hybrid of new- and existing-target approaches. This section compares new-target, existing-target and conservative copy in the context of model migration. Section~\ref{subsubsec:quantitive_data} describes the data used in the comparison. The method for the comparison is discussed in Section~\ref{subsubsec:quantitive_method}. Section~\ref{subsubsec:quantitive_model_operations} identifies model operations for each of the migration languages used in the comparison, and Section~\ref{subsubsec:quantitive_results} presents and analysis the results.

\subsubsection{Data}
\label{subsubsec:quantitive_data}
Five examples of co-evolution were used to compare new-target, existing-target and conservative copy. This section briefly discusses the data used in the comparison.

\paragraph{Co-evolution Examples}
To remove one of the possible threats to the validity of the comparison, the examples used were distinct from those identified in Chapter~\ref{Analysis}, which were used to define requirements for Flock and conservative copy. The five examples used in this section are taken from three projects.

Two examples were taken from the \emph{Newsgroup} project, which performs statistical analysis of NNTP newsgroups and is developed by Dimitris Kolovos, a lecturer in this department. One example was taken from \emph{UML} (the Unified Modeling Language), an OMG specification of a language for modelling software systems. Two examples were taken from \emph{GMF} (Graphical Modeling Framework) \cite{gronback09emp}, an Eclipse project for generating graphical model editors.

\paragraph{Selection of Migration Languages}
As discussed above, there are two ways in which existing migration languages relate original and migrated model elements, new- and existing-target. Flock contributes a third way, conservative copy. For the comparison with Flock, one new- and one existing-target language was chosen.

The Atlas Transformation Language (ATL), a model-to-model transformation language has been used in \cite{cicchetti08automating,garces09managing} for model migration. As discussed in Section~\ref{subsec:existing_migration_languages}, model-to-model transformation languages support only new-target transformations for model migration\footnote{Because, in model migration, the source and target metamodels are not the same.}.

The author is aware of two approaches to migration that use existing-target transformations. In COPE \cite{herrmannsdoerfer09cope}, migration strategies are hand-written in Groovy when no co-evolutionary operator can be applied. As discussed in Section~\ref{subsec:existing_migration_languages}, COPE's Groovy migration strategies use an existing-target approach. COPE provides six operations for interacting with model elements, such as \texttt{set}, for changing the value of a feature, and \texttt{unset}, for removing all values from a feature. In the remainder of this section, the term \emph{Groovy-for-COPE} is used to refer to the combination of the Groovy programming language and the operators provided by COPE for use in hand-written migration strategies. In Ecore2Ecore \cite{hussey06advanced}, migration is performed when the original model is loaded, effectively an existing-target approach.

The comparison to Flock described in this section uses ATL to represent new-target approaches and Groovy-for-COPE to represent existing-target approaches. Groovy-for-COPE was preferred to Ecore2Ecore because the latter is not as expressive\footnote{Communication with Ed Merks, Eclipse Modeling Project leader, 2009, available at \url{http://www.eclipse.org/forums/index.php?t=tree&goto=486690&S=b1fdb2853760c9ce6b6b48d3a01b9aac}} and cannot be used for migration in the co-evolution examples considered in this section.

\subsubsection{Method}
\label{subsubsec:quantitive_method}
For each example of co-evolution, a migration strategy was written using each migration language (namely ATL, Groovy-for-COPE and Flock). The correctness of the migration strategy was assured by comparing the migrated models provided by the co-evolution example with the result of executing the migration strategy on the original models provided by the co-evolution example.

For each migration language, a set of model operations were identified, as described in Section~\ref{subsubsec:quantitive_model_operations}. A program was written to count the number of \emph{model operations} appearing in each migration strategy. The counting program was tested by writing migration strategies in each language for the co-evolution examples identified in Chapter~\ref{Analysis}.

There is one non-trivial threat to the validity of the comparison performed in this section. The author wrote the migration strategies for Flock (a migration language that the author developed) and for the other migration languages considered (which the author has not developed). Therefore, it is possible that the migration strategies written in the latter may contain more model operations than necessary. In some cases, it was possible to reduce the effects of this threat by re-using or adapting existing migration strategy code written by the migration language authors. This is discussed further in the sequel.

\subsubsection{Model Operations}
\label{subsubsec:quantitive_model_operations}
The variation in frequency of model operations was explored across three model migration languages, ATL, Groovy-for-COPE and Flock. Here, the model operations of each language are identified. In addition, the extent to which the comparison described in this section was able to use code written by the authors of each language is discussed.

The comparison described in this section counts two categories of model operation: copying operations, deletion operations. The former are used to assign values to elements of the migrated model, while the latter are used to remove values from elements of the migrated model.

\paragraph{Atlas Transformation Language (ATL)}
For the Atlas Transformation Language (ATL), the following model operations were counted:
	
\begin{itemize}
	\item Assignment to a feature:
	\subitem \texttt{<feature> <- <value>} 
\end{itemize}

Deletion operations are not used in new-target migration strategies. A new-target migration strategy specifies only those values that must appear in the migrated model and, unlike existing-target approaches and conservative copy, no values are copied automatically prior to the execution of the migration.

TODO discuss whether it was possible to use AML to generate ATL and hence reduce the impact of the threat to validity identified above.

\paragraph{Groovy-for-COPE}
For Groovy-for-COPE, the following model operations were counted:

\begin{itemize}
	\item Assignment to a feature:
	\subitem \texttt{<element>.<feature> = <value>}
	\subitem \texttt{<element>.<feature>.add(<value>)}
	\subitem \texttt{<element>.<feature>.addAll(<collection\_of\_values>)}
	\subitem \texttt{<element>.set(<feature>) = <value>}
	
	\item Unsetting a feature:
	\subitem \texttt{<element>.<feature>.unset()}	
	
	\item Removing a model element:
	\subitem \texttt{delete <element>}
\end{itemize}

Deletion operations (unset and remove above) are necessary for some existing-target migration strategies, because the migrated model (which is initialised as a copy of the original model) may contain data that is no longer captured in the evolved metamodel.

COPE provides a library of built-in, reusable co-evolutionary operators. Each co-evolutionary operator specifies a metamodel evolution along with a corresponding model migration strategy. For example, the ``Make Reference Containment'' operator evolves the metamodel such that a non-containment reference becomes a containment reference and migrates models such that the values of the evolved reference are replaced by copies.

As such, writing the Groovy migration strategy for the examples of co-evolution considered in this section involved, where possible, applying an appropriate COPE co-evolutionary operator and counting the number of model operations in the generated migration strategy. Not all examples could be completely specified using COPE co-evolutionary operator. In these cases, the Groovy migration strategy was written by the author.


\paragraph{Epsilon Flock}
Epsilon Flock, a transformation language tailored for model migration, was developed in this thesis and discussed in Chapter~\ref{Implementation}. Flock uses the Epsilon Object Language (EOL) \cite{kolovos06eol} to access and update model values. In addition, Flock defines \texttt{migrate} rules, which can be used to change the type of a model element. For Flock, the following model operations were counted:

\begin{itemize}
	\item Assignment to a feature:
	\subitem \texttt{<element>.<feature> := <value>} 
	\subitem \texttt{<element>.<feature>.add(<value>)}
	\subitem \texttt{<element>.<feature>.addAll(<collection\_of\_values>)}
	
	\item Removing a model element:
	\subitem \texttt{delete <element>}
\end{itemize}

Flock provides a remove operation but not an unset. The former is required to remove model elements that no longer conform to the target metamodel. The latter is not necessary because conservative copy will never copy to the migrated model any value that does not conform the evolved metamodel. 


\subsubsection{Results}
\label{subsubsec:quantitive_results}
By measuring the number of model operations in model migration strategies, the way in which each co-evolution approach relates original and migrated model elements was investigated. Five examples of model migration were measured to obtain the results shown in Table~\ref{tab:model_operations_results}. The results from measuring the examples identified from the analysis chapter are shown in Table~\ref{tab:model_operations_results_analysis_examples}.

Because the examples used to produce the measurements shown in Table~\ref{tab:model_operations_results_analysis_examples} were used to design Flock, the measurements in Table~\ref{tab:model_operations_results_analysis_examples} are less relevant to the evaluation presented here than the measurements shown in Table~\ref{tab:model_operations_results}. Nevertheless, the measurements made in Table~\ref{tab:model_operations_results_analysis_examples} are included in the interest of transparency, and because they were used to test the program which performed the measurements.

\begin{table}
	\centering
	\begin{tabular}{|r|c|c|c|}
		\hline
		                              & \multicolumn{3}{|c|}{\textbf{Migration Language}} \\
													  			& \multicolumn{3}{|c|}{Source-Target Relationship} \\
		\hline
		                              & \textbf{ATL} & \textbf{G-f-C} & \textbf{Flock} \\
		(Project) Example             & New & Existing & Conservative \\
		\hline
		\hline
		(Newsgroup) Extract Person    & 9  &  6  &  5  \\
		\hline                       
		(Newsgroup) Resolve Replies   &  8  &  3  &  2  \\
		\hline                       
		(UML) Activity Diagrams       &  15  &  15  &  8  \\
		\hline                       
		(GMF) Graph                   &  101  &  11  &  14  \\
		\hline                       
		(GMF) Gen2009                 &  310  &  16  &  16  \\
		\hline
	\end{tabular}
	\label{tab:model_operations_results}
	\caption{Model operation frequency (evaluation examples).}
\end{table}

\begin{table}
	\centering
	\begin{tabular}{|r|c|c|c|}
		\hline
		                              & \multicolumn{3}{|c|}{\textbf{Migration Language}} \\
													  			& \multicolumn{3}{|c|}{Source-Target Relationship} \\
		\hline
		                              & \textbf{ATL} & \textbf{G-f-C} & \textbf{Flock} \\
		(Project) Example             & New & Existing & Conservative \\
		\hline
		\hline
		(FPTC) Connections            & 6  & 6   & 3  \\
		\hline
		(FPTC) Fault Sets             & 7  & 5   & 3  \\
		\hline
		(GADIN) Enum to Classes       & 4  & 1   & 0  \\
		\hline
		(GADIN) Partition Cont        & 5  & 3   & 2  \\
		\hline
		(Literature) PetriNets        & 12  & 10   & 6  \\
		\hline
		(Newsgroup) Extract Person    & 9  & 6   & 5  \\
		\hline
		(Newsgroup) Resolve Replies   & 8  & 3   & 2  \\
		\hline
		(Process-Oriented) Split CP   & 8  & 1   & 1  \\
		\hline
		(Refactor) Cont to Ref        & 4  & 5   & 3  \\
		\hline
		(Refactor) Ref to Cont        & 3  & 4   & 3  \\
		\hline
		(Refactor) Extract Class      & 5  & 4   & 2  \\
		\hline
		(Refactor) Extract Subclass   & 6  & 0   & 0  \\
		\hline
		(Refactor) Inline Class       & 4  & 5   & 2  \\
		\hline
		(Refactor) Move Feature       & 6  & 2   & 1  \\
		\hline
		(Refactor) Push Down Feature  & 6  & 0 & 0  \\
		\hline
	\end{tabular}
	\label{tab:model_operations_results_analysis_examples}
	\caption{Model operation frequency (analysis examples).}
\end{table}

For all but one of the examples shown above, conservative copy requires less model operations than new-target and existing-target. For the majority of examples, no migration strategy specified with existing-target contained less model operations when encoded with new-target. These results are now investigated, starting by discussing the differences between the source-target relationships. Investigating the results led to the discovery of two limitations of the conservative copy implementation in Flock, relating to sub-typing and side-effects during initialisation. These limitations are also discussed below.

\paragraph{Source-Target Relationships}
New-target, existing-target and conservative copy initialise the migrated model in a different way. New-target initialises an empty model, while existing-target initialises a complete copy of the original model. Conservative copy initialises the migrated model by copying only those model elements from the original model that conform to the migrated metamodel.

New- and existing-target are opposites. In the former, explicit assignment operations must be used to copy values from original to migrated model for each feature that is not affected by the metamodel evolution. By contrast, in the latter unset operations must be used when the value of a feature should not have been copied.

In situations where a large number of metamodel features have not been affected by evolution, expressing migration with a new-target transformation language requires more model operations than using an existing-target transformation language. This is particularly noticeable in the GMF examples shown in Table~\ref{tab:model_operations_results}, where ATL requires many more model operations than Groovy-for-COPE and Flock.

In situations where a large number of metamodel features have been renamed, expressing migration with an existing-target transformation language requires more model operations than using a new-target transformation language. This is because, in an existing-target transformation language, two model operations (an unset and an assignment) are needed to migrate values in response to the renaming of a feature:

\texttt{<element>.<newFeature> = <element>.unset(<oldFeature>)}

By contrast, a new-target transformation language requires only one model operation (an assignment):

\texttt{<migrated\_element>.<feature> = <original\_element>.<feature>}

The UML (Table~\ref{tab:model_operations_results}) and Refactor Inline Class (Table~\ref{tab:model_operations_results_analysis_examples}) examples contained several feature renamings, and consequently the existing-target figure was nearer to the new-target figure than the conservative copy figure. This is contrary to the trend in Tables~\ref{tab:model_operations_results} and \ref{tab:model_operations_results_analysis_examples}.

Conservative copy is a hybrid of new- and existing target. Model values that have been affected by evolution are not copied to the migrated model, and so the migration strategy need not unset affected model values. Model values that have not been affected by evolution are copied to the migrated model, and so the migration strategy need not explicitly copy unaffected model values.

Two conclusions can be drawn from this discussion. Firstly, in general, less model operations are used when specifying a migration strategy with a conservative copy migration language than when specifying the same migration strategy with a new- or existing-target migration language. Secondly, in the examples studied here, there are often more features unaffected by metamodel evolution than affected. Consequently, specifying model migration with a new-target migration language requires more model operations than in an existing-target migration language for the examples shown in Tables~\ref{tab:model_operations_results} and ~\ref{tab:model_operations_results_analysis_examples}.

\paragraph{Subtyping}
The GMF Graph example shown in Table~\ref{tab:model_operations_results} is the one case where conservative copy requires more model operations than existing-target. Investigating this result revealed a limitation in conservative copy limitation in Flock, relating to the way in subtypes are migrated.

Figure~\ref{fig:subtyping} shows a simplified part of the GMF Graph metamodel prior to evolution. When the metamodel evolved, the type of the \texttt{figure} and \texttt{accessor} features were changed. Consequently, the migration strategy needed to change the values stored in the \texttt{figure} and \texttt{accessor} features. In the simplified example presented here, the type of the \texttt{figure} and \texttt{accessor} features was changed from string to integer. The intended migration semantics are for the integer value to be the length of the original string value. This is representative of the actual GMF Graph metamodel evolution.

\begin{figure}[htbp]
  \centering
  \includegraphics[scale=0.75]{6.Evaluation/images/subtyping.pdf}
  \caption{Simplified fragment of the GMF Graph metamodel.}
  \label{fig:subtyping}
\end{figure}

In ATL, the migration strategy for the metamodel evolution discussed above can be expressed using two model operations, because an ATL transformation rule may inherit the body of another. The \texttt{DiagramElements} rule on lines 1-4 of Listing~\ref{lst:graph_atl} specifies that the value of the figure feature should be the length of the original value. For \texttt{Node}s, \texttt{Connection}s and \texttt{Compartment}s, migration can be specified simply by extending the \texttt{DiagramElements} rule. For \texttt{DiagramLabel}s, the values of both the accessor and figure feature must be migrated. On lines X-Y of Listing~\ref{lst:graph_atl}, the \texttt{DiagramLabels} extends \texttt{Nodes} and hence \texttt{DiagramElements} to inherit the body of the latter for migrating figures. In addition, the \texttt{DiagramLabels} rule defines the migration for the value of the \texttt{accessor} feature.

\begin{lstlisting}[basicstyle=\ttfamily\footnotesize, flexiblecolumns=true, numbers=left, nolol=true, caption=Simplified GMF Graph model migration in ATL, label=lst:graph_atl, language=ATL, tabsize=2]
abstract rule DiagramElements {
	from o : Before!DiagramElement
	to   m : After!DiagramElement ( figure <- o.figure.length()	)
}

rule Nodes extends DiagramElements {
	from o : Before!Node
	to   m : After!Node
}

rule Connections extends DiagramElements {
	from o : Before!Connection
	to   m : After!Connection
}

rule Compartments extends DiagramElements {
	from o : Before!Compartment
	to   m : After!Compartment
}

rule DiagramLabels extends Nodes {
	from o : Before!DiagramLabel
	to   m : After!DiagramLabel (	accessor <- o.accessor.length()	)
}
\end{lstlisting}

In Groovy-for-COPE, the migration is similar to ATL. However, Groovy-for-COPE is entirely imperative, and so the migration, Listing~\ref{lst:graph_cope} is more concise than the ATL migration in Listing~\ref{lst:graph_atl}. In Listing~\ref{lst:graph_cope}, the loop iterates over each instance of \texttt{DiagramElement}, migrating the value of its figure feature (line 2). If the \texttt{DiagramElement} is also a \texttt{DiagramLabel} (line 4), the value of its accessor feature is also migrated (line 5).

\begin{lstlisting}[basicstyle=\ttfamily\footnotesize, flexiblecolumns=true, numbers=left, nolol=true, caption=Simplified GMF Graph model migration in COPE, label=lst:graph_cope, language=COPE, tabsize=2]
for (diagramElement in subtyping.DiagramElement.allInstances()) {
	diagramElement.figure = diagramElement.figure.length()
	
	if (subtyping.DiagramLabel.allInstances.contains(diagramElement)) {
		diagramElement.accessor = diagramElement.accessor.length()
	}
}
\end{lstlisting}

In both ATL and COPE, only 2 model operations are required for this migration: an assignment for each of the two features being migrated. However, the equivalent Flock migration strategy, shown in Listing~\ref{lst:graph_flock}, requires 5 model operations. In Flock, a migrate rule must be specified for each concrete subtype of \texttt{DiagramElement}. A \texttt{migrate DiagramElement} rule cannot be used because the semantics of Flock migrate rules state that, when no to part is specified, Flock will create an instance of the type named after the keyword migrate (\texttt{DiagramElement} here). Because \texttt{DiagramElement} is abstract, this will fail. Furthermore, because only one rule can be applied to each original model element, the \texttt{DiagramLabel} rule (lines 9-12) must migrate the values of both the figure and accessor features, and cannot exploit the kind of re-use provided by ATL with rule inheritance.

\begin{lstlisting}[basicstyle=\ttfamily\footnotesize, flexiblecolumns=true, numbers=left, nolol=true, caption=Simplified GMF Graph model migration in Flock, label=lst:graph_flock, language=Flock, tabsize=2]
migrate Compartment {
	migrated.figure := original.figure.length();
}

migrate Connection {
	migrated.figure := original.figure.length();
}

migrate DiagramLabel {
	migrated.figure   := original.figure.length();
	migrated.accessor := original.accessor.length();
}

migrate Node {
	migrated.figure := original.figure.length();
}
\end{lstlisting}

The example presented in this section highlights a limitation of the conservative copy algorithm as it is implemented in Flock. The extent to which this limitation can be addressed in Flock, and in general, is discussed in Section~\ref{sec:future_work}. This section now considers one further limitation of existing-target and conservative copy, relative to new-target.

\paragraph{Side-Effects during Initialisation}
The measurements observed for one of the examples of co-evolution from Chapter~\ref{Analysis}, Change Reference to Containment, cannot be explained by the conceptual differences between source-target relationship. Instead, the way in which the source-target relationship is implemented must be considered.

When a reference feature is changed to a containment reference during metamodel evolution, constructing the migrated model by starting from the original model (as is the case with existing-target and conservative copy) can have side-effects which complicate migration.

In the Change Reference to Containment example, a \texttt{System} initially comprises \texttt{Port}s and \texttt{Signature}s (Figure~\ref{fig:ref2cont_original_mm}). A \texttt{Signature} references any number of \texttt{ports}. The metamodel is to be evolved so that \texttt{Port}s can no longer be shared between \texttt{Signature}s.

\begin{figure}[htbp]
  \centering
  \includegraphics[scale=0.75]{6.Evaluation/images/change_ref_to_cont_before.pdf}
  \caption{Original metamodel.}
  \label{fig:ref2cont_original_mm}
\end{figure}

The evolved metamodel is shown in Figure~\ref{fig:ref2cont_evolved_mm}. \texttt{Signature}s now contain - rather than reference - \texttt{Port}s. Consequently, the \texttt{ports} feature of \texttt{System} is no longer required and is removed.

\begin{figure}[htbp]
  \centering
  \includegraphics[scale=0.75]{6.Evaluation/images/change_ref_to_cont_after.pdf}
  \caption{Evolved metamodel.}
  \label{fig:ref2cont_evolved_mm}
\end{figure}

The migration strategy is straightforward in a new-target migration language: for each \texttt{Signature} in the original model, each member of the \texttt{ports} feature is cloned, using a lazy rule, and added to the \texttt{ports} feature of the equivalent \texttt{Signature}.

\begin{lstlisting}[basicstyle=\ttfamily\footnotesize, flexiblecolumns=true, numbers=left, nolol=true, caption=Change R to C model migration in ATL, label=lst:ref2cont_atl, language=ATL, tabsize=2]
rule Systems {
	from
		o : Before!System
	to
		m : After!System ( signatures <- o.signatures )
}

rule Signature {
	from
		o : Before!Signature
	to
		m : After!Signature (
			ports <- o.ports->collect(p | thisModule.Port(p))
		)
}

lazy rule Port {
	from
		o : Before!Port
	to
		m : After!Port ( name <- o.name )
\end{lstlisting}

In existing-target and conservative copy migration languages, migration is less straightforward because the value of a containment reference (\texttt{Signature\#ports}) is set automatically by the migration strategy execution engine. When a containment reference is set, the contained objects are removed from their previous containment reference (i.e. setting a containment reference can have side-effects). Therefore, in a \texttt{System} where more than one \texttt{Signature} references the same \texttt{Port}, the migrated model cannot be formed by copying the contents of \texttt{Signature\#ports} from the original model. Attempting to do so causes each \texttt{Port} to be contained only in the last referencing \texttt{Signature} that was copied.

In existing-target migration languages, conformance is most likely only checked following the execution of the migration strategy, when the model is transformed to a metamodel-specific representation. Therefore, the containment nature of the reference is not enforced until after the migration strategy is executed. Hence, the migration strategy discussed here can be specified by unsetting the contents of the \texttt{ports} reference (line 4 of Listing~\ref{lst:ref2cont_cope}), and creating a copy of each referenced \texttt{Port} (lines 5-7 of Listing~\ref{lst:ref2cont_cope}).

Unlike the ATL migration strategy, the ports in the Groovy-for-COPE migration strategy are cloned in the same model as the original port. Consequently, the Groovy-for-COPE migration strategy must either only clone ports that are referenced by more than one signature or clone every referenced port, but delete all of the original ports. The latter approach requires 2 more model operations (to populate and delete the original ports) than the former (shown in Listing~\ref{lst:ref2cont_cope}).

\begin{lstlisting}[basicstyle=\ttfamily\footnotesize, flexiblecolumns=true, numbers=left, nolol=true, caption=Change R to C model migration in COPE, label=lst:ref2cont_cope, language=COPE, tabsize=2]
def contained = []

for(signature in refactorings_changeRefToCont.Signature.allInstances) {
  for(port in signature.ports)) {
	  // when more than one Signature references this port
	  if (contained.contains(port)) {
      def clone = Port.newInstance()
      clone.name = port.name
      signature.ports.add(clone)
      signature.ports.remove(port)
		} else {
			contained.add(port)
		}
  }
}

for(port in refactorings_changeRefToCont.Port.allInstances) {
	if (not refactorings_changeRefToCont.Signature.allInstances.any { it.ports.contains(port) }) {
	  	port.delete()
	}
}
\end{lstlisting}

In Flock, the containment nature of the reference is enforced when the migrated model is initialised. Because changing the contents of a containment reference can have side-effects, a \texttt{Port} that appears in the \texttt{ports} reference of a \texttt{Signature} in the original model may not have been automatically copied to the \texttt{ports} reference of the equivalent \texttt{Signature} in the migrated model during initialisation. Consequently, the migration strategy must check the \texttt{ports} reference of each migrated \texttt{Signature}, cloning only those \texttt{Port}s that have not be automatically copied during initialisation (see line 3 of Listing~\ref{lst:ref2cont_flock}).

\begin{lstlisting}[basicstyle=\ttfamily\footnotesize, flexiblecolumns=true, numbers=left, nolol=true, caption=Change R to C model migration in Flock, label=lst:ref2cont_flock, language=Flock, tabsize=2]
migrate Signature {
	for (port in original.ports) {
		if (migrated.ports.excludes(port.equivalent())) {
			var clone := new Migrated!Port;
			clone.name := port.name;
			migrated.ports.add(clone);
		}
	}
}

delete Port when: not Original!Signature.all.exists(s|s.ports.includes(original))
\end{lstlisting}

The Groovy-for-COPE and Flock migration strategies must also remove any \texttt{Port}s which are not referenced by any \texttt{Signature} (lines 17-21 of Listing~\ref{lst:ref2cont_cope}, and line 11 of Listing~\ref{lst:ref2cont_flock} respectively), whereas the ATL migration strategy, which initialises any empty migrated model, does not copy unreferenced \texttt{Port}s.

When a non-containment reference is changed to a containment reference, producing a corresponding migration strategy in Flock and Groovy-for-COPE requires the user to be aware of the side-effects that can occur during initialisation. It may be possible to extend the existing-target and conservative copy algorithms used in COPE and Flock, respectively, to automatically perform cloning when a reference is changed to be a containment reference. This is discussed further, for conservative copy, in Section~\ref{sec:future_work}.


\subsubsection{Summary}
By measuring frequency of model operations, this section has compared, in the context of model migration, three approaches to relating source-target relationship: new-target, existing-target and conservative copy. The results have been analysed and the measurement method described thoroughly.

The analysis of the measurements obtained has shown that a new- and existing-target migration languages are most suitable for specifying migration strategies for different types of migration language. New-target requires less model operations than existing-target when metamodel evolution involves the renaming of features. Conversely, existing-target requires less model operations than new-target when metamodel evolution does not affect most model elements. Conservative copy requires less model operations than both new- and existing-target in almost all of the examples studied here.

This section has highlighted two limitations of the conservative copy algorithm implemented in Epsilon Flock, and shown how these limitations are problematic for specifying some types of migration strategy. 

The author is not aware of any existing quantitive comparisons of migration languages, and, as such, the best practices for conducting such comparisons are not clear. The method used in obtaining these measurements has been described, in the hope that similar comparisons might be conducted in the future. 




\section{Discussion}
% We will discuss the limitations of our work, using for context the feedback of users, reviews of publications and scenarios from the case study discussed in Section~\ref{subsubsec:case_study}

\subsection{Threats to validity}



\section{Dissemination / Reception / ??}

\subsection{Publications}
% Publication in academic journals, and at international conferences and workshops ensure that our work is reviewed by experts, and is well-established and communicated in our field of research. So far, I have been the primary author for publications at one international conference (\cite{rose08hutn}), one European conference (\cite{rose08egl}), and one workshop (\cite{rose09patterns}). The first was published at MoDELS/UML, the leading international conference on model-driven engineering, in a year when it had a record number of submissions (274, 20\% acceptance), and has been nominated by HISE for the annual departmental award for best paper by a research student.

% We will submit our work to software evolution conferences, as well as at model-driven engineering conferences. Doing so will allow us to assess the impact of our research for a broader audience.


\subsection{Delivery through Eclipse}
% The tools produced as part of our research have been and will continue to be released as part of the Epsilon project, a member of the research incubator for the Eclipse Modeling Project (EMP), arguably the most active MDE community at present. EMP's research incubator hosts a limited number of participants, selected through a rigorous process. Contributions made to the incubator undergo regular technical review.

% Contributing to Epsilon allows us to deliver our research to the growing community \cite{kolovos08thesis} of Epsilon users.
