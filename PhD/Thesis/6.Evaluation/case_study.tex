%!TEX root = /Users/louis/Documents/PhD/Deliverables/Thesis/thesis.tex

\subsection{Case Study}
% We will apply our structures and processes to the Eclipse Generative Modelling Framework (GMF) project \cite{gronback06gmf}. GMF allows the definition of graphical concrete syntax for metamodels. GMF prescribes a model-driven approach: Users of GMF define concrete syntax as a model, which is used to generate a graphical editor. In fact, five models are used together to define a single editor using GMF.

% GMF defines the metamodels for graphical, tooling and mapping definition models; and for generator models. The metamodels have changed considerably during the development of GMF. Some changes have caused inconsistency with GMF models. Presently, migration is encoded in Java. Gronback has stated\footnote{Private communication, 2008.} that the migration code is being ported to QVT (a model-to-model transformation language) as the Java code is difficult to maintain.

% We identified GMF as the most appropriate candidate for the analysis phase of our research. Consequently, we decided to reserve GMF for the evaluation of our work.

