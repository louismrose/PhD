%!TEX root = /Users/louis/Documents/PhD/Deliverables/Thesis/thesis.tex

\chapter{Evaluation}
\label{Evaluation}

%The evaluation chapter will outline our evaluation method and results, including the impact and limitations of our research; and discuss the extent to which the requirements identified in the analysis chapter have been fulfilled. Evaluation will be conducted in three ways: application of our structures and processes in a case study; publication of our research in academic journals, international conferences and workshops; and assessing the contribution made when delivering our work through an Eclipse research incubation project.

\section{Case Study}
% We will apply our structures and processes to the Eclipse Generative Modelling Framework (GMF) project \cite{gronback06gmf}. GMF allows the definition of graphical concrete syntax for metamodels. GMF prescribes a model-driven approach: Users of GMF define concrete syntax as a model, which is used to generate a graphical editor. In fact, five models are used together to define a single editor using GMF.

% GMF defines the metamodels for graphical, tooling and mapping definition models; and for generator models. The metamodels have changed considerably during the development of GMF. Some changes have caused inconsistency with GMF models. Presently, migration is encoded in Java. Gronback has stated\footnote{Private communication, 2008.} that the migration code is being ported to QVT (a model-to-model transformation language) as the Java code is difficult to maintain.

% We identified GMF as the most appropriate candidate for the analysis phase of our research. Consequently, we decided to reserve GMF for the evaluation of our work.


\section{Publications}
% Publication in academic journals, and at international conferences and workshops ensure that our work is reviewed by experts, and is well-established and communicated in our field of research. So far, I have been the primary author for publications at one international conference (\cite{rose08hutn}), one European conference (\cite{rose08egl}), and one workshop (\cite{rose09patterns}). The first was published at MoDELS/UML, the leading international conference on model-driven engineering, in a year when it had a record number of submissions (274, 20\% acceptance), and has been nominated by HISE for the annual departmental award for best paper by a research student.

% We will submit our work to software evolution conferences, as well as at model-driven engineering conferences. Doing so will allow us to assess the impact of our research for a broader audience.


\section{Delivery through Eclipse}
% The tools produced as part of our research have been and will continue to be released as part of the Epsilon project, a member of the research incubator for the Eclipse Modeling Project (EMP), arguably the most active MDE community at present. EMP's research incubator hosts a limited number of participants, selected through a rigorous process. Contributions made to the incubator undergo regular technical review.

% Contributing to Epsilon allows us to deliver our research to the growing community \cite{kolovos08thesis} of Epsilon users.


\section{Limitations}
% We will discuss the limitations of our work, using for context the feedback of users, reviews of publications and scenarios from the case study discussed in Section~\ref{subsubsec:case_study}.
