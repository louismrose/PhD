%!TEX root = /Users/louis/Documents/PhD/Deliverables/Thesis/thesis.tex

\section{Textual Modelling Notation}
\label{sec:notation}
The analysis of co-evolution examples in Chapter~\ref{Analysis} highlighted two ways in which co-evolution is managed. In \emph{developer-driven} co-evolution, migration is specified by the metamodel developer in an executable format; while in \emph{user-driven co-evolution} migration is specified by the metamodel developer in prose or not at all. Performing user-driven co-evolution with modelling frameworks presents two key challenges that have not been explored by existing research. Firstly, user-driven co-evolution often involves editing the storage representation of the model, such as XMI. Model storage representations are typically not optimised for human use and hence user-driven co-evolution can be error-prone. Secondly, non-conformant model elements must be identified during user-driven co-evolution. When a multi-pass parser is used to load models, as is the case with EMF, not all conformance problems are reported at once, and user-driven co-evolution is an iterative process. In Section~\ref{sec:requirements_identification}, these challenges led to the identification of the following requirement: \emph{This thesis must demonstrate a user-driven co-evolution process that enables the editing of non-conformant models without directly manipulating the underlying storage representation and provides a sound and complete conformance report for the original model and evolved metamodel.}

The remainder of this section describes a textual notation for models, which has been implemented for EMF, and discusses the way in which the notation has been integrated with the metamodel independent syntax described in Section~\ref{sec:mmi_syntax} to produce conformance reports. 

\subsection{Model Migration with XMI}
The co-evolution example from Section~\ref{sec:mmi_syntax} is now used to illustrate the way in which model migration is performed by editing the underlying storage representation of a model, such as XMI (Section~\ref{subsec:mof}). Consider again the evolution of the families metamodel (Figure~\ref{}) and a model conforming to the original metamodel (Figure~\ref{}).

\begin{figure}[htbp]
	\centering
	\subfigure[Original metamodel.]
	{
	    \label{fig:original_families_mm_repeated}
	    \includegraphics[scale=0.9]{5.Implementation/images/families.pdf}
	}
	\subfigure[Evolved metamodel.]
	{
	    \label{fig:evolved_families_mm_repeated}
	    \includegraphics[scale=0.9]{5.Implementation/images/families_evolved.pdf}
	}
	\caption{Evolution of a families metamodel, based on the metamodel in \cite{hutn}.}
\label{fig:families_mms_repeated}
\end{figure}

\begin{figure}[htbp]
  \begin{center}
    \leavevmode
    \includegraphics[width=10cm]{5.Implementation/images/family_model.png}
  \end{center}
  \caption[A family model]{A family model, which conforms to the metamodel in Figure~\ref{fig:original_families_mm_repeated}}
  \label{fig:families_model_repeated}
\end{figure}

The model in Figure~\ref{fig:families_model_repeated} does not conform to the evolved metamodel (because it uses the \texttt{na\-tu\-ralCh\-il\-dr\-en} and \texttt{ad\-op\-t\-edCh\-il\-dr\-en} features, which are not defined for \texttt{Pe\-rs\-on}), and hence cannot be loaded by the modelling framework. Migration might be achieved by editing the underlying storage representation directly (i.e. manually manipulating XMI). Listing~\ref{lst:xmi} shows the XMI for the model in Figure~\ref{fig:families_model_repeated}.

\begin{lstlisting}[caption=XMI for the family model in Figure~\ref{fig:families_model_repeated}, label=lst:xmi, language=XML]
<?xml version="1.0" encoding="ASCII"?>
<families:Family xmi:version="2.0" xmlns:xmi="http://www.omg.org/XMI" xmlns:families="families" xmi:id="_kE2LkAagEeC-FIOYrvUj0A" name="Smiths">
  <naturalChildren xmi:id="_q8RWYAagEeC-FIOYrvUj0A" name="Paul"/>
  <adoptedChildren xmi:id="_nj6TcAagEeC-FIOYrvUj0A" name="John"/>
</families:Family>
\end{lstlisting}

XMI is a concrete syntax for models, which has been optimised for use by machines and not by humans \cite{hutn}. Models often contain information that is not relevant to the domain, such as the universally unique identifiers (xmi:id attributes) on lines 2, 3 and 4 of Listing~\ref{lst:xmi}. Furthermore, information is often omitted to reduce the size of the model on disk. For example, the model elements on lines 3 and 4 of Listing~\ref{lst:xmi} do not specify their type (\texttt{Pe\-rs\-on}) and this is inferred from the type of the \texttt{na\-tu\-ralCh\-il\-dr\-en} and \texttt{ad\-op\-t\-edCh\-il\-dr\-en} features (which is accessed by the modelling framework via the XML namespace import for the families metamodel on line 2). A model references its metamodel  These issues affect the usability of XMI, and the evaluation presented in Section~\ref{sec:exemplar_user-driven_co-evo} further explores the suitability of XMI for user-driven co-evolution. The remainder of this section discusses the design and implementation of a syntax that provides an alternative to XMI.

\subsection{Potential Alternatives to XMI}
Two characteristics were considered when designing a notation that provides an alternative to representing models with XMI. Models can be represented textually or graphically (Section~\ref{subsec:modelling_languages}), and using a metamodel-specific or a metamodel-independent syntax (Section~\ref{sec:mmi_syntax}). The benefits and drawbacks of each option have been considered particularly with respect to their implications for user-driven co-evolution, and are now discussed.

\paragraph{Metamodel-independent vs metamodel-specific} A metamodel-specific syntax is defined in terms from the metamodel, and is often more concise than a metamodel-independent syntax. A metamodel-specific (and textual) syntax for part of the evolved families metamodel (Figure~\ref{}) is shown in Listing~\ref{}. Notice that the syntax is defined in metamodel terms, such as \texttt{Fa\-mi\-ly} and \texttt{ch\-il\-dr\-en}. Consequently, the syntax definition can be affected by metamodel evolution, and hence cannot be used to load a model that does not conform to its metamodel. As the alternative syntax was to be used during user-driven co-evolution (to manipulate models that do not conform to their metamodel), a metamodel-independent representation was preferred to a metamodel-specific representation. 

\paragraph{Textual vs graphical} For user-driven co-evolution, the usability of the notation is important because a metamodel user manipulates models with the notation to perform migration. The choice between a textual or graphical notation likely has a significant impact on usability, but it was not feasible to conduct a thorough user analysis given the time constraints of the thesis. Instead, a textual notation was selected (to reduce implementation effort). The textual notation was implemented such that the addition of an equivalent graphical notation could be added in future work. 

Currently, several tools exist for representing models with textual, metamodel-specific syntaxes (such as the text-to-model transformation tools discussed in Section~\ref{subsubsec:model_transformation}), but no tools exist for representing models in a metamodel-independent syntax other than XMI. \cite{steel01hutn} describe the Distributed Systems Technology Centre's TokTok project, which provided a human-usable textual modelling notation, and is now inactive. The metamodel-independent representation described in \cite{muller05hutn} has been abandoned in favour of Sintaks\footnote{\url{http://www.kermeta.org/sintaks/}}, a tool for constructing metamodel-specific representations. However, the metamodel-independent representations described in \cite{steel01hutn,muller05hutn} were both based on an OMG standard, Human-Usable Textual Notation (HUTN) \cite{hutn}, which defines a textual modelling notation that aims to conform to human-usability criteria \cite{hutn}. As a metamodel-independent, textual concrete syntax, HUTN was seen as an ideal starting point for designing a textual modelling notation for use in user-driven co-evolution.

\subsection{OMG Human-Usable Textual Notation}
\label{subsec:hutn}
The HUTN specification states its primary design goal as human-usability and ``this is achieved through consideration of the successes and failures of common programming languages'' \cite[Section 2.2]{hutn}. The HUTN specification refers to two studies of programming language usability to justify design decisions, but, because no reference implementation exists, the OMG specification does not evaluate the human-usability of the notation. As HUTN is optimised for human-usability, using HUTN rather than XMI for user-driven co-evolution should lead to increased developer productivity. This claim is explored in Chapter~\ref{Evaluation}.

Like the generic metamodel presented in Section~\ref{sec:mmi_syntax}, HUTN is a metamodel-independent syntax for MOF. However, the OMG HUTN specification focuses on concrete syntax, whereas the metamodel-independent syntax presented in Section~\ref{sec:mmi_syntax} focuses on abstract syntax. In this section, the key features of HUTN are introduced, and the sequel presents a new reference implementation of HUTN. Throughout the remainder of this section, the original families metamodel (Figure~\ref{fig:original_families_mm_repeated}) is used to illustrate the notation.


\subsubsection{Basic Notation}
Listing \ref{lst:attributes} shows the construction of an \emph{object} (an instance of a metamodel class) in OMG HUTN, here an instance of the Family class from Figure \ref{fig:example-mm}. Line 1 specifies the metamodel \emph{package} containing the metamodel classes that can be instantiated by this model (\texttt{Fa\-mi\-lyPa\-ck\-a\-ge}). A package declaration in OMG HUTN is equivalent to a namespace import at the start of an XMI document (e.g. line 2 of Listing~\ref{lst:xmi}). In Listing~\ref{lst:attributes}, line 2 names the metamodel class to be instantiated (\texttt{Fa\-mi\-ly}) and gives an identifier for the object (\texttt{The Sm\-it\-hs}). Lines 3 to 7 define \emph{attribute values}; in each case, the data value is assigned to the attribute with the specified name. The encoding of the value depends on its type: strings are delimited by any form of quotation mark; multi-valued attributes use comma separators, etc.

The metamodel in Figure \ref{fig:original_families_mm_repeated} has a \emph{simple reference} (\texttt{fa\-mi\-lyFr\-ie\-n\-ds}) and two \emph{containment references} (\texttt{ad\-op\-t\-edCh\-il\-dr\-en}; \texttt{na\-tu\-r\-alCh\-il\-dr\-en}). The OMG HUTN representation embeds a contained object directly in the parent object, as shown in Listing \ref{lst:containment}. A simple reference can be specified using the type and identifier of the referred object, as shown in Listing \ref{lst:non-contained}. Like attribute values, both styles of reference are preceded by the name of the meta-feature.

\begin{lstlisting}[caption={[Specifying attributes with HUTN]Specifying attributes with HUTN, taken from \cite{rose08hutn}}, label=lst:attributes, language=HutnFamilies]
FamilyPackage "families" {
    Family "The Smiths" {
        nuclear: true
        name: "The Smiths"
        averageAge: 25.7
        numberOfPets: 2
        address: "120 Main Street", "37 University Road"
    }
}
\end{lstlisting}

\begin{lstlisting}[caption={[Specifying a containment reference with HUTN]Specifying a containment reference with HUTN, taken from \cite{rose08hutn}}, label=lst:containment, language=HutnFamilies]
FamilyPackage "families" {
    Family "The Smiths" {
        naturalChildren: Person "John" { name: "John" },
                                Person "Jo" { gender: female }
    }
}
\end{lstlisting}


\begin{lstlisting}[caption={[Specifying a simple reference with HUTN]Specifying a simple reference with HUTN, taken from \cite{rose08hutn}}, label=lst:non-contained, language=HutnFamilies]
FamilyPackage "families" {
    Family "The Smiths" {
        familyFriends: Family "The Does"
    }
    Family "The Does" {}
}
\end{lstlisting}


\subsubsection{Keywords and Adjectives}
In general, a metamodel-independent syntax (such as OMG HUTN) will not be as concise as a metamodel-specific concrete syntax. However, OMG HUTN does define optional syntactic shortcuts to make model specifications more compact. The OMG HUTN specification aims to make the syntactic shortcuts intuitive \cite[pg2-4]{hutn}.

Two of the syntactic shortcuts relate to Boolean-valued attributes and are now discussed; a complete list of syntactic shortcuts is given in \cite{hutn}. OMG HUTN permits the use of an attribute name to represent the value \texttt{true}, or the attribute name prefixed with a tilde to represent the value \texttt{false}). When used in the body of the object, this style of Boolean-valued attribute represents a \emph{keyword}. A keyword used to prefix an object declaration is called an \emph{adjective}. Listing \ref{lst:boolean} shows the use of both an attribute keyword (\texttt{\textasciitilde nuclear} on line 6) and adjective (\texttt{\textasciitilde migrant} on line 2), and states that \texttt{The Sm\-it\-hs} are \texttt{mi\-gr\-a\-nt} and that \texttt{The Do\-es} are not \texttt{nu\-cl\-e\-ar}.

\begin{lstlisting}[caption={[Using keywords and adjectives in HUTN]Using keywords and adjectives in HUTN, taken from \cite{rose08hutn}}, label=lst:boolean, language=HutnFamilies]
FamilyPackage "families" {
    migrant Family "The Smiths" {}

    Family "The Does" {
        averageAge: 20.1
        ~nuclear
        name: "The Does"
    }
}
\end{lstlisting}


% \subsubsection{Inter-Package References}
% \label{subsubsec:inter-package_references}
% To conclude the summary of the notation, two advanced features defined in the HUTN specification are discussed. The first enables objects to refer to other objects in a different package, while the second provides means for specifying the values of a reference for all objects in a single construct (which can be used, in some cases, to simplify the specification of complicated relationships).
% 
% \begin{lstlisting}[caption=Referencing objects in other packages with HUTN., label=lst:fullyqualified, language=HutnFamilies]
% FamilyPackage "families" {
%     Family "The Smiths" {}
% }
% VehiclePackage "vehicles" {
%     Vehicle "The Smiths' Car" {
%         owner: FamilyPackage.Family "families"."The Smiths"
%     }
% }
% \end{lstlisting}
% 
% To reference objects between separate package instances in the same document, the package identifier is used to construct a fully-qualified name. Suppose a second package is introduced to the metamodel in Figure \ref{fig:example-mm}. Among other concepts, this package introduces a Vehicle class, which defines an owner reference of type Family. Listing \ref{lst:fullyqualified} illustrates the way in which the owner feature can be populated. Note that the fully-qualified form of the class utilises the names of elements of the metamodel, while the fully-qualified form of the object utilises only HUTN identifiers defined in the current document.
% 
% The HUTN specification defines name scope optimisation rules, which allow the definition above to be simplified to: \texttt{owner: Family "The Smiths"}, assuming that the VehiclePackage does not define a Family class, and that the identifier ``The Smiths'' is not used in the VehiclePackage block.


\subsubsection{Alternative Reference Syntax}
In addition to the syntax defined in Listings \ref{lst:containment} and \ref{lst:non-contained}, OMG HUTN defines two alternative syntactic constructs for specifying the value of references. For example, Listing \ref{lst:assocblock} demonstrates the use of a reference block for defining \texttt{The Does} as friends with both \texttt{The Smiths} and \texttt{The Bloggs}.

\begin{lstlisting}[caption={[Using a reference block in HUTN]Using a reference block in HUTN, taken from \cite{rose08hutn}}, label=lst:assocblock, language=HutnFamilies]
FamilyPackage "families" {
    Family "The Smiths" {}
    Family "The Does" {}
    Family "The Bloggs" {}
    
    familyFriends {
        "The Does" "The Smiths"
        "The Does" "The Bloggs"
    }
}
\end{lstlisting}

Listing \ref{lst:associnfix} illustrates a further alternative syntax for references, which employs an infix notation. 

\begin{lstlisting}[caption={[Using an infix reference in HUTN]Using an infix reference in HUTN, taken from \cite{rose08hutn}}, label=lst:associnfix, language=HutnFamilies]
FamilyPackage "families" {
    Family "The Smiths" {}
    Family "The Does" {}
    Family "The Bloggs" {}
    
    Family "The Smiths" familyFriends Family "The Does";
    Family "The Smiths" familyFriends Family "The Bloggs";
}
\end{lstlisting}

The reference block (Listing~\ref{lst:assocblock}) and infix (Listing~\ref{lst:associnfix}) notations are syntactic variations on -- and have identical semantics to -- the reference notation shown in Listings \ref{lst:containment} and \ref{lst:non-contained}.


\subsubsection{Customisation via Configuration}
The OMG HUTN specification allows some limited, metamodel-specific customisation of the notation, using \emph{configuration files}. Customisations include a parametric form of object instantiation; renaming of metamodel elements; specifying the default value of a feature; and providing a default identifier for classes of object.


\subsection{Reference Implementation: Epsilon HUTN}
\label{subsec:epsilon_hutn}
To investigate the extent to which OMG HUTN can be used for user-driven co-evolution, an implementation, Epsilon HUTN, has been designed and implemented. This section describes the way in which Epsilon HUTN was implemented using a combination of model-management operations. From text conforming to the OMG HUTN syntax (described above), Epsilon HUTN produces an equivalent model that can be managed with EMF (Section~\ref{subsec:emf}). The sequel demonstrates the way in which Epsilon HUTN can be used for user-driven co-evolution.

\subsubsection{Design of Epsilon HUTN}
Implementing OMG HUTN involved building a tool for producing an EMF model (i.e. a model represented in XMI) from text conforming to the OMG HUTN syntax (described above). Essentially then, Epsilon HUTN can be regarded as a parser (that emits models), or as a text-to-model transformation. Several approaches to constructing Epsilon HUTN were considering, including: using a text-to-model (T2M) transformation tool (Section~\ref{subsubsec:model_transformation}), using a domain-specific language (DSL) framework (Section~\ref{subsec:dsls}), and using MDE tools and techniques such as EMF (Section~\ref{subsec:emf}), Epsilon (Section~\ref{subsec:epsilon}) and metamodelling.

As was the case for the design and implementation of the metamodel-independent syntax (Section~\ref{}), the author preferred to avoid dependencies on tools that were not part of the Eclipse Modelling Project (in order not to complicate installation of the notation for users). In 2008, the Eclipse Modeling Project\footnote{\url{http://www.eclipse.org/modeling/}} did not provide a standard T2M language or DSL framework, and so these implementation strategies were discounted.

Instead, Epsilon HUTN was constructed using existing languages of the Epsilon platform. To parse HUTN source, a parser was generated with the ANTLR parser generator tool \cite{parr07antlr}, which had been used successfully to implement parsers for the other task-specific languages of Epsilon. A parser generated with ANTLR emits an abstract syntax tree (a set of Java objects that conform to a simple tree data structure), from which the Epsilon HUTN tool needs to produce an EMF model.

The abstract syntax tree produced by ANTLR can be regarded as a model (conforming to the metamodel in Figure~\ref{fig:ast_metamodel}) and hence, producing an EMF model from the abstract syntax tree can be regarded as a model-to-model transformation. Epsilon HUTN, however, was designed as two separate transformations, for two reasons. Firstly, initial prototyping highlighted that the difference between a model represented in terms of the tree metamodel in Figure~\ref{fig:ast_metamodel} and the same model represented in metamodel-specific terms is vast, and the logic required to perform a one-step transformation quickly became complicated even for simple models. In particular, each transformation rule would have required a lengthly guard statement, which would have been difficult to debug and maintain. Secondly, it became apparent that the concrete syntax defined in OMG HUTN could be transformed to the metamodel-independent syntax defined in Section~\ref{}, which would reduce implementation effort by re-using the conformance checking service.

\subsubsection{Implementation of Epsilon HUTN}
For the reasons outline above, Epsilon HUTN is implemented using two model-to-model transformations. Figure \ref{fig:architecture} outlines the workflow through Epsilon HUTN, from HUTN source text to an EMF instantiation of the target model. The HUTN model specification is parsed to an abstract syntax tree using a HUTN parser specified in ANTLR \cite{parr07antlr}. From this, a Java postprocessor is used to construct an instance of the simple AST metamodel in Figure~\ref{fig:ast_metamodel}. Using ETL, a M2M transformation is applied to produce an intermediate model, which is an instance of the metamodel-independent syntax discussed in Section~\ref{sec:mmi_syntax}. Validation is performed on the intermediate model to ensure that the syntactic constraints specified in the OMG HUTN specification are satisfied\footnote{For example, no two objects may have the same identifier.}, and that the model conforms to the target metamodel. Conformance checking is achieved by re-using the service presented in Section~\ref{sec:mmi_syntax}. Finally, a M2T transformation on the target metamodel, specified in EGL, produces a further M2M transformation, which consumes the intermediate model and produces the target model\footnote{This final step involves a higher-order transformation (EGL is used to produce a transformation in ETL), and is described in more detail below.}.

\begin{figure}[htbp]
  \begin{center}
    \leavevmode
    \includegraphics[scale=0.44]{5.Implementation/hutn_workflow.png}
  \end{center}
  \caption{The architecture of Epsilon HUTN.}
  \label{fig:architecture}
\end{figure}

The modular architecture in Figure~\ref{fig:architecture} facilitates the re-use of the metamodel-independent syntax and conformance checking service described in Section~\ref{sec:mmi_syntax}, and hence reduced implementation effort. A small modification was made to the metamodel-independent syntax to facilitate the implementation of Epsilon HUTN: an additional metaclass, \texttt{Pa\-ck\-a\-geOb\-je\-ct}, was added to the metamodel-independent syntax. In OMG HUTN, packages are used to segregate a model such that different parts of a OMG HUTN document can refer to different metamodels. Consequently, a \texttt{Pa\-ck\-a\-geOb\-je\-ct} has a type (i.e. the metamodel to which is contents refer), an optional identifier (used for inter-package references) and contains any number of \texttt{Ob\-je\-ct}s. To avoid confusion with \texttt{Pa\-ck\-a\-geOb\-je\-ct}, the \texttt{Ob\-je\-ct} class in the metamodel-independent syntax was renamed to \texttt{Cl\-a\-ssOb\-je\-ct}. The version of the metamodel-independent syntax used with Epsilon HUTN is shown in Figure~\ref{fig:mmi_syntax_hutn}.

\begin{figure}[htbp]
  \centering
  \includegraphics[width=12cm]{5.Implementation/images/slot_model_complete_and_final.pdf}
  \caption{Final version of the metamodel-independent syntax, in Ecore}
  \label{fig:mmi_syntax_hutn}
\end{figure}

Each phase of the architecture in Figure~\ref{fig:mmi_syntax_hutn} is now discussed in detail. Note that, in this section, instances of the metamodel-independent syntax producing during the execution of the HUTN workflow are termed an \textit{intermediate model}.

\paragraph{Parsing the HUTN Source}
A parser for OMG HUTN was constructed using ANTLR \cite{parr07antlr}, a parser generator tool. ANTLR produces a parser from an annotated EBNF grammar definition. An extract of the grammar definition used by Epsilon HUTN is shown in Listing~\ref{lst:hutn_grammar}. Epsilon HUTN uses a simple, bespoke Java post-processor to construct instances of the abstract syntax tree metamodel (Figure~\ref{}) from the Java objects produced by ANTLR. Specifically, the post-processor copies the Java objects produced by the parser into an EMF resource, and hence produces a model that can be managed with EMF.

\paragraph{AST Model to Intermediate Model}
Epsilon HUTN uses ETL \cite{kolovos08etl} for specifying M2M transformation. One of the transformation rules from Epsilon HUTN is shown in Listing \ref{lst:m2m}. The rule transforms a name node in the AST model (which could represent a package or a class object) to a package object in the intermediate model. The guard (line 5) specifies that a name node will only be transformed to a package object if the node has no parent (i.e. it is a top-level node, and hence a package rather than a class). The body of the rule states that the type, line number and column number of the package are determined from the text, line and column attributes of the node object. On line 11, a containment slot is instantiated to hold the children of this package object. The children of the node object are transformed to the intermediate model (using a built-in method, \verb|equivalent()|), and added to the containment slot.

\begin{lstlisting}[caption=Transforming Nodes to PackageObjects with ETL., label=lst:m2m, language=ETL]
rule NameNode2PackageObject
    transform n : AntlrAst!Node
    to p : Intermediate!PackageObject {

    guard : n.type == 'Name' and n.parent.isUndefined()

    p.type := n.text;
    p.line := n.line;
    p.col  := n.column;

    var slot := new Intermediate!ContainmentSlot;
    for (child in n.children) {
        slot.objects.add(child.equivalent());
    }
    if (slot.objects.notEmpty()) {
        p.slots.add(slot);
    }
}
\end{lstlisting}

\paragraph{Intermediate Model Validation}
An advantage of the two-stage transformation is that contextual analysis can be specified in an abstract manner -- that is, without having to express the traversal of the AST. This gives clarity and minimises the amount of code required to define syntatic constraints.

\begin{lstlisting}[caption=A constraint (in EVL) to check that all identifiers are unique., label=lst:constraint, language=EVL]
context ClassObject {
    constraint IdentifiersMustBeUnique {
        guard: self.id.isDefined()
        check: ClassObject.all
                   .select(c|c.id = self.id).size() = 1;
        message: `Duplicate identifier: ' + self.id
    }
}
\end{lstlisting}

Epsilon HUTN uses EVL \cite{kolovos08evl} to specify validation, resulting in highly expressive syntactic constraints. An EVL constraint comprises a guard, the logic that specifies the constraint, and a message to be displayed if the constraint is not met. For example, Listing \ref{lst:constraint} specifies the constraint that every HUTN class object has a unique identifier.

In addition to the syntactic constraints defined in the OMG HUTN specification, the EVL constraints for checking conformance (Section~\ref{sec:mmi_syntax}) are also executed on the model at this stage.

\paragraph{Intermediate Model to Target Model}
When the intermediate model conforms to the target metamodel, the intermediate model can be transformed to an instance of the target metamodel. In other words, the model can be represented in a metamodel-specific manner and, for example, saved to disk using XMI. In generating the target model from the intermediate model (Figure \ref{fig:architecture}), the transformation uses information from the target metamodel, such as the names of classes and features. A typical approach to this category of problem is to use a higher-order transformation (HOT) on the target metamodel to generate the desired transformation \cite{tisi09hot}. Currently, ETL cannot be used to produce a transformation from a transformation and hence Epsilon HUTN uses a slightly different approach: the transformation to the target model is produced by executing a M2T transformation on the target metamodel, using EGL \cite{rose08egl}. EGL is a template-based M2T language; \verb|[% %]| tag pairs are used to denote dynamic sections, which may produce text when executed; any code not enclosed in a \verb|[% %]| tag pair is included verbatim in the generated text.

Listing \ref{lst:generate} shows part of the M2T transformation used by Epsilon HUTN. When executed on the target metamodel, the M2T transformation generates an ETL program (i.e. a M2M transformation). The generated ETL code transforms an intermediate model to a model that conforms to the target metamodel. The loop beginning on line 1 iterates over each meta-class in the target metamodel, producing a M2M transformation rule. The generated transformation rule consumes a \texttt{Cl\-a\-ssOb\-je\-ct} in the intermediate model and produces an element of the target model. The guard of the generated transformation rule (line 6) ensures that only \texttt{Cl\-a\-ssOb\-je\-ct} with a type equal to the current meta-class are transformed by the generated rule. To generate the body of the rule, the M2T transformation iterates over each structural feature of the current meta-class, and generates appropriate transformation code for populating the values of each structural feature from the slots on the class object in the intermediate model. The part of the M2T transformation that generates the body of M2M transformation rule is omitted in Listing~\ref{lst:generate} because it contains a large amount of code for interacting with EMF, which is not relevant to this discussion.

\begin{lstlisting}[caption={[Higher-order transformation with EGL]Part of the M2T transformation (in EGL) that takes a target metamodel and generates an intermediate model to target model transformation (in ETL).}, label=lst:generate, language=EGL]
[% for (class in EClass.allInstances()) { %]
rule Object2[%=class.name%]
  transform o : Intermediate!ClassObject
  to t : Model![%=class.name%] {

    guard: o.type = `[%=class.name%]'

    -- body omitted
  }
[% } %]
\end{lstlisting}

To illustrate the way in which Epsilon HUTN generates a target model from an intermediate model, the M2T transformation in Listing~\ref{lst:generate} is applied to the the families metamodel in Figure~\ref{fig:original_families_metamodel_repeated}. The M2T transformation generates the two M2M transformation rules in Listing~\ref{lst:hutn_generated_transformation}. The rules produce instances of \texttt{Fa\-mi\-ly} and \texttt{Pe\-rs\-on} from instances of \texttt{Cl\-a\-ssOb\-je\-ct} in the intermediate model. The body of each rule copies the values from the slots of the \texttt{Cl\-as\-sOb\-je\-ct} to the \texttt{Fa\-mi\-ly} or \texttt{Pe\-rs\-on} in the target model. Lines 7-9, for example, copy the value of the name \texttt{Sl\-ot} (if one is specified) to the target \texttt{Fa\-mi\-ly}.

\begin{lstlisting}[caption=The M2M transformation generated for the Families metamodel, label=lst:hutn_generated_transformation, language=ETL]
rule Object2Family
  transform o : Intermediate!ClassObject
  to t : Model!Family {

    guard: o.type = `Family'

    if (o.hasSlot(`name')) {
			t.name := o.findSlot(`name').values.first;
		}
		
		if (o.hasSlot(`address')) {
			for (value in o.findSlot(`address').values) {
				t.address.add(value);
			}
		}
		
		-- remainder of body omitted
  }

rule Object2Person
  transform o : Intermediate!ClassObject
  to t : Model!Person {

    guard: o.type = `Person'

    if (o.hasSlot(`name')) {
			t.name := o.findSlot(`name').values.first;
		}
		
		-- remainder of body omitted
  }
\end{lstlisting}

Currently, Epsilon HUTN can be used only to generate EMF models. Support for other modelling languages would require different transformations between intermediate and target model. In other words, for each target modelling language, a new EGL template would be required. The transformation from AST to intermediate model is independent of the target modelling language and would not need to change.


\subsection{Migration with Epsilon HUTN}
\label{subsec:migration_with_hutn}
Used in combination with the metamodel-independent syntax presented in Section~\ref{sec:mmi_syntax}, Epsilon HUTN facilitates user-driven co-evolution. To this end, Epsilon HUTN provides development tools (menu items and user-interface wizards) to realise the workflow shown in Figure~\ref{fig:hutn_process_implementation}, which provides an alternative to the user-driven co-evolution workflow observed in Section~\ref{subsec:user-driven_co-evolution}. (The workflow in Figure~\ref{fig:hutn_process_implementation} assumes a graphical model editor, such as those generated by GMF, but any editor built atop EMF will exhibit the same behaviour). First, the user attempts to load a model in the graphical editor. If the model is non-conformant and cannot be loaded, the user clicks the ``Generate HUTN'' menu item, and the model is loaded with the metamodel-independent syntax and then a HUTN representation of the model is generated by Epsilon HUTN. The generated HUTN is presented in an editor that automatically reports conformance problems using the metamodel-independent syntax. The user edits the HUTN to reconcile conformance problems, and the conformance report is automatically updated as the user edits the model. When the conformance problems are fixed, XMI for the conformant model is automatically generated by Epsilon HUTN, and migration is complete. The model can then be loaded in the graphical editor.

\begin{figure}[htbp]
	\centering
	\includegraphics*[viewport=80 290 760 550,height=4.75cm]{6.Evaluation/images/user_driven/hutn_process.pdf}
	\caption{User-driven co-evolution with dedicated structures}
	\label{fig:hutn_process_implementation}
\end{figure}

To demonstrate the way in which HUTN can be used to perform migration, the exemplar XMI shown in Listing~\ref{lst:xmi} is represented using HUTN in Listing~\ref{lst:non-conformant_hutn}. Recall that the XMI describes three Persons, Franz, Julie and Hermann. Julie and Hermann are the mother and father of Franz.

\begin{lstlisting}[caption=HUTN for people with mothers and fathers., label=lst:non-conformant_hutn, language=HutnFamilies]
Persons "kafkas" {
    Person "Franz"   { name: "Franz"   }
    Person "Julie"   { name: "Julie"   }
    Person "Hermann" { name: "Hermann" }
    
    Person "Franz" mother Person "Julie";
    Person "Franz" father Person "Hermann";
}
\end{lstlisting}

Note that, by using a configuration file to specify that a Person's name is taken from its identifier, the body of the Person objects could be omitted.

If the Persons metamodel now evolves such that mother and father are merged to form a parents reference, Epsilon HUTN reports conformance problems on the HUTN document, as illustrated by the screenshot in Figure~\ref{fig:hutn_conformance_reporting}.

\begin{figure}[htbp]
  \begin{center}
    \leavevmode
    \includegraphics[scale=0.44]{5.Implementation/hutn_conformance_reporting.png}
  \end{center}
  \caption{Conformance problem reporting in Epsilon HUTN.}
  \label{fig:hutn_conformance_reporting}
\end{figure}

Resolving the conformance problems requires the user to change the feature named in the infix associations from mother (father) to parents. The Epsilon HUTN development tools provide content assistance, which might be useful in this situation. Listing~\ref{lst:conformant_hutn} shows a HUTN document that conforms to the metamodel defining parents rather than mother and father.

\begin{lstlisting}[caption=HUTN for people with parents., label=lst:conformant_hutn, language=HutnFamilies]
Persons "kafkas" {
    Person "Franz"   { name: "Franz"   }
    Person "Julie"   { name: "Julie"   }
    Person "Hermann" { name: "Hermann" }
    
    Person "Franz" parents Person "Julie";
    Person "Franz" parents Person "Hermann";
}
\end{lstlisting}


\subsection{Limitations}
During the development of Epsilon HUTN, two primary limitations of the notation became apparent. The first relates to the nature of a metamodel-independent syntax, and the latter to the suitability of HUTN for performing user-driven co-evolution.

\subsubsection{Generic vs specific concrete syntax}
Notwithstanding the power of genericity, there are situations where a metamodel-specific concrete syntax is preferable. An example of where HUTN is unhelpful arose when developing a metamodel for the recording of failure behaviour of components in complex systems, based on the work of \cite{wallace05modular}.

Failure behaviours comprise a number of expressions that specify how each component reacts to system faults, and there is an established concrete syntax for expressing failure behaviours. The failure syntax allows various shortcuts, such as the use of underscore to denote a wildcard. For example, the syntax for a possible failure behaviour of a component that receives input from two other components (on the left-hand side of the expression), and produces output for a single component is denoted:

\begin{eqnarray}\label{failure}
(\{\_\}, \{\_\}) \rightarrow (\{late\})
\end{eqnarray}

The above expression is written using a domain-specific syntax. In HUTN, the specification of these behaviours is less concise. For example, Listing \ref{lst:fptc-hutn} gives the HUTN syntax for failure behaviour (\ref{failure}), above.

\begin{lstlisting}[caption=Failure behaviour specified in HUTN., label=lst:fptc-hutn, language=FPTC]
Behaviour {
    lhs: Tuple {
        contents: IdentifierSet { contents: Wildcard {} },
                     IdentifierSet { contents: Wildcard {} }
    }

    rhs: Tuple {
        contents: IdentifierSet { contents: Fault "late" {} }
    }
}
\end{lstlisting}

The domain-specific syntax exploits two characteristics of failure expressions to achieve a compact notation. Firstly, structural domain concepts are mapped to symbols: tuples to parentheses and identifier sets to braces. Secondly, little syntactic sugar is needed for many domain concepts, as they define only one feature: a fault is referred to only by its name, the contents of identifier sets and tuples are separated using only commas.

In general, HUTN is less concise than a domain-specific syntax for metamodels containing a large number of classes with few attributes, and in cases where most attributes are used to define structural relationships among concepts. However, a domain-specific syntax is specified using metamodel concepts and hence can be affected by metamodel evolution.

\subsubsection{Optimised XMI omits type information}
When HUTN is used for user-driven co-evolution, non-conformant models (represented in XMI) are transformed to HUTN source. The default EMF serialisation mechanism is configure to reduce the size of models on disk and consequently the XMI produced by EMF omits type information when it may be inferred from a metamodel. This is problematic for managing co-evolution because, when a metamodel evolves, type information might be erased. The implementation of Epsilon HUTN has been extended to account for optimised XMI, and reports type inference errors along with conformance problems. 

\subsection{Summary}
In this section, HUTN was introduced and its syntax described. An implementation of HUTN for EMF, built atop Epsilon, was discussed. Integration of HUTN for the metamodel-independent syntax discussed in Section~\ref{sec:mmi_syntax} facilitates user-driven co-evolution with a textual modelling notation other than XMI, as demonstrated by the example above. The remainder of this chapter focuses on developer-driven co-evolution, in which model migration strategies are executable.