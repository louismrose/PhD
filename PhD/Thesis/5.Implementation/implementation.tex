%!TEX root = /Users/louis/Documents/PhD/Deliverables/Thesis/thesis.tex

\chapter{Implementation}
\label{Implementation}
Section~\ref{sec:requirements_identification} presented requirements for structures and processes for identifying and managing co-evolution. This chapter describes the way in which the requirements have been approached. Several related structures were implemented, using domain-specific languages, metamodelling and model management operations. Figure~\ref{fig:implementation_overview} summarises the contents of the chapter. To facilitate the management of non-conformant models with existing modelling frameworks, a metamodel-independent syntax was devised and implemented (Section~\ref{sec:mmi_syntax}). To address some of the challenges faced in user-driven co-evolution, an OMG specification for a textual modelling notation was implemented (Section~\ref{sec:notation}). To determine their merits and drawbacks, existing languages for specifying model migration were identified, analysed and compared (Section~\ref{sec:analyis_of_languages_used_for_migration}). Finally, a new model transformation language -- tailored for model migration and centred around a novel approach to relating source and target model elements -- was designed and implemented (Section~\ref{sec:flock}). 

\begin{figure}[htbp]
  \begin{center}
    \leavevmode
    \includegraphics[width=12cm]{5.Implementation/overview.pdf}
  \end{center}
  \caption{Implementation chapter overview.}
  \label{fig:implementation_overview}
\end{figure}


%!TEX root = /Users/louis/Documents/PhD/Deliverables/Thesis/thesis.tex

\section{Metamodel-Independent Syntax}
\label{sec:mmi_syntax}
Section~\ref{subsec:modelling_framework_characteristics} discussed the way in which modelling frameworks implicitly enforce conformance, and hence prevent the loading of non-conformant models. Additionally, modelling frameworks provide little support for checking the conformance of a model with other versions of a metamodel, which is potentially useful during metamodel installation. In Section~\ref{sec:requirements_identification}, these concerns lead to the identification of the following requirement: \emph{This thesis must investigate the extension of existing modelling frameworks to support the loading of non-conformant models and conformance checking of models against other metamodels.}

This section describes the way in which existing modelling frameworks load and store models using a metamodel-specific syntax, proposes an alternative syntax and demonstrates how this facilitates automatic consistency checking. The work presented in this section has been published in \cite{rose09enhanced}.

% TODO - intro example
% TODO - present MM independent syntax. Make clear that it is an ABSTRACT syntax.

\subsection{Example} % TODO - need a better name
The example of metamodel evolution in Figure~\ref{fig:x} is used throughout this section. In Figure~\ref{fig:xa}, \texttt{na\-tu\-r\-alCh\-il\-dr\-en} and \texttt{ad\-op\-t\-edCh\-il\-dr\-en} are modelled as separate features, and, in Figure~\ref{fig:xb}, they are modelled as a single feature, \texttt{ch\-il\-dr\-en}.



The model in Figure~\ref{fig:z}, which conforms to Figure~\ref{fig:x}, is used in the remainder of this section. After the metamodel evolves to Figure~\ref{fig:xb}, the model in Figure~\ref{fig:z} no longer conforms to its metamodel. The sequel describes why loading a non-conformant model fails.

% Use EMF's tree editor to display fig:z


\subsection{Binding to a specific metamodel}
\label{subsec:binding_specific}
When represented with XMI, a model refers to its metamodel's types by name, because models and metamodels are kept separate (Section~\ref{subsec:modelling_framework_characteristics}). At the start of an XMI document, one or more metamodels are imported using XML namespaces. In the body of an XMI document, the namespaces are used to refer to metamodel types. For example, Listing~\ref{lst:xmi} is the XMI representation of the model show in Figure~\ref{fig:z}. Line 2 imports metamodels under the \texttt{xmi} and \texttt{fa\-mi\-li\-es} namespaces. In the rest of Listing~\ref{lst:xmi}, the \texttt{xmi} namespace can be used to reference types provided by the modelling framework, and the \texttt{fa\-mi\-li\-es} namespace to reference types provided by the metamodel in Figure~\ref{fig:x}. Line 2, for example, uses the \texttt{fa\-mi\-li\-es} namespace to reference the \texttt{Fa\-mi\-ly} type from the metamodel in Figure~\ref{fig:x}. 

\begin{lstlisting}[caption=Family model in XMI, label=lst:xmi, language=XML]
<?xml version="1.0" encoding="ASCII"?>
<families:Family xmi:version="2.0" xmlns:xmi="http://www.omg.org/XMI" xmlns:families="http://www.cs.york.ac.uk/families" xmi:id="_kE2LkAagEeC-FIOYrvUj0A" name="Smiths">
  <naturalChildren xmi:id="_q8RWYAagEeC-FIOYrvUj0A" name="Paul"/>
  <adoptedChildren xmi:id="_nj6TcAagEeC-FIOYrvUj0A" name="John"/>
</families:Family>
\end{lstlisting}

Loading a model involves constructing objects in the underlying programming language. When a model is loaded, modelling frameworks bind the model to its metamodel using the underlying programming language (Section~\ref{subsec:modelling_framework_characteristics}). The metamodel defines the way in which model elements will be bound, and binding is often strongly-typed: each metamodel type is mapped to a corresponding type in the underlying programming language. Binding a model element involves instantiating, in the underlying programming language, the metamodel type, and populating the attributes of the instantiated object with values that correspond to those specified in the model. Figure~\ref{fig:a} shows the (Java) objects that would be created by EMF when loading the XMI in Listing~\ref{lst:xmi}.

Binding fails for non-conformant models. For example, attempting to bind the model shown in Figure~\ref{fig:y} to the metamodel shown in Figure~\ref{fig:xb} fails because the \texttt{Fa\-mi\-ly} class no longer defines features called  \texttt{na\-tu\-r\-alCh\-il\-dr\-en} and \texttt{ad\-op\-t\-edCh\-il\-dr\-en}. Change to a metamodel that affect the conformance of models prevent models non-conformant from being loaded and, hence, used with model editors and model management operations, such as model transformations.



\subsection{Potential solutions for loading non-conformant models}
A metamodel-specific binding requires that a model conforms to its metamodel and, therefore, metamodel evolution can cause metamodel-specific binding to fail. To address this, two potential approaches to binding (and hence loading) non-conformant models have been considered and are now discussed. The benefits and drawbacks of each approach have been compared, which resulted in the selection of the second approach, binding to a generic metamodel.

\subsubsection{Preserving metamodel history}
Presently, modelling frameworks store only the latest version of a metamodel, and hence binding fails for models that conform to a previous version of the metamodel. Storing the entire history of every metamodel would avoid any conformance problems that might be introduced by metamodel evolution. Binding would involve locating the correct version of the metamodel from the history, and proceed in the same manner as a metamodel-specific binding.  

\subsubsection{Binding to a generic metamodel}
Conformance indicates whether a metamodel and model are consistent in terms of their definition and use of types. A metamodel-specific binding fails for a non-conformant model because the types defined in the metamodel cannot be used to construct a representation of the model in the underlying programming language. An alternative, therefore, is to bind models to a metamodel-independent representation in the underlying programming language. Binding is then not dependent on the types defined in metamodels, and will succeed for non-conformant models.
   
\subsubsection{Benefits and drawbacks of the potential solutions}
History
+ Load any version of a model with model editors
+ If mod man operation history is stored too, can use models with mod man operations too

+/- Size of metamodel will be larger. Might need to investigate an efficient storage representation, patches for example.

- Relies on MM dev.
- There's no standard way of storing metamodel histories.


MMI
+ Can load any model, even if we don't have the metamodel / know what it looks like.
+ Doesn't rely on MM user.

- MM is not represented in domain-specific concepts, and hence cannot be used with editors, transformations, etc.


\subsection{Proposed solution: binding to a generic metamodel}
\label{subsec:binding}
For situations when a model does not conform to its metamodel, this thesis proposes an alternative to the binding mechanism described in Section~\ref{subsec:binding_specific}, which binds a model to a \emph{generic} metamodel. A generic metamodel reflects the characteristics of the metamodelling language and consequently every model conforms to the generic metamodel. Figure~\ref{fig:slot_model} shows a minimal generic metamodel for MOF, which is based on the MOF metamodel \cite{mof}. Model elements are bound to \texttt{Object}, data values to \texttt{Slot}.

\begin{figure}[htbp]
  \centering
  \includegraphics[width=3.3in]{5.Implementation/slot_model.pdf}
  \caption[A generic metamodel for MOF]{A generic metamodel for MOF, based on \cite{mof} and taken from \cite{rose09enhanced}.}
  \label{fig:slot_model}
\end{figure}

Using the metamodel in Figure~\ref{fig:slot_model} in conjunction with MOF, conformance constraints can be expressed, as shown below. A minimal subset of MOF is shown in Figure~\ref{fig:minimal_mof}.

\begin{figure}[htbp]
  \centering
  \includegraphics[width=3.3in]{5.Implementation/mof.pdf}
  \caption[Minimal MOF metamodel]{Minimal MOF metamodel, based on \cite{mof} and taken from \cite{rose09enhanced}.}
  \label{fig:minimal_mof}
\end{figure}

The following constraints between metamodels (e.g. instances of MOF, Figure~\ref{fig:minimal_mof}) and models represented with a generic metamodel (e.g. instances of Figure~\ref{fig:slot_model}) can be used to express conformance:

\begin{enumerate}
	\item Each object's type must be the name of some non-abstract metamodel class.
	\item Each object must specify a slot for each mandatory feature of its type.
	\item Each slot's feature must be the name of a metamodel feature. That metamodel feature must belong to the slot's owning object's type.
	\item Each slot must be multiplicity-compatible with its feature. More specifically, each slot must contain at least as many values as its feature's lower bound, and at most as many values as its feature's upper bound.
  \item Each slot must be type-compatible with its feature.
\end{enumerate}

% TODO illustrate for person (think I do this in the next section, maybe move to here)

The way in which type-compatibility is checked depends on the way in which the modelling framework is implemented. In EMF, for example, model values conform either to types defined in a metamodel, or to types defined in the underlying programming language, Java. EMF provides programmatic access to a metamodel's type system, which can be used to implement type-compatibility checks.

Conformance constraints vary over modelling languages. For example, Ecore, the modelling language of EMF, is similar to but not the same as MOF. Metamodel features defined in Ecore can be marked as transient (not stored to disk) or unchangeable (read-only). Consequently in EMF, conformance constraints are required to restrict the feature value of slots to only non-transient, changeable features.


\subsection{Example}
\label{subsec:mmi_syntax_example}
By binding a model to the generic metamodel presented in Figure~\ref{fig:slot_model} rather than to the underlying programming languages types defined in its metamodel, conformance can be checked using the above constraints. Binding the XMI in Listing~\ref{lst:xmi} to the generic metamodel shown in Figure~\ref{fig:slot_model} produces three instances of \texttt{Ob\-je\-ct}, illustrated as a UML object diagram in Figure~\ref{fig:generic_binding}. For clarity, instances of \texttt{Ob\-je\-ct} are shaded, and instances of \texttt{Sl\-ot} are unshaded. The first \texttt{Ob\-je\-ct} represents the \texttt{Fa\-mi\-ly} model element and has three slots. Two of the slots are used to reference the \texttt{Pe\-rs\-on} model elements via the \texttt{na\-tu\-r\-alCh\-il\-dr\-en} and \texttt{ad\-op\-t\-edCh\-il\-dr\-en} references. When binding to a generic metamodel, models are represented in terms of a set of types that is metamodel-independent (\texttt{Mo\-d\-el}, \texttt{Ob\-je\-ct} and \texttt{Sl\-ot}), and hence binding succeeds for conformant and non-conformant models.

\begin{figure}[htbp]
  \centering
  \includegraphics[width=4in]{5.Implementation/GenericBinding.pdf}
  \caption{Exemplar instantiation of generic metamodel.}
  \label{fig:generic_binding}
\end{figure}

After binding to the generic metamodel, the conformance of a model can be checked against any specific metamodel. To illustrate the value of the generic metamodel, consider again the metamodel evolution in Figure~\ref{fig:x} and the model in Figure~\ref{fig:y}. Conformance checking for the model element representing the \texttt{Fa\-mi\-ly} would now fail because it defines slots for features named  \texttt{na\-tu\-r\-alCh\-il\-dr\-en} and \texttt{ad\-op\-t\-edCh\-il\-dr\-en}, which are no longer defined for the metamodel class \texttt{Fa\-mi\-ly}. Specifically, the model element representing the \texttt{Fa\-mi\-ly} does not satisfy conformance constraint 4 (Section~\ref{subsec:binding}), which states: \emph{each slot's feature must be the name of a metamodel feature. That metamodel feature must belong to the slot's owning object's type}. 

\subsection{Structures built atop the metamodel-independent syntax}
There are many potential uses for the metamodel-independent syntax described in this section. Section~\ref{sec:notation} describes a textual modelling notation integrated with the metamodel-independent syntax to achieve live conformance checking. The migration language presented in Section~\ref{sec:flock} can be used with the metamodel independent syntax to perform partial migration (i.e. to produce models that conform to a generic metamodel rather than their evolved metamodel), but implementation of this work is not complete\footnote{The current version does not support metamodels that contain non-containment references and enumeration types.}.

One of the model migration tools discussed in Section~\ref{sec:analyis_of_languages_used_for_migration}, COPE \cite{herrmannsdoerfer09cope}, uses a metamodel-independent syntax similar to the one presented in this section. The metamodel-independent syntax presented here was developed independently of the metamodel-independent syntax developed for COPE. The two syntaxes were first published in 2008 (\cite{rose08hutn,herrmannsdoerfer08cope}), and are both conceptually similar to the metamodel for UML Object diagrams \cite{uml14}.

In addition to these uses, the metamodel-independent syntax is potentially useful during metamodel installation. As discussed in Section~\ref{subsec:modelling_framework_characteristics}, metamodel developers do not have access to downstream models, and conformance is implicitly enforced by modelling frameworks. Consequently, the conformance of models may be affected by the installation of a new version of a metamodel, and the conformance of models cannot be checked during installation. Typically, installing a new version of a metamodel can result in models that no longer conform to their metamodel and cannot be used with the modelling framework. Moreover, a user discovers conformance problems only when attempting to use a model after installation has completed, and not as part of the installation process.

% TODO - Consider discussing the number of revisions of say, UML (and other metamodels), in Chapter 2. This will allow the reader to get a sense of the scale of the problem.

To enable conformance checking as part of metamodel installation in EMF, the metamodel-independent syntax has been integrated with Concordance in \cite{rose10concordance}. The work was conducted outside of the scope of the thesis, and is now summarised to indicate the usefulness of the metamodel-independent syntax for supporting the automation of co-evolution activities. Concordance provides a mechanism for resolving inter-model references (such as those between models and their metamodels). Without Concordance, determining the the instances of a metamodel is possible only by checking every model in the workspace. Integrating Concordance and the metamodel-independent syntax resulted in a service, which Epsilon (Section~\ref{subsec:epsilon}) executes after the installation of a metamodel to identify the models that are affected by the metamodel changes. All models that conform to the old version of the metamodel are checked for conformance with the new metamodel. As such, conformance checking occurs automatically and immediately after metamodel installation. Conformance problems are detected and reported immediately, rather than when an affected model is next used.

\subsubsection{Summary}
Modelling frameworks implicitly enforce conformance, which presents challenges for managing co-evolution. In particular, detecting and reconciling conformance problems involves managing non-conformant models, which cannot be loaded by modelling frameworks and hence cannot be used with model editors or model management operations. The metamodel-independent syntax proposed in this section enables modelling frameworks to load non-conformant models by binding models to a generic metamodel. The metamodel-independent syntax has been integrated with Concordance \cite{rose10concordance} to facilitate the reporting of conformance problems during metamodel installation, and underpins the implementation of the textual modelling notation presented in Sections~\ref{sec:notation}. The benefits and drawbacks of the metamodel-independent syntax in the context of user-driven co-evolution are explored in Chapter~\ref{Evaluation}. 

%!TEX root = /Users/louis/Documents/PhD/Deliverables/Thesis/thesis.tex

\section{Textual Modelling Notation}
\label{sec:notation}
% The Human-Usable Textual Notation is an OMG standard textual concrete syntax for the MOF metamodelling architecture. The notation is metamodel-independent -- it can be used with any model that conforms to any MOF-based metamodel. HUTN provides a human-usable means for visualising and specifying models, even when those models are inconsistent with their metamodel.
The analysis of co-evolution examples in Chapter~\ref{Analysis} highlighted two categories of process for managing co-evolution, developer-driven and user-driven. In the former, migration strategies are executable, while in the latter they are not. Performing user-driven co-evolution with modelling frameworks presents two key challenges that have not been explored by existing research. Firstly, user-driven co-evolution frequently involves editing the storage representation of the model, such as XMI. Model storage representations are typically not optimised for human use and hence user-driven co-evolution can be error-prone. Secondly, non-conformant model elements must be identified during user-driven co-evolution. When a multi-pass parser is used to load models, as is the case with EMF, not all conformance problems are reported at once, and user-driven co-evolution is an iterative process. In Section~\ref{sec:requirements_identification}, these challenges lead to the identification of the following requirement: \emph{This thesis must demonstrate a user-driven co-evolution process that enables the editing of non-conformant models without directly manipulating the underlying storage representation and provides a sound and complete conformance report for the original model and evolved metamodel.}

The remainder of this section describes a textual notation for models, which has been implemented for EMF, and discusses the way in which the notation has been integrated with the metamodel independent syntax described in Section~\ref{sec:mmi_syntax} to produce conformance reports. 


\subsection{Human-Usable Textual Notation}
\label{subsec:hutn}
The OMG's Human-Usable Textual Notation (HUTN) \cite{hutn} defines a textual modelling notation, which aims to conform to human-usability criteria \cite{hutn}. There is no current reference implementation of HUTN: the Distributed Systems Technology Centre's TokTok project (an implementation of the HUTN specification) is inactive (and the source code can no longer be found), whilst work on implementing the HUTN specification by Muller and Hassenforder \cite{muller05hutn} has been abandoned in favour of Sintaks \cite{sintaks}, which operates on domain-specific concrete syntax.

Model storage representations are often optimised to reduce storage space or to increase the speed of random access, rather than for human usability. By contrast, the HUTN specification states its primary design goal as human-usability and ``this is achieved through consideration of the successes and failures of common programming languages'' \cite[Section 2.2]{hutn}. The HUTN specification refers to two studies of programming language usability to justify design decisions. Because no reference implementation exists, the specification does not evaluate the human-usability of the notation. This thesis proposes that HUTN be used instead of XMI for user-driven co-evolution. Further discussion of the human-usability of HUTN is deferred to Chapter~\ref{Evaluation}.

Like the generic metamodel presented in Section~\ref{sec:mmi_syntax}, HUTN is a metamodel-independent syntax for 
MOF. In this section, the core syntax and key features of HUTN are introduced. The complete definition is available
in \cite{hutn}. To illustrate usage of the notation, the MOF-based metamodel of families in Figure \ref{fig:example-mm} is used. (A nuclear family ``consists only of a father, a mother, and children.'' \cite{nucleardef}).

\begin{figure}[htbp]
  \begin{center}
    \leavevmode
    \includegraphics[scale=0.85]{5.Implementation/families.pdf}
  \end{center}
  \caption{Exemplar families metamodel. (Shading is irrelevant).}
  \label{fig:example-mm}
\end{figure}


\subsubsection{Basic Notation}
Listing \ref{lst:attributes} shows the construction of an \emph{object} in HUTN, here an instance of the Family class from Figure \ref{fig:example-mm}. Line 1 specifies the package containing the classes to be constructed (\texttt{FamilyPackage}) and a corresponding identifier (\texttt{families}), used for fully-qualifying references to objects (Section \ref{subsubsec:inter-package_references}). Line 2 names the class (\texttt{Family}) and gives an identifier for the object (\texttt{The Smiths}). Lines 3 to 7 define \emph{attribute values}; in each case, the data value is assigned to the attribute with the specified name. The encoding of the value depends on its type: strings are delimited by any form of quotation mark; multi-valued attributes use comma separators, etc.

The metamodel in Figure \ref{fig:example-mm} defines a \emph{simple reference} (familyFriends) and two \emph{containment references} (adoptedChildren; naturalChildren). The HUTN representation embeds a contained object directly in the parent object, as shown in Listing \ref{lst:containment}. A simple reference can be specified using the type and identifier of the referred object, as shown in Listing \ref{lst:non-contained}. Like attribute values, both styles of reference are preceded by the name of the meta-feature.

\begin{lstlisting}[caption=Specifying attributes with HUTN., label=lst:attributes, language=HutnFamilies]
FamilyPackage "families" {
    Family "The Smiths" {
        nuclear: true
        name: "The Smiths"
        averageAge: 25.7
        numberOfPets: 2
        address: "120 Main Street", "37 University Road"
    }
}
\end{lstlisting}

\begin{lstlisting}[caption=Instantiation of naturalChildren -- a HUTN containment reference., label=lst:containment, language=HutnFamilies]
FamilyPackage "families" {
    Family "The Smiths" {
        naturalChildren: Person "John" { name: "John" },
                                Person "Jo" { gender: female }
    }
}
\end{lstlisting}


\begin{lstlisting}[caption=Specifying a simple reference with HUTN., label=lst:non-contained, language=HutnFamilies]
FamilyPackage "families" {
    Family "The Smiths" {
        familyFriends: Family "The Does"
    }
    Family "The Does" {}
}
\end{lstlisting}


\subsubsection{Keywords and Adjectives}
While HUTN is unlikely to be as concise as a metamodel-specific concrete syntax, the notation does define syntactic shortcuts to make model specifications more compact. Shortcut use is optional, and the HUTN specification aims to make their syntax intuitive \cite[pg2-4]{hutn}. Two example notational shortcuts are described here, to illustrate some of the ways in which HUTN can be used to construct models in a concise manner.

When specifying a \emph{Boolean-valued attribute}, it is sufficient to simply use the attribute name (value \texttt{true}), or the attribute name prefixed with a tilde (value \texttt{false}). When used in the body of the object, this style of Boolean-valued attribute represents a \emph{keyword}. A keyword used to prefix an object declaration is called an \emph{adjective}. Listing \ref{lst:boolean} shows the use of both an attribute keyword (\texttt{\textasciitilde nuclear} on line 6) and adjective (\texttt{\textasciitilde migrant} on line 2).

\begin{lstlisting}[caption=Using keywords and adjectives in HUTN., label=lst:boolean, language=HutnFamilies]
FamilyPackage "families" {
    ~migrant Family "The Smiths" {}

    Family "The Does" {
        averageAge: 20.1
        ~nuclear
        name: "The Does"
    }
}
\end{lstlisting}


\subsubsection{Inter-Package References}
\label{subsubsec:inter-package_references}
To conclude the summary of the notation, two advanced features defined in the HUTN specification are discussed. The first enables objects to refer to other objects in a different package, while the second provides means for specifying the values of a reference for all objects in a single construct (which can be used, in some cases, to simplify the specification of complicated relationships).

\begin{lstlisting}[caption=Referencing objects in other packages with HUTN., label=lst:fullyqualified, language=HutnFamilies]
FamilyPackage "families" {
    Family "The Smiths" {}
}
VehiclePackage "vehicles" {
    Vehicle "The Smiths' Car" {
        owner: FamilyPackage.Family "families"."The Smiths"
    }
}
\end{lstlisting}

To reference objects between separate package instances in the same document, the package identifier is used to construct a fully-qualified name. Suppose a second package is introduced to the metamodel in Figure \ref{fig:example-mm}. Among other concepts, this package introduces a Vehicle class, which defines an owner reference of type Family. Listing \ref{lst:fullyqualified} illustrates the way in which the owner feature can be populated. Note that the fully-qualified form of the class utilises the names of elements of the metamodel, while the fully-qualified form of the object utilises only HUTN identifiers defined in the current document.

The HUTN specification defines name scope optimisation rules, which allow the definition above to be simplified to: \texttt{owner: Family "The Smiths"}, assuming that the VehiclePackage does not define a Family class, and that the identifier ``The Smiths'' is not used in the VehiclePackage block, or this HUTN document is configured to require unique identifiers over the entire document.


\subsubsection{Alternative Reference Syntax}
In addition to the syntax defined in Listings \ref{lst:containment} and \ref{lst:non-contained}, the value of references may be specified independently of the object definitions. For example, Listing \ref{lst:assocblock} demonstrates this alternate syntax by defining The Does as friends with both The Smiths and The Bloggs.

\begin{lstlisting}[caption=Using a reference block in HUTN., label=lst:assocblock, language=HutnFamilies]
FamilyPackage "families" {
    Family "The Smiths" {}
    Family "The Does" {}
    Family "The Bloggs" {}
    
    familyFriends {
        "The Does" "The Smiths"
        "The Does" "The Bloggs"
    }
}
\end{lstlisting}

Listing \ref{lst:associnfix} illustrates a further alternative syntax for references, which employs an infix notation. 

\begin{lstlisting}[caption=Using an infix reference in HUTN., label=lst:associnfix, language=HutnFamilies]
FamilyPackage "families" {
    Family "The Smiths" {}
    Family "The Does" {}
    Family "The Bloggs" {}
    
    Family "The Smiths" familyFriends Family "The Does";
    Family "The Smiths" familyFriends Family "The Bloggs";
}
\end{lstlisting}

The reference block (Listing~\ref{lst:assocblock}) and infix (Listing~\ref{lst:associnfix}) notations are syntactic variations on -- and have identical semantics to -- the reference notation shown in Listings \ref{lst:containment} and \ref{lst:non-contained}.


\subsubsection{Customisation via Configuration}
Some limited customisation of HUTN for particular metamodels can be achieved using \emph{configuration files}. Customisations permitted include a parametric form of object instantiation (not yet implemented); renaming of metamodel elements; giving default values for attributes; and stating an attribute whose values are used to infer a default identifier.


\subsection{Epsilon HUTN}
\label{subsec:epsilon_hutn}
To investigate the extent to which HUTN can be used during user-driven co-evolution, an implementation, Epsilon HUTN, was constructed. This section describes the way in which Epsilon HUTN was implemented using a combination of model-management operations. From text conforming to the HUTN syntax (described above), Epsilon HUTN produces an equivalent model that can be managed with the Eclipse Modeling Framework \cite{steinberg09emf}. The sequel demonstrates the way in which Epsilon HUTN can be used for user-driven co-evolution.

\subsubsection{Implementation of Epsilon HUTN}
Epsilon HUTN, makes extensive use of the Epsilon model management platform, which was introduced in Section~\ref{subsec:epsilon}. Epsilon provides infrastructure for implementing uniform and interoperable model management languages, for performing tasks such as model merging, model transformation and inter-model consistency checking. Epsilon HUTN is implemented using the model-to-model transformation, model-to-text transformation and model validation languages of Epsilon. Although any languages for model-to-model transformation 
(M2M), model-to-text transformation (M2T) and model validation could have been used, Epsilon's existing domain-specific languages are tightly integrated and inter-operable, making it feasible to chain model management operations together to implement Epsilon HUTN.

\begin{figure}[htbp]
  \begin{center}
    \leavevmode
    \includegraphics[scale=0.44]{5.Implementation/hutn_workflow.png}
  \end{center}
  \caption{The architecture of Epsilon HUTN.}
  \label{fig:architecture}
\end{figure}

Figure \ref{fig:architecture} outlines the workflow through Epsilon HUTN, from HUTN source text to instantiated target model. The HUTN model specification is parsed to an abstract syntax tree using a HUTN parser specified in ANTLR \cite{parr07antlr}. From this, a Java postprocessor is used to construct an instance of a simple AST metamodel (which comprises two meta-classes, Tree and Node). Using ETL, M2M transformations are then applied to produce an instance of the generic metamodel discussed in Section~\ref{sec:mmi_syntax}. Finally, a M2T transformation on the target metamodel, specified in EGL, produces a further M2M transformation, from the generic metamodel to the target model.

The workflow uses an extension of the generic metamodel defined in Section~\ref{sec:mmi_syntax}. Because the HUTN specification allows the use of packages, an extra element, \texttt{PackageObject}, was added to the generic metamodel. A \texttt{PackageObject} has a type, an optional identifier and contains any number of \texttt{Object}s. To avoid confusion with \texttt{PackageObjects}, the \texttt{Object} class in the generic metamodel was renamed to \texttt{ClassObject}.

Using two M2M transformation stages with the (extended) generic metamodel as an intermediary has two advantages. Firstly, the form of the AST metamodel is not suited to a one-step transformation. There is a mismatch between the features of the AST metamodel and the needs of the target model -- for example, between the Node class in the AST metamodel and classes in the target metamodel. If a one-step transformation were used, each transformation rule would need a lengthly guard statement, which is hard to understand and verify. Secondly, Section~\ref{sec:mmi_syntax} discussed a mechanism for binding XMI to the generic metamodel, which can be used in conjunction with the latter half of the Epsilon HUTN workflow (Figure~\ref{fig:architecture}) to generate HUTN from XMI. This process is discussed further in Section~\ref{subsec:migration_with_hutn}.

Throughout the remainder of this section, instances of the generic metamodel producing during the execution of the HUTN workflow are termed an \textit{intermediate model}. The two M2M transformations are now discussed in depth, along with a model validation phase which is performed prior to the second transformation.

\paragraph{AST Model to Intermediate Model}
Epsilon HUTN uses ETL for specifying M2M transformation. One of the transformation rules from Epsilon HUTN is shown in Listing \ref{lst:m2m}. The rule transforms a name node in the AST model (which could represent a package or a class object) to a package object in the intermediate model. The guard (line 5) specifies that a name node will only be transformed to a package object if the node has no parent (i.e. it is a top-level node, and hence a package rather than a class). The body of the rule states that the type, line number and column number of the package are determined from the text, line and column attributes of the node object. On line 11, a containment slot is instantiated to hold the children of this package object. The children of the node object are transformed to the intermediate model (using a built-in method, \verb|equivalent()|), and added to the containment slot.

\begin{lstlisting}[caption=Transformation rule (in ETL) to convert AST nodes to package objects., label=lst:m2m, language=ETL]
rule NameNode2PackageObject
    transform n : AntlrAst!NameNode
    to p : Intermediate!PackageObject {

    guard : n.parent.isUndefined()

    p.type := n.text;
    p.line := n.line;
    p.col  := n.column;

    var slot := new Intermediate!ContainmentSlot;
    for (child in n.children) {
        slot.objects.add(child.equivalent());
    }
    if (slot.objects.notEmpty()) {
        p.slots.add(slot);
    }
}
\end{lstlisting}

\paragraph{Intermediate Model Validation}
An advantage of the two-stage transformation is that contextual analysis can be specified in an abstract manner -- that is, without having to express the traversal of the AST. This gives clarity and minimises the amount of code required to define syntatic constraints.

\begin{lstlisting}[caption=A constraint (in EVL) to check that all identifiers are unique., label=lst:constraint, language=EVL]
context ClassObject {
    constraint IdentifiersMustBeUnique {
        guard: self.id.isDefined()
        check: ClassObject.allInstances()
                   .select(c|c.id = self.id).size() = 1;
        message: `Duplicate identifier: ' + self.id
    }
}
\end{lstlisting}

Epsilon HUTN uses EVL \cite{kolovos08evl} to specify verification, resulting in highly expressive syntactic constraints. An EVL constraint comprises a guard, the logic that specifies the constraint, and a message to be displayed if the constraint is not met. For example, Listing \ref{lst:constraint} specifies the constraint that every HUTN class object has a unique identifier.

In addition to the syntactic constraints defined in the HUTN specification, the conformance constraints described in Section~\ref{sec:mmi_syntax} are executed on the model at this stage. For this purpose, the conformance constraints are specified in EVL.

\paragraph{Intermediate Model to Target Model}
Because the contextual analysis is performed on the intermediate model, models conform to the target metamodel. In generating the target model from the intermediate model (Figure \ref{fig:architecture}), the transformation uses information from the target metamodel, such as the names of classes and features. A typical approach to this category of problem is to use a higher-order transformation on the target metamodel to generate the desired transformation. Epsilon HUTN uses a different approach: the transformation to the target model is produced by executing an EGL template on the target metamodel. EGL is a template-based text generation language. \verb|[% %]| tag pairs are used to denote dynamic sections, which may produce text when executed. Any code not enclosed in a \verb|[% %]| tag pair is included verbatim in the generated text.

Listing \ref{lst:generate} is the EGL template for a M2T transformation on the target metamodel; it generates the M2M transformation used for generating the target model. The loop beginning on line 1 iterates over each meta-class in the metamodel, producing a transformation rule to generate target model instances of that meta-class from class objects in the intermediate model. The template guard (line 6) specifies that only class objects of the same type as the meta-class be transformed by the current rule. For the body of the rule the template iterates over each structural feature of the current meta-class, and generates appropriate transformation code for populating the values of each structural feature from the slots on the class object in the intermediate model. The template body is omitted in Listing~\ref{lst:generate} because it contains a large amount of code for interacting with EMF, which is not relevant to this discussion.

\begin{lstlisting}[caption= Initial sections of the template (in EGL) for generating  rules (in ETL) to instantiate classes of the target metamodel., label=lst:generate, language=EOL]
[% for (class in EClass.allInstances()) { %]
rule Object2[%=class.name%]
  transform o : Intermediate!ClassObject
  to t : Model![%=class.name%] {

    guard: o.type = `[%=class.name%]'

    -- body omitted
  }
[% } %]
\end{lstlisting}

Presently, Epsilon HUTN can be used only to generate EMF models. Support for other modelling languages, such as MDR, would require different transformations between intermediate and target model. In other words, for each target modelling language, a new EGL template would be required. The transformation from AST to intermediate model is independent of the target modelling language and would not need to change.


\subsection{Migration with HUTN}
\label{subsec:migration_with_hutn}
Epsilon HUTN uses the generic metamodel (from Section~\ref{sec:mmi_syntax}) as an intermediary, facilitating transformation from XMI to HUTN (i.e. the inverse of the transformation discussed above): XMI is parsed to produce an instance of the generic metamodel, and an unparser (implemented using the visitor design pattern \cite{gamma95patterns}) generates HUTN source. In this manner, HUTN can be generated for any XMI document, regardless of whether the model described by the XMI conforms to its metamodel.\footnote{TODO: Somewhere, I need to discuss loss of information. (e.g. model element type information when a metaclass is removed)}

To demonstrate the way in which HUTN can be used to perform migration, the exemplar XMI shown in Listing~\ref{lst:xmi} is represented using HUTN in Listing~\ref{lst:non-conformant_hutn}. Recall that the XMI describes three Persons, Franz, Julie and Hermann. Julie and Hermann are the mother and father of Franz.

\begin{lstlisting}[caption=HUTN for people with mothers and fathers., label=lst:non-conformant_hutn, language=HutnFamilies]
Persons "kafkas" {
    Person "Franz"   { name: "Franz"   }
    Person "Julie"   { name: "Julie"   }
    Person "Hermann" { name: "Hermann" }
    
    Person "Franz" mother Person "Julie";
    Person "Franz" father Person "Hermann";
}
\end{lstlisting}

Note that, by using a configuration file to specify that a Person's name is taken from its identifier, the body of the Person objects could be omitted.

If the Persons metamodel now evolves such that mother and father are merged to form a parents reference, Epsilon HUTN reports conformance problems on the HUTN document, as illustrated by the screenshot in Figure~\ref{fig:hutn_conformance_reporting}.

\begin{figure}[htbp]
  \begin{center}
    \leavevmode
    \includegraphics[scale=0.44]{5.Implementation/hutn_conformance_reporting.png}
  \end{center}
  \caption{Conformance problem reporting in Epsilon HUTN.}
  \label{fig:hutn_conformance_reporting}
\end{figure}

Resolving the conformance problems requires the user to change the feature named in the infix associations from mother (father) to parents. The Epsilon HUTN development tools provide content assistance, which might be useful in this situation. Listing~\ref{lst:conformant_hutn} shows a HUTN document that conforms to the metamodel defining parents rather than mother and father.

\begin{lstlisting}[caption=HUTN for people with parents., label=lst:conformant_hutn, language=HutnFamilies]
Persons "kafkas" {
    Person "Franz"   { name: "Franz"   }
    Person "Julie"   { name: "Julie"   }
    Person "Hermann" { name: "Hermann" }
    
    Person "Franz" parents Person "Julie";
    Person "Franz" parents Person "Hermann";
}
\end{lstlisting}


\subsection{Limitations}
Notwithstanding the power of genericity, there are situations where a metamodel-specific concrete syntax is preferable. An example of where HUTN is unhelpful arose when developing a metamodel for the recording of failure behaviour of components in complex systems, based on the work of \cite{wallace05modular}.

Failure behaviours comprise a number of expressions that specify how each component reacts to system faults, and there is an established concrete syntax for expressing failure behaviours. The failure syntax allows various shortcuts, such as the use of underscore to denote a wildcard. For example, the syntax for a possible failure behaviour of a component that receives input from two other components (on the left-hand side of the expression), and produces output for a single component is denoted:

\begin{eqnarray}\label{failure}
(\{\_\}, \{\_\}) \rightarrow (\{late\})
\end{eqnarray}

The above expression is written using a domain-specific syntax. In HUTN, the specification of these behaviours is less concise. For example, Listing \ref{lst:fptc-hutn} gives the HUTN syntax for failure behaviour (\ref{failure}), above.

\begin{lstlisting}[caption=Failure behaviour specified in HUTN., label=lst:fptc-hutn, language=FPTC]
Behaviour {
    lhs: Tuple {
        contents: IdentifierSet { contents: Wildcard {} },
                     IdentifierSet { contents: Wildcard {} }
    }

    rhs: Tuple {
        contents: IdentifierSet { contents: Fault "late" {} }
    }
}
\end{lstlisting}

The domain-specific syntax exploits two characteristics of failure expressions to achieve a compact notation. Firstly, structural domain concepts are mapped to symbols: tuples to parentheses and identifier sets to braces. Secondly, little syntactic sugar is needed for many domain concepts, as they define only one feature: a fault is referred to only by its name, the contents of identifier sets and tuples are separated using only commas.

In general, HUTN is less concise than a domain-specific syntax for metamodels containing a large number of classes with few attributes, and in cases where most attributes are used to define structural relationships among concepts. However, there might still be benefits from using HUTN in such cases, if the metamodel is likely to be modified frequently, of it the model does not yet have a formal metamodel.

\subsection{Summary}
In this section, HUTN was introduced and its syntax described. An implementation of HUTN for EMF, built atop Epsilon, was discussed. Integration of HUTN for the metamodel-independent syntax discussed in Section~\ref{sec:mmi_syntax} facilitates user-driven co-evolution with a textual modelling notation other than XMI, as demonstrated by the example above. The remainder of this chapter focuses on developer-driven co-evolution, in which model migration strategies are executable.
%!TEX root = /Users/louis/Documents/PhD/Deliverables/Thesis/thesis.tex

\section{Analysis of Languages used for Migration}
\label{sec:analyis_of_languages_used_for_migration}
In contrast to the previous two sections, this section focuses not on \emph{user-driven} but rather on \emph{developer-driven} co-evolution, in which migration is specified in a programming language. Section~\ref{subsec:co-evolution_categorisation} discussed existing approaches to model migration, highlighting variation in the languages used for specifying migration strategies. In this section, migration strategy languages are compared, using the example of metamodel evolution given in Section~\ref{subsec:co-evo_example}. From this comparison, requirements for a domain-specific language for specifying and executing model migration strategies are derived (Section~\ref{subsec:analysis}). The sequel describes an implementation of a model model migration language based on the analysis presented here. The work described in this section has been published in \cite{rose10flock}.

\subsection{Co-Evolution Example}
\label{subsec:co-evo_example}
Throughout this section, the following exemplar evolution of a Petri net metamodel is used to compare model migration languages. The same example has been used previously in co-evolution literature \cite{cicchetti08automating,garces09managing,wachsmuth07metamodel}.

\begin{figure}[bp]
	\centering
	\subfigure[Original metamodel.]
	{
	    \label{fig:original_mm}
	    \includegraphics[scale=0.75]{5.Implementation/images/petri_nets_before.pdf}
	}
	\subfigure[Evolved metamodel.]
	{
	    \label{fig:evolved_mm}
	    \includegraphics[scale=0.75]{5.Implementation/images/petri_nets_after.pdf}
	}
	\caption[Exemplar metamodel evolution (Petri nets)]{Exemplar metamodel evolution. Taken from \cite{rose10flock}.}
\label{fig:petri_nets_mms}
\end{figure}

In Figure~\ref{fig:original_mm}, a Petri \texttt{Net} comprises \texttt{Place}s and \texttt{Transition}s. A \texttt{Place} has any number of \texttt{src} or \texttt{dst} \texttt{Transition}s. Similarly, a \texttt{Transition} has at least one \texttt{src} and \texttt{dst} \texttt{Place}. In this example, the metamodel in Figure~\ref{fig:original_mm} is to be evolved so as to support weighted connections between \texttt{Place}s and \texttt{Transition}s and between \texttt{Transition}s and \texttt{Place}s.

The evolved metamodel is shown in Figure~\ref{fig:evolved_mm}. \texttt{Place}s are connected to \texttt{Transition}s via instances of \texttt{PTArc}. Likewise, \texttt{Transition}s are connected to \texttt{Place}s via \texttt{TPArc}. Both \texttt{PTArc} and \texttt{TPArc} inherit from \texttt{Arc}, and therefore can be used to specify a \texttt{weight}.

Models that conformed to the original metamodel might not conform to the evolved metamodel. The following strategy can be used to migrate models from the original to the evolved metamodel:

\begin{enumerate}
	\item For every instance, t, of \texttt{Transition}: 
	\subitem For every \texttt{Place}, s, referenced by the \texttt{src} feature of t: 
	\subsubitem Create a new instance, arc, of \texttt{PTArc}. 
	\subsubitem Set s as the \texttt{src} of arc. 
	\subsubitem Set t as the \texttt{dst} of arc. 
	\subsubitem Add arc to the \texttt{arcs} reference of the \texttt{Net} referenced by t.
	
	\subitem For every \texttt{Place}, d, referenced by the \texttt{dst} feature of t: 
	\subsubitem Create a new instance, arc, of \texttt{TPArc}. 
	\subsubitem Set t as the \texttt{src} of arc. 
	\subsubitem Set d as the \texttt{dst} of arc. 
	\subsubitem Add arc to the \texttt{arcs} reference of the \texttt{Net} referenced by t.
	
	\item And nothing else changes.
\end{enumerate}

\subsection{Existing Approaches}
\label{subsec:existing_migration_languages}
Using the above example, the existing approaches for specifying and executing model migration strategies are now compared. From this comparison, the strengths and weakness of each approach are highlighted and requirements for a model migration language are synthesised in the sequel.

\subsubsection{Manual Specification with Model-to-Model Transformation}
\label{subsubsec:m2m}

A model-to-model transformation specified between original and evolved metamodel can be used for performing model migration. Part of the model migration for the Petri nets metamodel is codified with the Atlas Transformation Language (ATL) \cite{jouault05transforming} in Listing~\ref{lst:atl}. Rules for migrating \texttt{Places} and \texttt{TPArcs} have been omitted for brevity, but are similar to the \texttt{Nets} and \texttt{PTArcs} rules.

Model transformation in ATL is specified using \texttt{rule}s, which transform source model elements (specified using the \texttt{fr\-om} keyword) to target model elements (specified using \texttt{to} keyword). For example, the \texttt{Nets} rule on line 1 of Listing~\ref{lst:atl} transforms an instance of \texttt{Net} from the original (source) model to an instance of \texttt{Net} in the evolved (target) model. The source model element (the variable \texttt{o} in the \texttt{Net} rule) is used to populate the target model element (the variable \texttt{m}). ATL allows rules to be specified as \emph{lazy} (not scheduled automatically and applied only when called by other rules).

The \texttt{Transitions} rule in Listing~\ref{lst:atl} codifies in ATL the migration strategy described previously. The rule is executed for each \texttt{Transition} in the original model, \texttt{o}, and constructs a \texttt{PTArc} (\texttt{TPArc}) for each reference to a \texttt{Place} in \texttt{o.src} (\texttt{o.dst}). Lazy rules must be used to produce the arcs to prevent circular dependencies with the \texttt{Transitions} and \texttt{Places} rules. Here, ATL, a typical rule-based transformation language, is considered and model migration would be similar in QVT. With Kermeta, migration would be specified in an imperative style using statements for copying \texttt{Net}s, \texttt{Place}s and \texttt{Transition}s, and for creating \texttt{PTArc}s and \texttt{TPArc}s.

\begin{lstlisting}[caption=Fragment of the Petri nets model migration in ATL, label=lst:atl, language=ATL]
rule Nets {
	from o : Before!Net
	to m : After!Net ( places <- o.places, transitions <- o.transitions )
}

rule Transitions {
	from o : Before!Transition
	to m : After!Transition (
			name <- o.name,
			"in" <- o.src->collect(p | thisModule.PTArcs(p,o)),
			out  <- o.dst->collect(p | thisModule.TPArcs(o,p))
		)
}

unique lazy rule PTArcs {
	from place : Before!Place, destination : Before!Transition
	to ptarcs : After!PTArc (
			src <- place, dst <- destination, net <- destination.net
		)
}
\end{lstlisting}

In model transformation, \cite{czarnecki06survey} identifies two common categories of relationship between source and target model, \emph{new-target} and \emph{existing-target}. In the former, the target model is constructed afresh by the execution of the transformation, while in the latter, the target model contains the same data as the source model before the transformation is executed. ATL supports both new- and existing-target relationships (the latter is termed a refinement transformation). However, ATL refinement transformations may only be used when the source and target metamodel are the same, as is typical for existing-target transformations. 

In model migration, source and target metamodels differ, and hence existing-target transformations cannot be used to specify model migration strategies. Consequently, model migration strategies are specified with new-target model-to-model transformation languages, and often contain sections for copying from original to migrated model those model elements that have not been affected by metamodel evolution. For the Petri nets example, the \texttt{Nets} rule (in Listing~\ref{lst:atl}) and the \texttt{Places} rule (not shown) exist only for this reason.


\subsubsection{Manual Specification with Ecore2Ecore Mapping}
\label{subsubsec:ecore2ecore}
\cite{hussey06advanced} explain the way in which integration with the model loading mechanisms of the Eclipse Modeling Framework (EMF) \cite{steinberg09emf} can be used to perform model migration. In this approach, the default metamodel loading strategy is augmented with model migration code.

Because EMF binds models to their metamodel (discussed in Section~\ref{subsec:modelling_framework_characteristics}), EMF cannot use an evolved metamodel to load an instance of the original metamodel. Therefore, Hussey and Paternostro's approach requires the metamodel developer to provide a mapping between the metamodelling language of EMF, Ecore, and the concrete syntax used to persist models, XMI. Mappings are specified using a tool that can suggest relationships between source and target metamodel elements by comparing names and types. For the Petri nets example, the mappings shown in Figure~\ref{fig:petri_nets_ecore2ecore} were defined between the original and evolved metamodels.

\begin{figure}[htbp]
	\centering
		\includegraphics[scale=0.75]{5.Implementation/images/petri_nets_ecore2ecore.png}
	\caption{Mappings between the original and evolved Petri nets metamodels}
	\label{fig:petri_nets_ecore2ecore}
\end{figure}

The mappings are used by the EMF XMI parser to determine the metamodel types to which pieces of the XMI will be bound. When a type or feature is not bound, the user must specify a custom migration strategy in Java. For the Petri nets metamodel, the \texttt{src} and \texttt{dst} features of \texttt{Place} and \texttt{Transition} are not bound, because migration is more complicated than a one-to-one mapping. Instead, the migration of the \texttt{src} and \texttt{dst} features is specified with Java.

Model migration is specified on the XMI representation of the model and hence presumes some knowledge of the XMI standard. For example, in XMI, references to other model elements are serialised as a space delimited collection of URI fragments \cite{steinberg09emf}. Listing~\ref{lst:java} shows a section of the Ecore2Ecore model migration for the Petri net example presented above. The method shown converts a \texttt{String} containing URI fragments to a \texttt{Collection} of \texttt{Place}s. The method is used to access the \texttt{src} and \texttt{dst} features of \texttt{Transition}, which no longer exist in the evolved metamodel and hence are not loaded automatically by EMF. To specify the migration strategy for the Petri nets example, the metamodel developer must know the way in which the \texttt{src} and \texttt{dst} features are represented in XMI. The complete listing, not shown here, exceeds 200 lines of code.

\begin{lstlisting}[basicstyle=\ttfamily\footnotesize, flexiblecolumns=true, numbers=left, nolol=true, caption=Java method for deserialising a reference., label=lst:java, language=Java, tabsize=2]
private Collection<Place> toCollectionOfPlaces
(String value, Resource resource) {

  final String[] uriFragments    = value.split(" ");
  final Collection<Place> places = new LinkedList<Place>();
 
  for (String uriFragment : uriFragments) {
		final EObject eObject = resource.getEObject(uriFragment);
		final EClass place    = PetriNetsPackage.eINSTANCE.getPlace();

    if (eObject == null || !place.isInstance(eObject))
      // throw an exception
						
		places.add((Place)eObject);
  }
 
  return places;
}
\end{lstlisting}

\subsubsection{Operator-based Co-evolution with COPE}
\label{subsubsec:cope}

Operator-based approaches to managing co-evolution, such as COPE \cite{herrmannsdoerfer09cope}, provide a library of \emph{co-evolutionary operators}. Each co-evolutionary operator specifies both a metamodel evolution and a corresponding model migration strategy. For example, the ``Make Reference Containment'' operator from COPE \cite{herrmannsdoerfer09cope} evolves the metamodel such that a non-containment reference becomes a containment reference and migrates models such that the values of the evolved reference are replaced by copies. By composing co-evolutionary operators, metamodel evolution can be performed and a migration strategy can be generated without writing any code.

To perform metamodel evolution using an operator-based approach, the library of co-evolutionary operators must be integrated with tools for editing metamodels. COPE provides integration with the EMF tree-based metamodel editor. Operators may be applied to an EMF metamodel, and COPE tracks their application. Once metamodel evolution is complete, a migration strategy can be generated automatically from the record of changes maintained by COPEs. The migration strategy is distributed along with the updated metamodel, and metamodel users choose when to execute the migration strategy on their models.

To be effective, operator-based approaches must provide a rich yet navigable library of co-evolutionary operators, as discussed in Section~\ref{subsec:co-evolution_categorisation}. To this end, COPE allows model migration strategies to be specified manually when no co-evolutionary operator is appropriate. Rather than use either of the two manual specification approaches discussed above (model-to-model transformation and Ecore2Ecore mapping), COPE employs a fundamentally different approach using an existing-target transformation.

As discussed above, existing-target transformations cannot be used for specifying model migration strategies as the source (original) and target (evolved) metamodels differ. However, models can be structured independently of their metamodel using a metamodel-independent syntax, such as the one introduced in Section~\ref{sec:mmi_syntax}. Figure~\ref{fig:cope_mmi} shows a simplification of the metamodel-independent syntax used by COPE. By using a metamodel-independent syntax as an intermediary, an existing-target transformation can be used for performing model migration. Further details of this technique are given in \cite{herrmannsdoerfer09cope}.

\begin{figure}[tbp]
  \centering
  \includegraphics[scale=0.75]{5.Implementation/cope_mm.pdf}
  \caption[The metamodel-independent representation used by COPE]{Simplification of the metamodel-independent representation used by COPE, based on \cite{herrmannsdoerfer09cope}.}
  \label{fig:cope_mmi}
\end{figure}

Listing~\ref{lst:cope} shows the COPE model migration strategy for the Petri net example given above\footnote{In Listing~\ref{lst:cope}, some of the concrete syntax has been changed in the interest of readability.}. Most notably, slots for features that no longer exist must be explicitly \texttt{unset}. In Listing~\ref{lst:cope}, slots are \texttt{unset} on four occasions (on lines 2, 8 and 16), once for each feature that exists in the original metamodel but not the evolved metamodel. Namely, these features are: \texttt{src} and \texttt{dst} of \texttt{Transition} and of \texttt{Place}. Failing to \texttt{unset} slots that do not conform with the evolved metamodel causes migration to fail with an error.

\begin{lstlisting}[caption=Petri nets model migration in COPE, label=lst:cope, language=COPE]
for (transition in Transition.allInstances) {
  for (source in transition.unset(`src')) {
    def arc = petrinets.PTArc.newInstance()
    arc.src = source;  arc.dst = transition;
    arc.net = transition.net
  }

  for (destination in transition.unset(`dst')) {
    def arc = petrinets.TPArc.newInstance() 
    arc.src = transition; arc.dst = destination;
    arc.net = transition.net
  }
}

for (place in Place.allInstances) {
  place.unset(`src');  place.unset(`dst');
}
\end{lstlisting}


\subsection{Requirements Identification}
\label{subsec:analysis}
By analysing the languages used for model migration in existing approaches to managing developer-driven co-evolution, requirements were derived for a domain-specific language for specifying and executing model migration. The derivation of the requirements is now summarised, by considering two orthogonal concerns: the source-target relationship of the language used for specifying migration strategies and the way in which models are represented during migration. %and the structures provided by the language for specifying and re-using migration strategies.


\subsubsection{Source-Target Relationship}
When migration is specified as a new-target transformation, as was the case for the ATL transformation shown in Listing~\ref{lst:atl}, model elements that have not been affected by metamodel evolution must be explicitly copied from the original to the migrated model. When migration is specified as an existing-target transformation, as was the case for the COPE transformation shown in Listing~\ref{lst:cope}, model elements and values that no longer conform to the target metamodel must be explicitly removed from the migrated model. By contrast, the Ecore2Ecore approach does not require explicit copying or unsetting code. Instead, the relationship between original and evolved metamodel elements is captured in a mapping model specified by the metamodel developer. The mapping model can be derived automatically and customised by the metamodel developer. To explore the appropriateness for model migration of an alternative to new- and existing-target transformations, the following requirement was derived: \emph{The migration language must \textbf{automatically} copy every model element that conforms to the evolved metamodel from original to migrated model, and must not automatically copy any model element that does not conform to the evolved metamodel from original to migrated model.}


\subsubsection{Model Representation}
When using the Ecore2Ecore approach, model elements that do not conform to the evolved metamodel are accessed by maniuplating XMI. Consequently, the metamodel developer must be familiar with XMI and must perform tasks such as dereferencing URI fragments (Listing~\ref{lst:java}) and type conversion. Transformation languages abstract away from the underlying storage representation of models (such as XMI) by using a modelling framework to load, store and access models. Consequently, migration strategies written in a transformation language need not manage details specific to XMI, such as dereferencing URI fragments. Furthermore, decoupling a transformation language from the model representation facilitates interoperability with more than one modelling technology, as demonstrated by the languages of the Epsilon platform \cite{kolovos09thesis}. Consequently, the following requirement was identified: \emph{The migration language must not expose the underlying representation of original or migrated models.}

To apply co-evolution operators, COPE requires the metamodel developer to use a specialised metamodel editor, which can manipulate only metamodels defined with EMF. Similarly, the mapping tool used in the Ecore2Ecore approach can be used only with metamodels defined with EMF. Although EMF is arguably the most widely-used modelling framework, other frameworks are used today. Adapting to interoperate with new systems is recognised as a common reason for software evolution \cite{sjoberg93quantifying}, and as such migration between modelling frameworks should be regarded as a possible use case for a model migration language. To better support integration with modelling frameworks other than EMF, the following requirement was derived: \emph{The migration language must be loosely coupled with modelling frameworks and must not assume that models and metamodels will be represented in EMF.}
%!TEX root = /Users/louis/Documents/PhD/Deliverables/Thesis/thesis.tex

\section{Epsilon Flock: A Model Migration Language}
\label{sec:flock}
Driven by the analysis presented above, a domain-specific language for model migration, Epsilon Flock (subsequently referred to as Flock), was designed and implemented. Section~\ref{subsec:flock_design} discusses the principle tenets of Flock, which include user-defined migration rules and a novel algorithm for relating source and target model elements. In Section~\ref{subsec:flock_examples}, Flock is demonstrated via application to three examples of model migration. The work described in this section has been published in \cite{rose10flock}.

\subsection{Design and Implementation}
\label{subsec:flock_design}
Flock is a rule-based transformation language that mixes declarative and imperative parts. Its style is inspired by hybrid model-to-model transformation languages such as the Atlas Transformation Language \cite{jouault05transforming} and the Epsilon Transformation Language \cite{kolovos08etl}. Flock has a compact syntax. Much of its design and implementation is focused on the runtime. The way in which Flock relates source to target elements is novel; it is neither a new- nor an existing-target relationship. Instead, elements are copied conservatively, as described in Section~\ref{subsubsec:conservative_copying}.

Like Epsilon HUTN (Section~\ref{subsec:epsilon_hutn}), Flock is built atop Epsilon, which was described in Section~\ref{subsec:epsilon}. In particular, Flock uses the Epsilon Model Connectivity layer to provide interoperability with several modelling frameworks, and the Epsilon Object Language (EOL) for specifying the imperative part of user-defined migration rules.

\subsubsection{Abstract Syntax}
\label{subsubsec:abstract_syntax}
As illustrated by Figure~\ref{fig:abstract_syntax}, Flock migration strategies are organised into modules (\texttt{Fl\-ockMo\-du\-le}). Flock modules inherit from EOL modules (\texttt{Eo\-lMod\-ule}) and hence provide language constructs for specifying user-defined operations and for re-using modules. Flock modules comprise any number of rules (\texttt{Ru\-le}). Each rule has an original metamodel type (\texttt{or\-ig\-in\-alTy\-pe}) and can optionally specify a \texttt{gu\-ard}, which is either an EOL statement or a block of EOL statements. \texttt{Mi\-gr\-ateRu\-le}s must specify an evolved metamodel type (\texttt{ev\-ol\-vedTy\-pe}) and/or a \texttt{bo\-dy} comprising a block of EOL statements.

\begin{figure}
  \centering
  \includegraphics[scale=0.75]{5.Implementation/flock_abstract_syntax.pdf}
  \caption{The abstract syntax of Flock.}
  \label{fig:abstract_syntax}
\end{figure}

\subsubsection{Concrete Syntax}
\label{subsubsec:concrete_syntax}

Listing~\ref{lst:flock_concrete_syntax} shows the concrete syntax of migrate and delete rules. All rules begin with a keyword indicating their type (either \texttt{migrate} or \texttt{delete}), followed by the original metamodel type. Guards are specified using the \texttt{when} keywords. Migrate rules may also specify an evolved metamodel type using the \texttt{to} keyword and a \texttt{body} as a (possibly empty) sequence of EOL statements.

Note that Flock does not define a create rule. The creation of new model elements is instead encoded in the imperative part of a migrate rule specified on the containing type.

\begin{lstlisting}[float=tbp, caption=Concrete syntax of migrate and delete rules., label=lst:flock_concrete_syntax, language=Flock]
migrate <originalType> (to <evolvedType>)?
(when (:<eolExpression>)|({<eolStatement>+}))? {
	<eolStatement>*
} 

delete <originalType>
(when (:<eolExpression>)|({<eolStatement>+}))?
\end{lstlisting}

\subsubsection{Execution Semantics}
\label{subsubsec:execution_semantics}
When executed, a Flock module consumes an original model, \texttt{O}, and constructs a migrated model, \texttt{M}. The transformation is performed in three phases: rule selection, equivalence establishment and rule execution. The behaviour of each phase is described below, and the first example in Section~\ref{subsec:flock_examples} demonstrates the way in which a Flock module is executed.

\paragraph{Rule Selection}
The rule selection phase determines an \emph{applicable} rule for every model element, \texttt{e}, in \texttt{O}. As such, the result of the rule selection phase is a set of pairs of the form \texttt{<r,e>} where \texttt{r} is a migration rule.

A rule, \texttt{r}, is \emph{applicable} for a model element, \texttt{e}, when the original type of \texttt{r} is the same type as (or is a supertype of) the type of \texttt{e}; and the guard part of \texttt{r} is satisfied by \texttt{e}.

The rule selection phase has the following behaviour:

\begin{itemize}
	\item For each original model element, \texttt{e}, in \texttt{O}:
	\subitem $-$ Identify for \texttt{e} the set of all applicable rules, \texttt{R}. Order \texttt{R} by the occurrence of rules in the Flock source file.
	\subsubitem $\circ$ If \texttt{R} is empty, let \texttt{r} be a default rule, which has the type of \texttt{e} as both its original and evolved type, and an empty body.
	\subsubitem $\circ$ Otherwise, let \texttt{r} be the first element of \texttt{R}.
	\subitem $-$ Add the pair \texttt{<r,e>} to the set of selected rules.
\end{itemize}


\paragraph{Equivalence Establishment}
The equivalence establishment phase creates an equivalent model element, \texttt{e'}, in M for every pair of rules and original model elements, \texttt{<r,e>}. The equivalent establishment phase produces a set of triples of the form \texttt{<r,e,e'>}, and has the following behaviour:

\begin{itemize}
	\item For each pair \texttt{<r,e>} produced by the rule selection phase:
	\subitem $-$ If \texttt{r} is a delete rule, do nothing.
	\subitem $-$ If \texttt{r} is a migrate rule:
	\subsubitem $\circ$ Create a model element, \texttt{e'}, in M. The type of \texttt{e'} is determined from the the \texttt{evolvedType} (or the \texttt{originalType} when no \texttt{evolvedType} has been specified) of \texttt{r}.
	\subsubitem $\circ$ Copy the data contained in \texttt{e} to \texttt{e'} (using the \emph{conservative copy} algorithm described in the sequel).
	\subsubitem $\circ$ Add the triple \texttt{<r,e,e'>} to the set of equivalences.
\end{itemize}
	
\paragraph{Rule Execution}
The final phase executes the imperative part of the user-defined migration rules on the set of triples \texttt{<r,e,e'>}, and has the following behaviour:

\begin{itemize}
	\item For each triple \texttt{<r,e,e'>} produced by the equivalence establishment phase:
	\subitem $-$ Bind \texttt{e} and \texttt{e'} to EOL variables named \texttt{original} and \texttt{migrated}, respectively.
	\subitem $-$ Execute the body of \texttt{r} with EOL.
\end{itemize}


\subsubsection{Conservative Copy}
\label{subsubsec:conservative_copying}
Flock contributes an algorithm, termed \emph{conservative copy}, that copies model elements from original to migrated model only when those model elements conform to the evolved metamodel. Conservative copy is a hybrid of the new- and existing-target source-target relationships that are commonly used in M2M transformation \cite{czarnecki06survey}.

Conservative copy operates on an original model element, \texttt{e}, and its equivalent model element in the migrated model, \texttt{e'}, and has the following behaviour:

\begin{itemize}
	\item For each metafeature, \texttt{f} for which \texttt{e} has specified a value:
		\subitem $-$ Find a metafeature, \texttt{f'}, of \texttt{e'} with the same name as \texttt{f}.
			\subsubitem $\circ$ If no equivalent metafeature can be found, do nothing.
			\subsubitem $\circ$ Otherwise, copy the original value (\texttt{e.f}) to produce a migrated value (\texttt{e'.f'}) if and only if the migrated value conforms to \texttt{f'}.
\end{itemize}

The definition of conformance varies over modelling frameworks. Typically, conformance between a value, \texttt{v}, and a feature, \texttt{f}, specifies at least the following constraints:

\begin{itemize}
	\item The size of \texttt{v} must be greater than or equal to the lowerbound of \texttt{f}.
	\item The size of \texttt{v} must be less than or equal to the upperbound of \texttt{f}.
	\item The type of \texttt{v} must be the same as or a subtype of the type of \texttt{f}.
\end{itemize}


Epsilon includes a model connectivity layer (EMC), which provides a common interface for accessing and persisting models. Currently, EMC provides drivers for several modelling frameworks, permitting management of models defined with EMF, the Metadata Repository (MDR), Z or XML. To support migration between metamodels defined in heterogenous modelling frameworks, EMC was extended to include a conformance checking service, and each EMC driver to provide conformance checking semantics specific to its modelling framework. Flock implements conservative copy by delegating conformance checking responsibilities to EMC. 

Finally, some categories of model value must be converted before being copied from the original to the migrated model. Again, the need for and semantics of this conversion varies over modelling frameworks. For example, reference values typically require conversion before copying because, once copied, they must refer to elements of the migrated rather than the original model. In this case, the set of equivalences (\texttt{<r,e,e'>}) can be used to perform the conversion. In other cases, the target modelling framework must be used to perform the conversion, such as when EMF enumeration literals are copied.

 
\subsubsection{Development and User Tools}
As discussed in Section~\ref{sec:analysing_existing_techniques}, models and metamodels are typically kept separate. Flock migration strategies can be distributed by the metamodel developer in two ways. An extension point defined by Flock provides a generic user interface for migration strategy execution. Alternatively, metamodel developers can elect to build their own interface, delegating execution responsibility to \texttt{FlockModule}. The latter approach facilities interoperability with, for example, model and source code management systems.


\subsection{Examples}
\label{subsec:flock_examples}
Flock is now demonstrated using three examples of model migration. The first demonstrates the way in which a Flock module is executed and illustrates the semantics of conservative copy. The second describes the way in which the migration of the Petri net co-evolution example (introduced above) can be specified with Flock, and is included for direct comparison with the other languages discussed in Section~\ref{sec:analyis_of_languages_used_for_migration}. The final, larger example demonstrates all of the features of Flock, and is based on changes made to UML class diagrams between versions 1.5 and 2.0 of the UML specification.

\subsubsection{Process-Oriented Example}
The example presented below demonstrates, in detail, the way in which a Flock module executes a model migration strategy. Consider the original and evolved metamodels shown in Figure~\ref{fig:cc_eg_mms}, which are simplifications of two versions of the metamodel from the MDE project described in Appendix~\ref{ProcessOriented}. 

\begin{figure}
	\centering
	\subfigure[Original metamodel.]
	{
	    \label{fig:cc_eg_original}
	    \includegraphics[width=8.5cm]{5.Implementation/images/cc_eg_original.pdf}
	}
	\subfigure[Evolved metamodel.]
	{
	    \label{fig:cc_eg_evolved}
	    \includegraphics[width=8.5cm]{5.Implementation/images/cc_eg_evolved.pdf}
	}
	\caption{Exemplar Process-Oriented metamodel evolution}
\label{fig:cc_eg_mms}
\end{figure}

The original metamodel, shown in Figure~\ref{fig:cc_eg_original}, has been evolved to distinguish between \texttt{Co\-nn\-ec\-ti\-o\-nP\-oi\-nt}s that are a reader for a \texttt{Ch\-an\-n\-el} and \texttt{Co\-nn\-ec\-ti\-o\-nP\-oi\-nt}s that are a writer for a \texttt{Ch\-an\-n\-el} by making \texttt{Co\-nn\-ec\-ti\-o\-nP\-oi\-nt} abstract and introducing two subtypes, \texttt{Re\-a\-di\-ngCo\-nn\-ec\-ti\-o\-nP\-oi\-nt} and \texttt{Wr\-i\-ti\-ngCo\-nn\-ec\-ti\-o\-nP\-oi\-nt}, as shown in Figure~\ref{fig:cc_eg_evolved}.

Suppose that the model shown in Figure~\ref{fig:cc_eg_model}, which conforms to the original metamodel in Figure~\ref{fig:cc_eg_original} is to be migrated. The model comprises thre \texttt{Pr\-oc\-e\-ss}es named \emph{delta}, \emph{prefix} and \emph{minus}; three \texttt{Ch\-an\-n\-el}s named \emph{a}, \emph{b} and \emph{c}; and six \texttt{Co\-nn\-ec\-ti\-o\-nP\-oi\-nt}s named \emph{a?}, \emph{a!}, \emph{b?}, \emph{b!}, \emph{c?} and \emph{c!}.

\begin{figure}[htbp]
	\centering
		\includegraphics[scale=0.5]{A.2.ProcessOriented/images/4_model.png}
	\caption{Exemplar Process-Oriented model prior to migration}
	\label{fig:cc_eg_model}
\end{figure}

\begin{lstlisting}[float=tbp, caption=Redefining equivalences for the Component model migration., label=lst:cc_eg_rules, language=Flock]
migrate ConnectionPoint to ReadingConnectionPoint when: original.outgoing.isDefined()
migrate ConnectionPoint to WritingConnectionPoint when: original.incoming.isDefined()
\end{lstlisting}

For the migration strategy shown in Listing~\ref{lst:cc_eg_rules}, the Flock module will perform the following steps. Firstly, the rule selection phase produces a set of pairs \texttt{<r,e>}. For each \texttt{ConnectionPoint}, the guard part of the user-defined rules control which rule will be selected. \texttt{ConnectionPoint}s \texttt{a!}, \texttt{b!} and \texttt{c!} have outgoing \texttt{Channel}s (\texttt{a}, \texttt{b} and \texttt{c} respectively) and hence the migration rule on line 1 is selected. Similarly, the \texttt{ConnectionPoint}s \texttt{a?}, \texttt{b?} and \texttt{c?} have incoming \texttt{Channel}s (\texttt{a}, \texttt{b} and \texttt{c} respectively) and hence the migration rule on line 2 is selected. There is no \texttt{ConnectionPoint} with both an outgoing and an incoming \texttt{Channel}, but if there were, the first applicable rule (i.e. the rule on line 1) would be selected. For the other model elements (the \texttt{Process}es and \texttt{Channel}s) no user-defined rules are applicable, and so default rules are used instead. A default rule has an empty body and identical original and evolved types. In other words, a default rule for the \texttt{Process} type is equivalent to the user-defined rule: \texttt{migrate Process to Process \{\}}

Secondly, the equivalence establishment phase creates an element, \texttt{e'}, in the migrated model for each pair \texttt{<r,e>}. For each \texttt{ConnectionPoint}, the evolved type of the selected rule (\texttt{r}) controls the type of \texttt{e'}. The rule on line 1 of Listing~\ref{lst:cc_eg_rules} was selected for the \texttt{ConnectionPoint}s \texttt{a!}, \texttt{b!} and \texttt{c!} and hence an equivalent element of type \texttt{ReadingConnectionPoint} is created for \texttt{a!}, \texttt{b!} and \texttt{c!}. Similarly, an equivalent element of type \texttt{WritingConnectionPoint} is created for \texttt{a?}, \texttt{b?} and \texttt{c?}. For the other model elements (the \texttt{Process}es and \texttt{Channel}s) a default rule was selected, and hence the equivalent model element has the same type as the original model element.

Finally, the rule execution phase performs a conservative copy for each original and equivalent model element in the set of triples \texttt{<r,e,e'>} produced by the equivalent establishment phase. The metamodel evolution shown in Figure~\ref{fig:cc_eg_mms} has not affected the \texttt{Pr\-oc\-e\-ss} type, and hence for each \texttt{Pr\-oc\-e\-ss} in the original model, conservative copy will create a \texttt{Pr\-oc\-e\-ss} in the migrated model and copy the values of all features. For each \texttt{Ch\-an\-n\-el} in the original model, conservative copy will create an equivalent \texttt{Ch\-an\-n\-el} in the migrated model and copy the value of the \texttt{na\-me} feature from original to migrated model element. However, the values of the \texttt{re\-ad\-er} and \texttt{wr\-it\-er} features will not be copied by conservative copy because the type of these features has changed (from \texttt{Co\-nn\-ec\-ti\-o\-nP\-oi\-nt} to \texttt{Re\-a\-di\-ngCo\-nn\-ec\-ti\-o\-nP\-oi\-nt} and \texttt{Wr\-i\-ti\-ngCo\-nn\-ec\-ti\-o\-nP\-oi\-nt}, respectively). The values of the \texttt{re\-ad\-er} and \texttt{wr\-it\-er} features in the original model will not conform to the \texttt{re\-ad\-er} and \texttt{wr\-it\-er} features in the evolved metamodel. Finally, the values of the \texttt{na\-me}, \texttt{in\-co\-mi\-ng} and \texttt{ou\-tg\-oi\-ng} features of the \texttt{Co\-nn\-ec\-ti\-o\-nP\-oi\-nt} class have not evolved, and hence are copied directly from original to equivalent model elements.

The rule execution phase also executes the body of each rule, \texttt{r}, for every triple in the set \texttt{<r,e,e'>}. The user-defined rules in Listing~\ref{lst:cc_eg_rules} have no body, and hence no further execution is performed in this case.

\subsubsection{Petri Nets in Flock}
The exemplar Petri net metamodel evolution is now revisited to demonstrate the core functionality of Flock. In Listing~\ref{lst:flock}, \texttt{Net}s and \texttt{Place}s are migrated automatically. Unlike the ATL migration strategy (Listing~\ref{lst:atl}), no explicit copying rules are required. Compared to the COPE migration strategy (Listing~\ref{lst:cope}), the Flock migration strategy does not explicitly unset the original \texttt{src} and \texttt{dst} features of \texttt{Transition}.

\begin{lstlisting}[caption=Petri nets model migration in Flock, label=lst:flock, language=Flock]
migrate Transition {
  for (source in original.src) {
    var arc := new Migrated!PTArc;
    arc.src := source.equivalent();  arc.dst := migrated;
    arc.net := original.net.equivalent();
  }

  for (destination in original.dst) {
    var arc := new Migrated!TPArc;
    arc.src := migrated;  arc.dst := destination.equivalent();
    arc.net := original.net.equivalent();
  }
}
\end{lstlisting}

\subsubsection{UML Class Diagrams in Flock}
Figure~\ref{fig:uml_mms} illustrates a subset of the changes made between UML 1.5 and UML 2.0. Only class diagrams are considered, and features that did not change are omitted. In Figure~\ref{fig:original_uml_mm}, association ends and attributes are specified explicitly and separately. In Figure~\ref{fig:evolved_uml_mm}, the \texttt{Pr\-op\-er\-ty} class is used instead. The Flock migration strategy (Listing~\ref{lst:flock-uml}) for Figure~\ref{fig:uml_mms} is now discussed.

\begin{lstlisting}[caption=UML model migration in Flock, label=lst:flock-uml, language=Flock]
migrate Association {
	migrated.memberEnds := original.connections.equivalent();
}

migrate Class {
	var fs := original.features.equivalent();
	migrated.operations := fs.select(f|f.isKindOf(Operation));
	migrated.attributes := fs.select(f|f.isKindOf(Property));
	migrated.attributes.addAll(original.associations.equivalent())
}

delete StructuralFeature when: original.targetScope <> #instance

migrate Attribute to Property {
	if (original.ownerScope = #classifier) {
		migrated.isStatic = true;		
	}
}
migrate Operation {
	if (original.ownerScope = #classifier) {
		migrated.isStatic = true;
	}
}

migrate AssociationEnd to Property {
	if (original.isNavigable) {
		original.association.equivalent().navigableEnds.add(migrated)
	}
}
\end{lstlisting}

Firstly, \texttt{At\-tr\-ib\-ut\-e}s and \texttt{As\-so\-ci\-at\-i\-onEn\-d}s are now modelled as \texttt{Pr\-o\-pe\-rt\-ies} (lines 16 and 28). In addition, the \texttt{As\-so\-ci\-at\-i\-on\#na\-vi\-ga\-b\-leEn\-ds} reference replaces the \texttt{As\-so\-ci\-at\-i\-onE\-nd\#isN\-av\-ig\-ab\-le} attribute; following migration, each navigable \texttt{As\-so\-ci\-at\-i\-onE\-nd} must be referenced via the \texttt{na\-vi\-ga\-bl\-eEn\-ds} feature of its \texttt{As\-so\-ci\-at\-ion} (lines 29-31).

In UML 2.0, \texttt{St\-ru\-ct\-ur\-alFe\-at\-ur\-e\#o\-wn\-er\-Sc\-op\-e} has been replaced by \texttt{\#i\-sS\-ta\-ti\-c} (lines 17-19 and 23-25). The UML 2.0 specification states that \texttt{Sc\-op\-eKi\-nd\#cl\-as\-si\-fi\-er} should be mapped to true, and \texttt{\#i\-ns\-ta\-nce} to false. 

The UML 1.5 \texttt{St\-ru\-ct\-ur\-alFe\-at\-ur\-e\#t\-ar\-g\-et\-Sc\-op\-e} feature is no longer supported in UML 2.0, and no migration path is provided. Consequently, line 14 deletes any model element whose \texttt{t\-ar\-g\-et\-Sc\-op\-e} is not the default value.

\begin{figure}
	\centering
	\subfigure[Original metamodel.]
	{
	    \label{fig:original_uml_mm}
	    \includegraphics[width=10.5cm]{5.Implementation/images/uml_class_before.pdf}
	}
	\subfigure[Evolved metamodel.]
	{
	    \label{fig:evolved_uml_mm}
	    \includegraphics[width=10.5cm]{5.Implementation/images/uml_class_after.pdf}
	}
	\caption{Exemplar UML metamodel evolution}
\label{fig:uml_mms}
\end{figure}

Finally, \texttt{C\-la\-ss\#fe\-at\-ur\-es} has been split to form \texttt{C\-la\-ss\#op\-er\-at\-io\-ns} and \texttt{\#at\-tr\-ib\-ut\-es}. Lines 8 and 10 partition features on the original \texttt{Cl\-a\-ss}. \texttt{Cl\-as\-s\#a\-ss\-oc\-ia\-ti\-on\-s} has been removed in UML 2.0, and \texttt{As\-so\-ci\-at\-i\-onEn\-d}s are instead stored in \texttt{Cl\-a\-ss\#at\-tr\-ib\-ut\-es} (line 11).


\subsubsection{Summary}
Table~\ref{tab:differences} illustrates several characterising differences between Flock and the related languages presented in Section~\ref{sec:analyis_of_languages_used_for_migration}. Due to its conservative copying algorithm, Flock is the only language to provide both automatic copying and unsetting. The evaluation presented in Section~\ref{sec:quantitive} explores the extent to which automatic copying and unsetting affect the conciseness of migration strategies.

All of the approaches considered in Table~\ref{tab:differences} support EMF, arguably the most widely used modelling framework. The Ecore2Ecore approach, however, requires migration to be encoded at the level of the underlying model representation XMI. Both Flock and ATL support other modelling technologies, such as MDR and XML. However, ATL does not automatically copy model elements that have not been affected by metamodel changes. Therefore, migration between models of different technologies with ATL requires extra statements in the migration strategy to ensure that the conformance constraints of the target technology are satisfied. Because it delegates conformance checking to an EMC driver, Flock requires no such checks.

\begin{table}[b]
	\centering
	\begin{tabular}{|c|c|c|c|}
		\hline
		             & \multicolumn{2}{c|}{\textbf{Automatic}} & \textbf{Modelling} \\
		\textbf{Tool}& \textbf{Copy} & \textbf{Unset}          & \textbf{technologies} \\
		\hline
		Ecore2Ecore  & \tick             & \cross              & XMI                    \\
		\hline
		ATL          & \cross            & \tick               & EMF, MDR, KM3, XML     \\
		\hline
		COPE         & \tick             & \cross              & EMF                    \\
		\hline
		Flock        & \tick             & \tick               & EMF, MDR, XML, Z       \\
		\hline
	\end{tabular}
	\label{tab:differences}
	\caption{Properties of model migration approaches}
\end{table}

A more thorough examination of the similarities and differences between Flock and other migration strategy languages is provided by the evaluation presented in Chapter~\ref{Evaluation}.



\section{Chapter Summary}
Three structures for identifying and managing co-evolution have been designed and implemented to approach the thesis requirements outlined in Chapter~\ref{Analysis}. The way in which modelling frameworks implicitly enforce conformance makes managing non-conformant models challenging, and the proposed metamodel-independent syntax (Section~\ref{sec:mmi_syntax}) extends modelling frameworks to facilitate the management of non-conformant models. The proposed textual modelling notation Section~\ref{sec:notation}, Epsilon HUTN, provides a human-usable notation as an alternative to XMI for managing user-driven co-evolution. Finally, the model migration language (Section~\ref{sec:flock}) contributes a domain-specific language for describing model migration.

The metamodel-independent syntax is a modelling framework extension that makes explicit the conformance relationship between models and metamodels. By binding models not to their metamodel but to a generic metamodel, the metamodel-independent syntax allows non-conformant models to be managed with modelling tools and model management operations. Furthermore, conformance checking is provided as a service, which can be scheduled at any time, and not just when models are loaded. The metamodel-independent syntax has been integrated with Concordance \cite{rose10concordance} to provide a metamodel installation process that automatically reports conformance problems, and underpins the implementation of the second structure described in this chapter, a textual modelling notation.

For performing user-driven co-evolution, the textual modelling notation described in Section~\ref{sec:notation} provides an alternative to XMI. Unlike XMI, the notation introduced in this chapter implements the OMG standard for Human-Usable Textual Notation (HUTN) \cite{hutn} and is optimised for human usability. Epsilon HUTN, introduced here, is presently the sole reference implementation of HUTN. Constructing Epsilon HUTN atop the metamodel-independent syntax allows Epsilon HUTN to provide incremental and background conformance checking, and an XMI-to-HUTN transformation for loading non-conformant models. Section~\ref{sec:exemplar_user-driven_co-evo} explores the benefits and drawbacks of using the metamodel-independent syntax and Epsilon HUTN together to perform user-driven co-evolution.

The domain-specific language described in Section~\ref{sec:flock}, Epsilon Flock, combines several concepts from existing model-to-model transformation languages to form a language tailored to model migration. In particular, Flock contributes a novel mechanism for relating source and target model elements termed conservative copy, which is a hybrid of new- and existing-target styles of model-to-model transformation. Flock is built atop Epsilon and hence interoperates transparently with several modelling technologies via the Epsilon Model Connectivity layer. Conservative copy is compared with new- and existing-target styles of transformation in Section~\ref{sec:quantitive}, and Flock is evaluated with respect to other co-evolution tools in Sections~\ref{sec:collaborative_comparison} and~\ref{sec:ttc} respectively.

The metamodel-independent syntax, Epsilon HUTN, Epsilon Flock and Concordance have been released as part of Epsilon in the Eclipse GMT \cite{gmt} project, which is the research incubator of arguably the most widely used MDE modelling framework, EMF. By re-using parts of Epsilon, the structures were implemented more rapidly than would have been possible when developing the structures independently. In particular, re-using the Epsilon Model Connectivity layer facilitated interoperability of Flock with several MDE modelling frameworks, which was exploited to manage a practical case of model migration in Section~\ref{sec:ttc}.
