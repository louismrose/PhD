%!TEX root = /Users/louis/Documents/PhD/Deliverables/Thesis/thesis.tex

\chapter{Implementation}
\label{Implementation}
In Section~\ref{sec:requirements_identification}, specific requirements for structures and processes for managing co-evolution were identified. In this chapter, the way in which this thesis approaches those requirements is described. Several related solutions were implemented, including domain-specific languages, automation and extensions to existing modelling technologies. Figure~\ref{fig:implementation_overview} summarises the structure of this chapter. To better support co-evolution and to overcome restrictions with existing modelling frameworks, a metamodel-independent syntax was devised and implemented, enabling model and metamodel decoupling and consistency checking (Section~\ref{sec:mmi_syntax}). To address some of the challenges faced in user-driven co-evolution, an alternative, human-usable modelling notation was implemented, according to an OMG specification (Section~\ref{sec:hutn}). Languages used to specify model migration were identified, analysed and compared, leading to the derivation and implementation of a new model transformation language tailored for model migration and centred around a novel approach to relating source and target model elements (Section~\ref{sec:flock}). 

\begin{figure}[htbp]
  \begin{center}
    \leavevmode
    \includegraphics[scale=0.66]{5.Implementation/overview.pdf}
  \end{center}
  \caption{Implementation chapter overview.}
  \label{fig:implementation_overview}
\end{figure}


\section{Metamodel-Independent Syntax}
\label{sec:mmi_syntax}
Section~\ref{subsec:modelling_framework_characteristics} discussed the way in which modelling frameworks implicitly enforce conformance. Because of this, modelling frameworks cannot be used to load non-conformant models, and provide little support for checking the conformance of a model with other metamodels or other versions of a metamodel. In Section~\ref{sec:requirements_identification}, these concerns lead to the identification of the following requirement: \emph{This thesis must investigate the extension of existing modelling frameworks to support the loading of non-conformant models and conformance checking of models against other metamodels.}

This section describes the way in which existing modelling frameworks load and store models using a metamodel-specific syntax. An alternative storage representation is motivated by highlighting the problems that a metamodel-specific syntax poses for managing and automating co-evolution. The way in which automatic consistency checking can be performed using the alternative storage representation is demonstrated. The work presented in this section was published in \cite{rose09enhanced}.


\subsection{Model Storage Representation}
Throughout a model-driven development process, modelling frameworks are used to load and store models. XML Metadata Interchange (XMI) \cite{xmi}, the OMG standard for exchanging MOF-based models, is typically the canonical model representation used by contemporary modelling frameworks. XMI specifies the way in which models should be represented in XML.

An XMI document defines one or more namespaces from which type information is drawn. XMI itself provides a namespace for specifying the version of XMI begin used, for example. Metamodels are referenced via namespaces, allowing elements that instantiate metamodel types to be specified.

As discussed in Section~\ref{subsec:modelling_framework_characteristics}, modelling frameworks bind a model to its metamodel. The metamodel defines the way in which model elements will be bound, and frequently, binding is strongly-typed: each metamodel type is mapped to a corresponding type in the underlying programming language. When a metamodel changes, binding may fail when a model no longer conforms to its metamodel, as the following example demonstrates. 

Listing~\ref{lst:xmi} shows XMI for an exemplar model conforming to a metamodel that defines \texttt{Person} as a metaclass with two features: a string-valued \texttt{name} and an optional reference to another \texttt{Person}, \texttt{mother}.

\begin{lstlisting}[caption=Exemplar person model in XMI, label=lst:xmi, language=XML]
<?xml version="1.0" encoding="ASCII"?>
<xmi:XMI xmi:version="2.0" xmlns:xmi="http://www.omg.org/XMI" xmlns:families="http://www.cs.york.ac.uk/families">
	<families:Person xmi:id="_xNSb8KfZEd,0dNl1iq3EdQ" name="Franz" mother="_6ef33ff010b31df8a39080" />
	<families:Person xmi:id="_6ef33ff010b31df8a39080" name="Julie" />
</xmi:XMI>
\end{lstlisting}

The model shown in Listing~\ref{lst:xmi} contains two \texttt{Person}s, Franz, and Julie. Julie is the mother of Franz. The mother of Julie is not specified. On line 2, the XMI document specifies that the families namespace will be used to refer to types defined by the metamodel with the identifier: http://www.cs.york.ac.uk/families. Each person defines an XMI ID (a universally unique identifier), and a name. The first person, named Franz, specifies a value for the mother attribute.

Binding a model element involves instantiating, in the underlying programming language, the metamodel type, and populating the attributes of the instantiated object with values that correspond to those specified in the model. Because an XMI document refers to metamodel types and features by name, binding fails when a model does not conform to its metamodel. 


\subsection{Binding to a generic metamodel}
For situations when a model does not conform to its metamodel, this thesis proposes an alternative deserialisation mechanism, which binds a model to a \emph{generic} metamodel. A generic metamodel reflects the characteristics of the metamodelling language and consequently every model conforms to the generic metamodel. Figure~\ref{fig:slot_model} shows a minimal version of a generic metamodel for MOF. Model elements are bound to \texttt{Object}, data values to \texttt{Slot}.

\begin{figure}[htbp]
  \centering
  \includegraphics[width=3.3in]{5.Implementation/slot_model.pdf}
  \caption{A generic metamodel.}
  \label{fig:slot_model}
\end{figure}

Using the metamodel in Figure~\ref{fig:slot_model} in conjunction with MOF, conformance constraints can be expressed, as shown below. A minimal subset of MOF is shown in Figure~\ref{fig:mof}.

\begin{figure}[htbp]
  \centering
  \includegraphics[width=3.3in]{5.Implementation/mof.pdf}
  \caption{Minimal MOF metamodel.}
  \label{fig:mof}
\end{figure}

The following constraints between metamodels (e.g. instances of MOF, Figure~\ref{fig:mof}) and models represented with a generic metamodel (e.g. instances of Figure~\ref{fig:slot_model}) can be used to express conformance:

\begin{enumerate}
	\item Each object's type value must be the name of some metamodel class.
	\item Each object's type value must be the name of some non-abstract metamodel class.
	\item Each object must specify a slot for each mandatory feature of its type.
	\item Each slot's feature value must be the name of a metamodel feature. That metamodel feature must belong to the slot's owning object's type.
	\item Each slot must be multiplicity-compatible with its feature. More specifically, each slot must contain at least as many values as its feature's lowerbound, and at most as many values as its feature's upperbound.
  \item Each slot must be type-compatible with its feature.
\end{enumerate}

The way in which type-compatibility is checked depends on the way in which the modelling framework is implemented, and on its underlying programming language. EMF, for example, is implemented in Java and exposes some services for checking the type compatibility of model data with metamodel features. All metamodel features are typed and their types provide methods for determining the underlying programming language representation.

Conformance constraints vary over modelling languages. For example, Ecore, the modelling language of EMF, is similar to but not the same as MOF. For example, metamodel features defined in Ecore can be marked as transient (not stored to disk) and unchangeable (read-only). To cover transient and unchangeable features, extra conformance constraints are required which restrict the feature value of slots to only non-transient, changeable features.


\subsection{Example}
By binding a model not to the underlying programming languages types defined in its metamodel but to the generic metamodel presented in Figure~\ref{fig:slot_model}, conformance can be checked using the above constraints. Binding the exemplar XMI in Listing~\ref{lst:xmi} to the generic metamodel shown in Figure~\ref{fig:slot_model} produces two class objects, both with type Person. Both class objects contain a slot whose feature is name, one with the value ``Franz'' and the other with the value ``Julie''. The class object containing the slot with value ``Franz'' contains another slot whose feature is mother and whose value is a reference to the other class object (which contains the slot with the value ``Julie''). The generic metamodel used in this thesis implements reference values using the proxy design pattern \cite{gamma95patterns}. 

After binding to the generic metamodel, the conformance of a model can be checked against any metamodel. Suppose the metamodel used to construct the XMI shown in Figure~\ref{fig:slot_model} has now evolved. The mother attribute has now been renamed to parents. Conformance checking would identify the following issues:

\subsection{Applications}
As this section has shown, binding to a metamodel independent syntax is an alternative mechanism for deserialising models that can be used when a model no longer conforms to its metamodel and to check the conformance of a model with any metamodel. The metamodel independent syntax is used throughout this chapter to support several co-evolution

In Section~\ref{sec:notation}, a human-usable modelling notation is generated from the metamodel independent model representation discussed here. In Section~\ref{sec:flock}, partial migration strategies can be specified by producing models that conform to the metamodel independent representation discussed here rather than the latest version of their metamodel.

\subsubsection{Automatic Consistency Checking}
\label{subsec:automatic_checking}
In addition to the applications outlined above, a metamodel independent syntax is particularly useful during metamodel installation. As discussed in Section~\ref{subsec:co-evolution}, metamodel developers do not have access to downstream models. Consequently, instances of a metamodel can become inconsistent after a new version of a metamodel plug-in is installed. By default, an EMF metamodel plug-in does not check consistency during plug-in installation and non-conformant instance models are only detected when the user attempts to load them.

To improve metamodel installation, the binding to a generic metamodel discussed above is integrated with Concordance \cite{kolovos09concordance}, which provides a light-weight and efficient mechanism for resolving inter-model references. Concordance is used to monitor the models in the workspace and, for each model, to maintain a reference to its metamodel.

After plug-in installation, Concordance computes a hash for each available metamodel and compares it to a previously cached value. A hash that differs from its cached value indicates a metamodel evolution, and all instance models are checked for consistency. Consequently, consistency checking happens automatically and during metamodel installation. Models are marked inconsistent immediately rather than when next loaded by EMF. By integrating the consistency checking with Concordance, increased scalability is achieved, as models are only checked when their metamodel has evolved.


\section{A Notation for User-Driven Co-evolution}
\label{sec:notation}
% The Human-Usable Textual Notation is an OMG standard textual concrete syntax for the MOF metamodelling architecture. The notation is metamodel-independent -- it can be used with any model that conforms to any MOF-based metamodel. HUTN provides a human-usable means for visualising and specifying models, even when those models are inconsistent with their metamodel.
The analysis of co-evolution examples in Chapter~\ref{Analysis} highlighted two categories of process for managing co-evolution, developer-driven and user-driven. In the former, migration strategies are executable, while in the latter they are not. Performing user-driven co-evolution with modelling frameworks presents two key challenges that have not been explored by existing research. Firstly, user-driven co-evolution frequently involves editing the storage representation of the model, such as XMI. Model storage representations are typically not optimised for human use and hence user-driven co-evolution can be error-prone. Secondly, non-conformant model elements must be identified during user-driven co-evolution. When a multi-pass parser is used to load models, as is the case with EMF, not all conformance problems are reported at once, and user-driven co-evolution is an iterative process. In Section~\ref{sec:requirements_identification}, these challenges lead to the identification of the following requirement: \emph{This thesis must demonstrate a user-driven co-evolution process that employs a human-usable means for editing non-conformant models and provides a sound and complete conformance report for the original model and evolved metamodel.}

This section introduces an alternative model representation that has been optimised for use by humans and presents an implementation for EMF.


\subsection{Human-Usable Textual Notation}
\label{subsec:hutn}
The OMG's Human-Usable Textual Notation (HUTN) \cite{hutn} defines a generic concrete syntax, which aims to conform to human-usability
criteria \cite{hutn}. However, there is no current reference implementation of HUTN: the Distributed Systems Technology Centre's
TokTok project (an implementation of the HUTN specification) is inactive (and the source code can no longer be found), whilst work on 
implementing the HUTN specification by Muller and Hassenforder \cite{muller05hutn} has been abandoned in favour of Sintaks \cite{sintaks}, which 
operates upon domain-specific concrete syntax.

Because HUTN is a generic concrete syntax, it can be used to represent models that conform to any metamodel. HUTN can be used when a domain-specific concrete syntax is inappropriate or unnecessary. For example, if a metamodel is developed incrementally, use of a generic concrete syntax avoids the need for frequent revision that would apply if a domain-specific concrete syntax were used. Such revisions can significantly detract from the productivity of incremental development, as the domain-specific concrete syntax must be adapted each time the abstract syntax changes.

HUTN defines a generic concrete syntax for constructing instances of
MOF-based metamodels. In this section, we introduce the core syntax
and the key features of HUTN.  The complete definition is available
at \cite{hutn}.  To illustrate usage of the notation, we use the MOF-based
metamodel of families in Figure \ref{fig:example-mm}. (A nuclear family
``consists only of a father, a mother, and children.''
\cite{nucleardef}).

\begin{figure}[htbp]
  \begin{center}
    \leavevmode
    \includegraphics[scale=0.75]{Family.png}
  \end{center}
  \caption{Example metamodel: families and their children.}
  \label{fig:example-mm}
\end{figure}


\subsubsection{Basic Notation}

Listing \ref{lst:attributes} shows the construction of an \emph{object} 
in HUTN, here an instance of the Family class from Figure \ref{fig:example-mm}. Line 1 specifies
the package containing the classes to be constructed (\texttt{FamilyPackage}) and
a corresponding identifier (\texttt{families}), used for fully-qualifying references to objects (Section \ref{InterPackageReferences}).
Line 2 names the class (\texttt{Family}) and gives an identifier for the object (\texttt{The Smiths}).
Lines 3 to 7 define \emph{attribute values}; in each case, the data value is assigned to
the attribute with the specified name. The encoding of the value depends on
its type: strings are delimited by any form of quotation mark;
multi-valued attributes use comma separators, etc.

The metamodel in Figure \ref{fig:example-mm} defines a \emph{simple reference}
(familyFriends) and two \emph{containment references} (adoptedChildren;
naturalChildren). The HUTN representation embeds a contained object
directly in the parent object, as shown in Listing
\ref{lst:containment}. A simple reference can be specified
using the type and identifier of the referred object, as shown in
Listing \ref{lst:non-contained}. Like attribute values, both styles
of reference are preceded by the name of the meta-feature.

\begin{lstlisting}[caption=Specifying attributes with HUTN., label=lst:attributes, language=Families]
FamilyPackage "families" {
    Family "The Smiths" {
        nuclear: true
        name: "The Smiths"
        averageAge: 25.7
        numberOfPets: 2
        address: "120 Main Street", "37 University Road"
    }
}
\end{lstlisting}

\begin{lstlisting}[caption=Instantiation of naturalChildren -- a HUTN containment reference., label=lst:containment, language=Families]
FamilyPackage "families" {
    Family "The Smiths" {
        naturalChildren: Person "John" { name: "John" },
                                Person "Jo" { gender: female }
    }
}
\end{lstlisting}


\begin{lstlisting}[caption=Specifying a simple reference with HUTN., label=lst:non-contained, language=Families]
FamilyPackage "families" {
    Family "The Smiths" {
        familyFriends: Family "The Does"
    }
    Family "The Does" {}
}
\end{lstlisting}


\subsubsection{Keywords and Adjectives}
While HUTN is unlikely to be as concise as a domain-specific 
concrete syntax, the notation does define a number of syntactic shortcuts
in order to make model specifications more compact. These shortcuts 
can be used in place of more verbose (and more readable) full syntax. 
Shortcut use is optional, and their syntax is often intuitive \cite[pg2-4]{hutn}. 
Two example notational shortcuts are described here, in order to 
illustrate some of the ways in which HUTN can be used to construct 
models in a concise manner -- a key concern when using HUTN to construct 
models for use in testing, as we will discuss in Section \ref{sec:motivation}.

When specifying a \emph{Boolean-valued attribute}, it is sufficient
to simply use the attribute name (value \texttt{true}), or the
attribute name prefixed with a tilde (value \texttt{false}). When
used in the body of the object, this style of Boolean-valued
attribute represents a \emph{keyword}. A keyword used to prefix an
object declaration is called an \emph{adjective}. Listing
\ref{lst:boolean} shows the use of both an attribute keyword
(\texttt{\textasciitilde nuclear} on line 6) and adjective
(\texttt{\textasciitilde migrant} on line 2).

\begin{lstlisting}[caption=Using keywords and adjectives in HUTN., label=lst:boolean, language=Families]
FamilyPackage "families" {
    nuclear ~migrant Family "The Smiths" {}

    Family "The Does" {
        averageAge: 20.1
        ~nuclear
        name: "The Does"
    }
}
\end{lstlisting}


\subsubsection{Inter-Package References}
\label{InterPackageReferences}
To conclude our summary of the notation, we present two advanced features 
defined in the HUTN specification, which we have found to be useful when 
constructing large models. The first enables objects to refer to other 
objects in a different package, while the second provides means for 
specifying the values of a reference for all objects in a single 
construct (which can be used to improve the understandability of complex 
relationships in some cases). We conclude by discussing the way in which 
HUTN permits document customisation.

\begin{lstlisting}[caption=Referencing objects in other packages with HUTN., label=lst:fullyqualified, language=Families]
FamilyPackage "families" {
    Family "The Smiths" {}
}
VehiclePackage "vehicles" {
    Vehicle "The Smiths' Car" {
        owner: FamilyPackage.Family "families"."The Smiths"
    }
}
\end{lstlisting}

To reference objects between separate package instances in the same document, 
the package identifier is used in order to construct a fully-qualified name. 
Suppose we introduce a second package to our metamodel in Figure \ref{fig:example-mm}. 
Among other concepts, this package introduces a Vehicle class, which
defines an owner reference of type Family. Listing \ref{lst:fullyqualified} 
illustrates the way in which the owner feature can be populated. Note that the 
fully-qualified form of the class utilises the names of elements of the metamodel, 
while the fully-qualified form of the object utilises only HUTN identifiers defined 
in the current document.

The HUTN specification defines name scope optimisation rules, which allow the 
definition above to be simplified to: \texttt{owner: Family "The Smiths"}, assuming
that (1) the VehiclePackage does not define a Family class, and (2) the identifier 
``The Smiths'' is not used in the VehiclePackage block, or this HUTN document is 
configured to require unique identifiers over the entire document.


\subsubsection{Alternative Reference Syntax}
In addition to the syntax defined in Listings \ref{lst:containment}
and \ref{lst:non-contained}, the value of references may be specified 
independently of the object definitions. For example, Listing \ref{lst:assocblock} 
demonstrates this alternate syntax by defining The Does as friends 
with both The Smiths and The Bloggs.

\begin{lstlisting}[caption=Using a reference block in HUTN., label=lst:assocblock, language=Families]
FamilyPackage "families" {
    Family "The Smiths" {}
    Family "The Does" {}
    Family "The Bloggs" {}
    
    familyFriends {
        "The Does" "The Smiths"
        "The Does" "The Bloggs"
    }
}
\end{lstlisting}

Listing \ref{lst:associnfix} illustrates a further alternative syntax
for references, which employs an infix notation. 

\begin{lstlisting}[caption=Using an infix reference in HUTN., label=lst:associnfix, language=Families]
FamilyPackage "families" {
    Family "The Smiths" {}
    Family "The Does" {}
    Family "The Bloggs" {}
    
    Family "The Smiths" familyFriends Family "The Does"
    Family "The Smiths" familyFriends Family "The Bloggs"
}
\end{lstlisting}


\subsubsection{Customisation via Configuration}
Some limited customisation of HUTN for particular metamodels can be
achieved using \emph{configuration files}. Customisations permitted
include a parametric form of object instantiation (not yet implemented); renaming of
metamodel elements; giving default values for attributes; and
stating an attribute whose values are used to infer a default
identifier. The HUTN specification \cite{hutn} gives details
of shortcuts and of the rules supported by configuration files.


In the next section, we motivate the need for a HUTN implementation
in our development and testing of model management tools, reflecting
on the shortcomings of other approaches, and deriving requirements
for model specification.


\subsection{Example}
- TODO: add a positive example before the counterexample below


Notwithstanding the power of genericity, there are
situations where a domain-specific concrete syntax is preferable. An
example of where HUTN is unhelpful arose when developing a
metamodel for the recording of failure behaviour of components in
complex systems, based on the work of Wallace
\cite{wallace05modular}.

Failure behaviours comprise a number of expressions that specify how
each component reacts to system faults, and there is an established
concrete syntax for expressing failure behaviours.  The failure
syntax allows various shortcuts, such as the use of underscore to
denote a wildcard.  For example, the syntax for a possible failure
behaviour of a component that receives input from two other
components (on the left-hand side of the expression), and produces
output for a single component is denoted:

\begin{eqnarray}\label{failure}
(\{\_\}, \{\_\}) \rightarrow (\{late\})
\end{eqnarray}


A failure behaviour can contain many expressions, and each component
may be connected to many other components, so the metamodel for
 failure behaviours contains a large number of classes.  In the generic concrete syntax,
 the specification of these behaviours is unhelpfully terse.  For example.
 Listing \ref{lst:fptc-hutn} gives the HUTN syntax for failure behaviour
(\ref{failure}), above.

\begin{lstlisting}[caption=Failure behaviour specified in HUTN., label=lst:fptc-hutn, language=FPTC]
Behaviour {
    lhs: Tuple {
        contents: IdentifierSet { contents: Wildcard {} },
                     IdentifierSet { contents: Wildcard {} }
    }

    rhs: Tuple {
        contents: IdentifierSet { contents: Fault "late" {} }
    }
}
\end{lstlisting}

In general, HUTN is less concise than a domain-specific syntax for
 metamodels containing a large number of classes
with few attributes, and in cases where most attributes are used to
define structural relationships among concepts.  However, there
might still be benefits from using HUTN in such cases, if the
metamodel is likely to be modified frequently, of it the model
does not yet have a formal metamodel.





\section{Epsilon Flock}
\label{sec:flock}
Section~\ref{subsec:co-evolution_categorisation} discussed existing approaches to model migration, highlighting variation in the languages used for specifying migration strategies. In this section, migration strategy languages are compared, using the example of metamodel evolution given in Section~\ref{subsec:co-evo_example}. From this comparison, requirements for a domain-specific language for specifying and executing model migration strategies are derived (Section~\ref{subsec:analysis}) and an implementation is described (Section~\ref{subsec:flock_implementation}). This work described in this section was published in \cite{rose10flock}.


\subsection{Co-Evolution Example}
\label{subsec:co-evo_example}
Throughout this section, the following example of an evolution of a Petri net metamodel is used to discuss co-evolution and model migration. The same example has been used previously in co-evolution literature \cite{cicchetti08automating,garces09managing,wachsmuth07metamodel}.

\begin{figure}[bp]
	\centering
	\subfigure[Original metamodel.]
	{
	    \label{fig:original_mm}
	    \includegraphics[width=4.75cm]{5.Implementation/petri_nets_step0.pdf}
	}
	\subfigure[Evolved metamodel.]
	{
	    \label{fig:evolved_mm}
	    \includegraphics[width=6.25cm]{5.Implementation/petri_nets_step1.pdf}
	}
	\caption{Exemplar metamodel evolution. (Shading is irrelevant). Taken from \cite{rose10flock}.}
\label{fig:petri_nets_mms}
\end{figure}

In Figure~\ref{fig:original_mm}, a Petri \texttt{Net} comprises \texttt{Place}s and \texttt{Transition}s. A \texttt{Place} has any number of \texttt{src} or \texttt{dst} \texttt{Transition}s. Similarly, a \texttt{Transition} has at least one \texttt{src} and \texttt{dst} \texttt{Place}. In this example, the metamodel in Figure~\ref{fig:original_mm} is to be evolved so as to support weighted connections between \texttt{Place}s and \texttt{Transition}s and between \texttt{Transition}s and \texttt{Place}s.

The evolved metamodel is shown in Figure~\ref{fig:evolved_mm}. \texttt{Place}s are connected to \texttt{Transition}s via instances of \texttt{PTArc}. Likewise, \texttt{Transition}s are connected to \texttt{Place}s via \texttt{TPArc}. Both \texttt{PTArc} and \texttt{TPArc} inherit from \texttt{Arc}, and therefore can be used to specify a \texttt{weight}.

Models that conformed to the original metamodel might not conform to the evolved metamodel. The following strategy can be used to migrate models from the original to the evolved metamodel:

\begin{enumerate}
	\item For every instance, t, of \texttt{Transition}: 
	\subitem For every \texttt{Place}, s, referenced by the \texttt{src} feature of t: 
	\subsubitem Create a new instance, arc, of \texttt{PTArc}. 
	\subsubitem Set s as the \texttt{src} of arc. 
	\subsubitem Set t as the \texttt{dst} of arc. 
	\subsubitem Add arc to the \texttt{arcs} reference of the \texttt{Net} referenced by t.
	
	\subitem For every \texttt{Place}, d, referenced by the \texttt{dst} feature of t: 
	\subsubitem Create a new instance, arc, of \texttt{TPArc}. 
	\subsubitem Set t as the \texttt{src} of arc. 
	\subsubitem Set d as the \texttt{dst} of arc. 
	\subsubitem Add arc to the \texttt{arcs} reference of the \texttt{Net} referenced by t.
	
	\item And nothing else changes.
\end{enumerate}

Using the above example, the existing approaches for specifying and executing model migration strategies are now compared.


\subsection{Existing Approaches}
Using the above example, the existing approaches for specifying and executing model migration strategies are now compared.

\subsubsection{Manual Specification with Model-to-Model Transformation}
\label{subsubsec:m2m}

A model-to-model transformation specified between original and evolved metamodel can be used for performing model migration. Part of the model migration for the Petri nets metamodel is codified with the Atlas Transformation Language (ATL) \cite{jouault05transforming} in Listing~\ref{lst:atl}. Rules for migrating \texttt{Places} and \texttt{TPArcs} have been omitted for brevity, but are similar to the \texttt{Nets} and \texttt{PTArcs} rules.

In ATL, \emph{rule}s transform source model elements (specified using the \texttt{fr\-om} keyword) to target model elements (specified using \texttt{to} keyword). For example, the \texttt{Nets} rule on line 1 of Listing~\ref{lst:atl} transforms an instance of \texttt{Net} from the original (source) model to an instance of \texttt{Net} in the evolved (target) model. The source model element (the variable \texttt{o} in the \texttt{Net} rule) is used to populate the target model element (the variable \texttt{m}). ATL allows rules to be specified as \emph{lazy} (not scheduled automatically and applied only when called by other rules).

In model transformation, \cite{czarnecki06survey} identifies two common categories of relationship between source and target model, \emph{new target} and \emph{existing target}. In the former, the target model is constructed afresh by the execution of the transformation, while in the latter, the target model contains the same data as the source model before the transformation is executed. ATL supports both new and existing target relationships (the latter is termed a refinement transformation). However, ATL refinement transformations may only be used when the source and target metamodel are the same, as is typical for existing target transformations. 

\begin{lstlisting}[caption=Fragment of the Petri nets model migration in ATL, label=lst:atl, language=ATL]
rule Nets {
	from o : Before!Net
	to m : After!Net ( places <- o.places, transitions <- o.transitions )
}

rule Transitions {
	from o : Before!Transition
	to m : After!Transition (
			name <- o.name,
			"in" <- o.src->collect(p | thisModule.PTArcs(p,o)),
			out  <- o.dst->collect(p | thisModule.TPArcs(o,p))
		)
}

unique lazy rule PTArcs {
	from place : Before!Place, destination : Before!Transition
	to ptarcs : After!PTArc (
			src <- place, dst <- destination, net <- destination.net
		)
}
\end{lstlisting}

In model migration, source and target metamodels differ, and hence existing target transformations cannot be used to specify model migration strategies. Consequently, model migration strategies are specified with new target model-to-model transformation languages, and often contain sections for copying from original to migrated model those model elements that have not been affected by metamodel evolution. For the Petri nets example, the \texttt{Nets} rule (in Listing~\ref{lst:atl}) and the \texttt{Places} rule (not shown) exist only for this reason.

The \texttt{Transitions} rule in Listing~\ref{lst:atl} codifies in ATL the migration strategy described previously. The rule is executed for each \texttt{Transition} in the original model, \texttt{o}, and constructs a \texttt{PTArc} (\texttt{TPArc}) for each reference to a \texttt{Place} in \texttt{o.src} (\texttt{o.dst}). Lazy rules must be used to produce the arcs to prevent circular dependencies with the \texttt{Transitions} and \texttt{Places} rules. Here, ATL, a typical rule-based transformation language, is considered and model migration would be similar in QVT. With Kermeta, migration would be specified in an imperative style using statements for copying \texttt{Net}s, \texttt{Place}s and \texttt{Transition}s, and for creating \texttt{PTArc}s and \texttt{TPArc}s.


\subsubsection{Manual Specification with Ecore2Ecore Mapping}
\label{subsubsec:ecore2ecore}
Hussey and Paternostro \cite{hussey06advanced} explain the way in which integration with the model loading mechanisms of the Eclipse Modeling Framework (EMF) \cite{steinberg09emf} can be used to perform model migration. In this approach, the default metamodel loading strategy is augmented with model migration code.

Because EMF binds models to their metamodel (discussed in Section~\ref{subsec:modelling_framework_characteristics}), EMF cannot use an evolved metamodel to load an instance of the original metamodel. Therefore, Hussey and Paternostro's approach requires the metamodel developer to provide a mapping between the metamodelling language of EMF, Ecore, and the concrete syntax used to persist models, XMI. Mappings are specified using a tool that can suggest relationships between source and target metamodel elements by comparing names and types.

Model migration is specified on the XMI representation of the model and hence presumes some knowledge of the XMI standard. For example, in XMI, references to other model elements are serialised as a space delimited collection of URI fragments \cite{steinberg09emf}. Listing~\ref{lst:java} shows a section of the Ecore2Ecore model migration for the Petri net example presented above. The method shown converts a \texttt{String} containing URI fragments to a \texttt{Collection} of \texttt{Place}s. The method is used to access the \texttt{src} and \texttt{dst} features of \texttt{Transition}, which no longer exist in the evolved metamodel and hence are not loaded automatically by EMF. To specify the migration strategy for the Petri nets example, the metamodel developer must know the way in which the \texttt{src} and \texttt{dst} features are represented in XMI. The complete listing, not shown here, exceeds 200 lines of code.

\begin{lstlisting}[basicstyle=\ttfamily\footnotesize, flexiblecolumns=true, numbers=left, nolol=true, caption=Java method for deserialising a reference., label=lst:java, language=Java, tabsize=2]
private Collection<Place> toCollectionOfPlaces
(String value, Resource resource) {

  final String[] uriFragments    = value.split(" ");
  final Collection<Place> places = new LinkedList<Place>();
 
  for (String uriFragment : uriFragments) {
		final EObject eObject = resource.getEObject(uriFragment);
		final EClass place    = PetriNetsPackage.eINSTANCE.getPlace();

    if (eObject == null || !place.isInstance(eObject))
      // throw an exception
						
		places.add((Place)eObject);
  }
 
  return places;
}
\end{lstlisting}

\subsubsection{Operator-based Co-evolution with COPE}
\label{subsubsec:cope}

Operator-based approaches to managing co-evolution, such as COPE \cite{herrmannsdoerfer09cope}, provide a library of \emph{co-evolutionary operators}. Each co-evolutionary operator specifies both a metamodel evolution and a corresponding model migration strategy. For example, the ``Make Reference Containment'' operator from COPE \cite{herrmannsdoerfer09cope} evolves the metamodel such that a non-containment reference becomes a containment reference and migrates models such that the values of the evolved reference are replaced by copies. By composing co-evolutionary operators, metamodel evolution can be performed and a migration strategy can be generated without writing any code.

To perform metamodel evolution using an operator-based approach, the library of co-evolutionary operators must be integrated with tools for editing metamodels. COPE provides integration with the EMF tree-based metamodel editor. Operators may be applied to an EMF metamodel, and a record of changes tracks their application. Once metamodel evolution is complete, a migration strategy can be generated automatically from the record of changes. The migration strategy is distributed along with the updated metamodel, and metamodel users choose when to execute the migration strategy on their models.

To be effective, operator-based approaches must provide a rich yet navigable library of co-evolutionary operators, as discussed in Section~\ref{subsec:co-evolution_categorisation}. To this end, COPE allows model migration strategies to be specified manually when no co-evolutionary operator is appropriate. Rather than use either of the two manual specification approaches discussed above (model-to-model transformation and Ecore2Ecore mapping), COPE employs a fundamentally different approach using an existing target transformation.

As discussed above, existing target transformations cannot be used for specifying model migration strategies as the source (original) and target (evolved) metamodels differ. However, models can be structured independently of their metamodel using a \emph{metamodel-independent representation}. Figure~\ref{fig:cope_mmi} shows a simplification of the metamodel-independent representation used by COPE. By using a metamodel-independent representation of models as an intermediary, an existing target transformation can be used for performing model migration when the migration strategy is specified in terms of the metamodel-independent representation. Further details of this technique are given in \cite{herrmannsdoerfer09cope}.

\begin{figure}[tbp]
  \centering
  \includegraphics[scale=0.75]{5.Implementation/cope_mm.pdf}
  \caption{Simplification of the metamodel-independent representation used by COPE, based on \cite{herrmannsdoerfer09cope}.}
  \label{fig:cope_mmi}
\end{figure}

Listing~\ref{lst:cope} shows the COPE model migration strategy for the Petri net example given above\footnote{In Listing~\ref{lst:cope}, some of the concrete syntax has been changed in the interest of brevity.}. Most notably, slots for features that no longer exist must be explicitly \texttt{unset}. In Listing~\ref{lst:cope}, slots are \texttt{unset} on four occasions, once for each feature that exists in the original metamodel but not the evolved metamodel. Namely, these features are: \texttt{src} and \texttt{dst} of \texttt{Transition} and of \texttt{Place}. Failing to \texttt{unset} slots that do not conform with the evolved metamodel causes migration to fail with an error.

\begin{lstlisting}[caption=Petri nets model migration in COPE, label=lst:cope, language=COPE]
for (transition in Transition.allInstances) {
  for (source in transition.unset('src')) {
    def arc = petrinets.PTArc.newInstance()
    arc.src = source;  arc.dst = transition;
    arc.net = transition.net
  }

  for (destination in transition.unset('dst')) {
    def arc = petrinets.TPArc.newInstance() 
    arc.src = transition; arc.dst = destination;
    arc.net = transition.net
  }
}

for (place in Place.allInstances) {
  place.unset('src');  place.unset('dst');
}
\end{lstlisting}


\subsection{Analysis}
\label{subsec:analysis}
By analysing existing approaches to managing developer-driven co-evolution, requirements were derived for Epsilon Flock, a domain-specific language for specifying and executing model migration. The derivation of the requirements for Epsilon Flock is now summarised, by considering two dimensions: the source-target relationship of the language used for specifying migration strategies and the way in which models are represented during migration. %and the structures provided by the language for specifying and re-using migration strategies.


\subsubsection{Source-Target Relationship}
New target transformation languages (Section \ref{subsubsec:m2m}) require code for explicitly copying from the original to the evolved metamodel those model elements that are unaffected by the metamodel evolution. In contrast, model migration strategies written in COPE (Section~\ref{subsubsec:cope}) must explicitly unset any data that is not to be copied from the original to the migrated model. The Ecore2Ecore approach (Section~\ref{subsubsec:ecore2ecore}) does not require explicit copying or unsetting code. Instead, the relationship between original and evolved metamodel elements is captured in a mapping model specified by the metamodel developer. The mapping model can be configured by hand or, in some cases, automatically derived. 

In each case, extra effort is required when defining a migration strategy due to the way in which the co-evolution approach relates source (original) and target (migrated) model elements. This observation led to the following requirement: \emph{Epsilon Flock must \textbf{automatically} copy every model element that conforms to the evolved metamodel from original to migrated model, and must not automatically copy any model element that does not conform to the evolved metamodel from original to migrated model.}


\subsubsection{Model Representation}
When using the Ecore2Ecore approach, model elements that do not conform to the evolved metamodel are accessed via XMI. Consequently, the metamodel developer must be familiar with XMI and must perform tasks such as dereferencing URI fragments (Listing~\ref{lst:java}) and type conversion. With COPE and the Epsilon Transformation Language, models are loaded using a modelling framework (and so migration strategies need not be concerned with the representation used to store models). Consequently, the following requirement was identified: \emph{Epsilon Flock must not expose the underlying representation of original or migrated models.}

To apply co-evolution operators, COPE requires the metamodel developer to use a specialised metamodel editor, which can manipulate only metamodels defined with EMF. Like, the Ecore2Ecore approach, COPE can be used only to manage co-evolution for models and metamodels specified with EMF. Tight coupling to EMF allows the Ecore2Ecore approach to schedule migration automatically, during model loading. To better support integration with modelling frameworks other than EMF, the following requirement was derived: \emph{Epsilon Flock must be loosely coupled with modelling frameworks and must not assume that models and metamodels will be represented in EMF.}


% \subsubsection{Re-use of migration strategies}
% To produce a suitable migration strategy, each approach requires some effort on the part of the metamodel developer during metamodel evolution. As we discuss more thoroughly in \cite{rose09analysis}, there is a trade-off between the amount of effort required and the flexibility of the approach. 
% 
% COPE seeks to reduce the effort required to express a migration strategy with co-evolutionary operators, which specify re-usable fragments of a migration strategy. With COPE, co-evolutionary operators are applied to perform metamodel evolution and later used to automatically generate a corresponding migration strategy. To apply co-evolution operators, COPE requires the metamodel developer to use a specialised metamodel editor and therefore it is not clear whether operator-based co-evolution can be used with all categories of metamodel editing tool, as we discuss more thoroughly in \cite{rose09analysis}.
% 
% The Ecore2Ecore approach and Epsilon Transformation Language do not require metamodel evolution to occur in a specialised editor and do not provide structures for specifying or re-using migration strategy fragments.
% 
% No existing approach investigates whether it is possible to provide re-usable migration strategy fragments without requiring a specialised metamodel editor. To explore this, the following requirements were derived:  \emph{Epsilon Flock must provide re-usable structures for expressing commonly occurring fragments of migration strategies. Application of the re-usable structures must not require a specialised metamodel editor.}


\subsection{Implementation}
\label{subsec:flock_implementation}
Driven by the analysis presented above, Epsilon Flock (subsequently referred to as Flock) was designed and implemented. Flock is a domain-specific language for specifying and executing model migration strategies. Flock uses a model connectivity framework, which decouples migration from the representation of models and provides compatibility with several modelling frameworks (Section~\ref{subsubsec:epsilon}). Flock automatically maps each element of the original model to an equivalent element of the migrated model using a novel conservative copying algorithm and user-defined migration rules (Section~\ref{subsubsec:conservative_copying}).


\subsection{The Epsilon Platform}
\label{subsec:epsilon}
Before presenting Flock, it is necessary to revisit some details of the Epsilon \cite{kolovos09thesis} platform, which was introduced in Section~\ref{subsubsec:epsilon}\footnote{TODO: reference lit review}. Epsilon, a component of the Eclipse GMT project \cite{gmt}, provides infrastructure for implementing uniform and interoperable model management languages, for performing tasks such as model merging, model transformation and inter-model consistency checking. 

The core of the platform is the Epsilon Object Language (EOL) \cite{kolovos06eol}, a reworking and extension of OCL that includes the ability to update models, conditional and loop statements, statement sequencing, and access to standard I/O streams. EOL provides mechanisms for reusing sections of code, such as user-defined operators along with modules and import statements. The Epsilon task-specific languages are built atop EOL, giving highly efficient inheritance and reuse of features.

\subsection{Flock}
Flock is a rule-based transformation language that mixes declarative and imperative parts. Its style is inspired by hybrid model-to-model transformation languages such as the Atlas Transformation Language \cite{jouault05transforming} and the Epsilon Transformation Language \cite{kolovos08etl}. Flock has a compact syntax. Much of its design and implementation is focused on the runtime. The way in which Flock relates source to target elements is novel; it is neither a new nor an existing target relationship. 

\subsubsection{Abstract Syntax}
\label{subsubsec:abstract_syntax}
As illustrated by Figure~\ref{fig:abstract_syntax}, Flock migration strategies are organised into modules (\texttt{Fl\-ockMo\-du\-le}), which inherit from EOL modules (\texttt{Eo\-lMod\-ule}), which provides support for module reuse with import statements and user-defined operations. Modules comprise any number of rules (\texttt{Ru\-le}). Each rule has an original metamodel type (\texttt{or\-ig\-in\-alTy\-pe}) and can optionally specify a \texttt{gu\-ard}, which is either an EOL statement or a block of EOL statements. \texttt{Mi\-gr\-ateRu\-le}s must specify an evolved metamodel type (\texttt{ev\-ol\-vedTy\-pe}) and/or a \texttt{bo\-dy} comprising a block of EOL statements.

\begin{figure}
  \centering
  \includegraphics[scale=0.75]{5.Implementation/flock_abstract_syntax.pdf}
  \caption{The abstract syntax of Flock.}
  \label{fig:abstract_syntax}
\end{figure}

\subsubsection{Concrete Syntax}
\label{subsubsec:concrete_syntax}

Listing~\ref{lst:flock_concrete_syntax} shows the concrete syntax of migrate and delete rules. All rules begin with a keyword indicating their type (either \texttt{migrate} or \texttt{delete}), followed by the original metamodel type. Guards are specified using the \texttt{when} keywords. Migrate rules may also specify an evolved metamodel type using the \texttt{to} keyword and a \texttt{body} as a (possibly empty) sequence of EOL statements.

Note there is presently no create rule. In Flock, the creation of new model elements is usually encoded in the imperative part of a migrate rule specified on the containing type.

\begin{lstlisting}[float=tbp, caption=Concrete syntax of migrate and delete rules., label=lst:flock_concrete_syntax, language=Flock]
migrate <originalType> (to <evolvedType>)?
(when (:<eolExpression>)|({<eolStatement>+}))? {
	<eolStatement>*
} 

delete <originalType>
(when (:<eolExpression>)|({<eolStatement>+}))?
\end{lstlisting}

\subsubsection{Execution Semantics}
A Flock module has the following behaviour when executed:

\begin{enumerate}
	\item For each original model element, \texttt{e}:
	\subitem Identify an applicable rule, \texttt{r}. To be applicable for \texttt{e}, a rule must have as its original type the metaclass (or a supertype of the metaclass) of \texttt{e} and the guard part of the rule must be satisfied by \texttt{e}.
	\subitem When no rule can be applied, a default rule is used, which has the metaclass of \texttt{e} as its original type, and an empty body.
	
	\item For each mapping between original model element, \texttt{e}, and applicable delete rule, \texttt{r}:
	\subitem Do nothing.
	
	\item For each mapping between original model element, \texttt{e}, and applicable migrate rule, \texttt{r}:
	\subitem Create an equivalent model element, \texttt{e'} in the migrated model. The metaclass of \texttt{e'} is determined from the \texttt{evolvedType} (or the \texttt{originalType} when no \texttt{evolvedType} has been specified) of \texttt{r}.
	\subitem Copy the data contained in \texttt{e} to \texttt{e'} (using the \emph{conservative copy} algorithm described in the sequel).

	\item For each mapping between original model element, \texttt{e}, applicable migrate rule, \texttt{r}, and equivalent model element, \texttt{e'}:
	\subitem Execute the body of \texttt{r} binding \texttt{e} and \texttt{e'} to variables named \texttt{original} and \texttt{migrated}, respectively.
\end{enumerate}


\subsubsection{Conservative Copying}
\label{subsubsec:conservative_copying}
Flock contributes an algorithm, termed \emph{conservative copy}, that copies model elements from original to migrated model only when those model elements conform to the evolved metamodel. Because of its conservative copy algorithm, Flock is a hybrid of new target and existing target transformation languages. This section discusses the conservative copying algorithm in more detail.

The algorithm operates on an original model element, \texttt{o}, and its equivalent model element in the migrated model, \texttt{e}. When \texttt{o} has no equivalent in the migrated model (for example, when a metaclass has been removed and the migration strategy specifies no alternative metaclass), \texttt{o} is not copied to the migrated model. Otherwise, conservative copy is invoked for \texttt{o} and \texttt{e}, proceeding as follows:

\begin{itemize}
	\item For each metafeature, \texttt{f} for which \texttt{o} has specified a value
		\subitem Locate a metafeature in the evolved metamodel with the same name as \texttt{f} for which \texttt{e} may specify a value.
			\subsubitem When no equivalent metafeature can be found, do nothing.
			\subsubitem Otherwise, copy to the migrated model the original value (\texttt{o.f}) only when it conforms to the equivalent metafeature
\end{itemize}

The definition of conformance varies over modelling frameworks. Typically, conformance between a value, \texttt{v}, and a feature, \texttt{f}, specifies at least the following constraints:

\begin{itemize}
	\item The size of \texttt{v} must be greater than or equal to the lowerbound of \texttt{f}.
	\item The size of \texttt{v} must be less than or equal to the upperbound of \texttt{f}.
	\item The type of \texttt{v} must be the same as or a subtype of the type of \texttt{f}.
\end{itemize}


Epsilon includes a model connectivity layer (EMC), which provides a common interface for accessing and persisting models. Currently, EMC provides drivers for several modelling frameworks, permitting management of models defined with EMF, the Metadata Repository (MDR), Z or XML. To support migration between metamodels defined in heterogenous modelling frameworks, EMC was extended during the development of Flock. The connectivity layer now provides a conformance checking service. Each EMC driver was extended to include conformance checking semantics specific to its modelling framework. Flock implements conservative copy by delegate conformance checking responsibilities to EMC. 

Finally, some categories of model value must be converted before being copied from the original to the migrated model. Again, the need for and semantics of this conversion varies over modelling frameworks. Reference values typically require conversion before copying. In this case, the mappings between original and migrated model elements maintained by the Flock runtime can be used to perform the conversion. In other cases, the target modelling framework must be used to perform the conversion, such as when EMF enumeration literals are to be copied.


\subsubsection{Development and User Tools}
As discussed in Section~\ref{sec:analysing_existing_techniques}, models and metamodels are typically kept separate. Flock migration strategies can be distributed by the metamodel developer in two ways. An extension point defined by Flock provides a generic user interface for migration strategy execution. Alternatively, metamodel developers can elect to build their own interface, delegating execution responsibility to \texttt{FlockModule}. We anticipate the latter to be useful for production environments using model or source code management repositories.


\subsection{Example}
The exemplar Petri net metamodel evolution is now revisited to demonstrate the basic functionality of Flock. In Listing~\ref{lst:flock}, \texttt{Net}s and \texttt{Place}s are migrated automatically. Unlike the ATL migration strategy (Listing~\ref{lst:atl}), no explicit copying rules are required. Compared to the COPE migration strategy (Listing~\ref{lst:cope}), the Flock migration strategy does not explicitly unset the original \texttt{src} and \texttt{dst} features of \texttt{Transition}.

\begin{lstlisting}[caption=Petri nets model migration in Flock, label=lst:flock, language=Flock]
migrate Transition {
  for (source in original.src) {
    var arc := new Migrated!PTArc;
    arc.src := source.equivalent();  arc.dst := migrated;
    arc.net := original.net.equivalent();
  }

  for (destination in original.dst) {
    var arc := new Migrated!TPArc;
    arc.src := migrated;  arc.dst := destination.equivalent();
    arc.net := original.net.equivalent();
  }
}
\end{lstlisting}

Table~\ref{tab:differences} illustrates several characterising differences between Flock and the related approaches presented in Section~\ref{subsec:co-evo_example}. Due to its conservative copying algorithm, Flock is the only approach to provide both automatic copying and unsetting. Automatic copying is significant for metamodel evolutions with a large number of unchanging features.

All of the approaches considered in Table~\ref{tab:differences} support EMF, arguably the most widely used modelling framework. The Ecore2Ecore approach, however, requires migration to be encoded at the level of the underlying model representation XMI. Both Flock and ATL support other modelling technologies, such as MDR and XML. However, ATL does not automatically copy model elements that have not been affected by metamodel changes. Therefore, migration between models of different technologies with ATL requires extra statements in the migration strategy to ensure that the conformance constraints of the target technology are satisfied. Because it delegates conformance checking to an EMC driver, Flock requires no such checks.

\begin{table}[b]
	\caption{Properties of model migration approaches}
	\centering
	\begin{tabular}{|c|c|c|c|}
		\hline
		             & \textbf{Automatic copy} & \textbf{Automatic unset} & \textbf{Modelling technologies} \\
		\hline
		\textbf{Ecore2Ecore}  & \tick             & \cross              & XMI                    \\
		\hline
		\textbf{ATL}          & \cross            & \tick               & EMF, MDR, KM3, XML     \\
		\hline
		\textbf{COPE}         & \tick             & \cross              & EMF                    \\
		\hline
		\textbf{Flock}        & \tick             & \tick               & EMF, MDR, XML, Z       \\
		\hline
	\end{tabular}
	\label{tab:differences}
\end{table}

A more thorough examination of the similarities and differences between Flock and other migration strategy languages is provided in Chapter~\ref{Evaluation}.
