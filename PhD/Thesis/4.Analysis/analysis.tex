%!TEX root = /Users/louis/Documents/PhD/Deliverables/Thesis/thesis.tex

\chapter{Analysis}
%The literature review will motivate a deeper analysis of existing techniques for managing evolution in the context of MDE. The benefits and drawbacks of existing techniques will be highlighted by applying them to data from projects using MDE. The analysis will be used to identify requirements for our research.


\section{Locating Data}
% The first section of the analysis chapter will be based on a section of the progress report, which discusses the data (existing MDE projects) used to analysis existing techniques for managing evolution in the context of MDE. We will introduce and explain the requirements on the data to be used for analysis, identify candidate MDE projects, describe the selection process, and provide reasons for our choices.

To provide requirements for developing structures and processes for evolutionary changes in the context of model-driven engineering, the ways in which MDE development artefacts evolve over time were categorised. A study of example data (existing MDE projects containing evolution) was used to produce this categorisation. Requirements were defined for example MDE projects for the study. Existing MDE projects were located and analysed against the requirements. Finally, the most suitable candidate projects were selected.

\subsection{Requirements}
In Chapter~\ref{Background}, two categories of evolutionary change were identified: co-evolution and synchronisation. In this section, requirements are partitioned into three types: those necessary for studying each of the two categories of evolutionary change, and common requirements (applicable to both categories of evolutionary change).

\subsubsection{Common requirements}
Every candidate project needs to use MDE. Specifically, both metamodelling and model transformation must be used (requirement R1). In addition, each candidate project needs to provide historical information to trace the evolution of development artefacts (R2). For example, several versions of the project are needed perhaps in a source code management system. Finally, a candidate project needs to have undergone a number of significant changes\footnote{This is deliberately vague. Further details are given in Section~\ref{subsec:project_selection}.} (R3).

\subsubsection{Co-evolution requirements}
A candidate project for the study of co-evolution needs to define a metamodel and some changes to that metamodel (R4). In the projects considered, the metamodel changes took the form of either another version of the metamodel, or a history (which recorded each of the steps used to produce the adapted metamodel). A candidate project also needs to provide example instances of models before and after each migration activity (R5).

Ideally, a candidate project should include more than one consecutive metamodel adaptation, so as to represent the way in which the same development artefacts continue to evolve over time (optional requirement O1).

\subsubsection{Synchronisation requirements}
A candidate project for the study of synchronisation needs to define a model-to-model transformation (R6). Furthermore, a candidate project has to include many examples of source and target models for that transformation (R7). Crucially, a candidate project needs to provide many examples of the kinds of change (to either source or target model) that cause inconsistency between the models (R8). 

Ideally, a candidate project should also include transformation chains (more than one model-to-model transformation, executed sequentially) (O2). Chains of transformations are prescribed by the MDA guidelines \cite{kleppe03mda}.


\subsection{Project Selection}
\label{subsec:project_selection}
Eight candidates were considered for the study. Table \ref{tab:candidates} shows which of the requirements are fulfilled by each of the candidates. Each candidate is now discussed in turn.

\begin{table}
	\caption{Candidates for study of evolution in existing MDE projects}
	\centering
	\begin{tabular}{|c||c|c|c||c|c|c||c|c|c|c|}
		\hline
		\multirow{3}{*}{Name} & \multicolumn{10}{|c|}{Requirements} \\
		\cline{2-11}
		          & \multicolumn{3}{|c||}{Common} & \multicolumn{3}{|c||}{Co-evolution} & \multicolumn{4}{|c|}{Synchronisation} \\
		\cline{2-11}
		          & R1 & R2 & R3 & R4 & R5 & O1 & R6 & R7 & R8 & O2 \\
		\hline
		GSN       & x  &    &    & x  &    &    &    &    &    &    \\
		\hline
		OMG       & x  &    &    & x  &    &    & x  &    &    &    \\
		\hline
		Zoos      & x  & x  &    & x  &    &    &    &    &    &    \\
		\hline
		MDT       & x  & x  &    & x  &    & x  &    &    &    &    \\
		\hline
		MODELPLEX & x  & x  & x  & x  &    & x  & x  & x  &    &    \\
		\hline
		FPTC      & x  & x  & x  & x  & x  &    &    &    &    &    \\
		\hline
		xText     & x  & x  & x  & x  & x  & x  & x  & x  &    & x  \\
		\hline
		GMF       & x  & x  & x  & x  & x  & x  & x  & x  &    & x  \\
		\hline
	\end{tabular}
	\label{tab:candidates}
\end{table}

\subsubsection{GSN}
\label{par:gsn}
Georgios Despotou and Tim Kelly, members of this department's High Integrity Systems Engineering group, are constructing a metamodel for Goal Structuring Notation (GSN). The metamodel has been developed incrementally. There is no accurate and detailed version history for the GSN metamodel (requirement R2). \textbf{Suitability for study:} Unsuitable.

\subsubsection{OMG}
\label{par:omg}
The Object Management Group (OMG) \cite{omg} oversees the development of model-driven technologies. The Vice President and Technical Director of OMG, Andrew Watson, references the development of two MDE projects in \cite{watson08mdahistory}. Correspondence with Watson ascertained that source code is available for one of the projects, but there is no version history. \textbf{Suitability for study:} Unsuitable.

\subsubsection{Zoos}
\label{par:zoos}
A zoo is a collection of metamodels, authored in a common metamodelling language. I considered two zoos, but neither contained any significant external metamodel changes. Those changes that were made involved only renaming of meta-classes (trivial to migrate) or additive changes (which do not affect consistency, and therefore require no migration). \textbf{Suitability for study:} Unsuitable.

\subsubsection{MDT}
The Eclipse Model Development Tools (MDT) \cite{mdt} provides implementations of industry-standard metamodels, such as UML2 \cite{uml212} and OCL \cite{ocl2}. Like the metamodel zoos, the version history for the MDT metamodels contained no significant changes. \textbf{Suitability for study:} Unsuitable.

\subsubsection{MODELPLEX}
Jendrik Johannes, a research assistant at TU Dresden, has made available work from the European project, MODELPLEX. Johannes's work involves transforming UML models to Tool Independent Performance Models (TIPM) for simulation. Although the TIPM metamodel and the UML-to-TIPM transformation have been changed significantly, no significant changes have been made to the models. \textbf{Suitability for study:} Unsuitable.

\subsubsection{FPTC}
Failure Propagation and Transformation Calculus (FPTC), developed by Malcolm Wallace in this department, provides a means for reasoning about the failure behaviour of complex systems. Before starting my doctorate, I worked with Richard Paige to develop an implementation of FPTC in Eclipse. The implementation includes an FPTC metamodel. Recent work with Philippa Conmy, a research assistant in the department, has identified a significant flaw in the implementation, leading to changes to the metamodel. These changes caused existing FPTC models to become inconsistent with the metamodel. Conmy has made available copies of FPTC models from before and after the changes. \textbf{Suitability for study:} Suitable for studying co-evolution. Unsuitable for studying synchronisation.

\subsubsection{xText}
xText is an openArchitectureWare (oAW) \cite{oaw} tool for generating parsers, metamodels and editors for performing text-to-model transformation. Internally, xText defines a metamodel, which has been changed significantly over the last two years. In several cases, changes have caused inconsistency with existing models. xText provides examples of use, which have been updated alongside the metamodel. \textbf{Suitability for study:} Suitable for studying co-evolution. Unsuitable for studying synchronisation.

\subsubsection{GMF}
The Graphical Modelling Framework (GMF) \cite{gronback06gmf} allows the definition of graphical concrete syntax for metamodels that have been defined in EMF. GMF prescribes a model-driven approach: Users of GMF define concrete syntax as a model, which is used to generate a graphical editor. In fact, five models are used together to define a single editor using GMF.

GMF defines the metamodels for graphical, tooling and mapping definition models; and for generator models. The metamodels have changed considerably during the development of GMF. Some changes have caused inconsistency with GMF models. Presently, migration is encoded in Java. Gronback has stated\footnote{Private communication, 2008.} that the migration code is being ported to QVT (a model-to-model transformation language) as the Java code is difficult to maintain.

GMF fulfils almost all of the requirements for the study. A large amount of the co-evolution data is available, including migration strategies. The GMF source code repository does not contain examples of the kinds of change that cause inconsistency between the models. However, GMF has a large number of users, and it may be possible to gather this information elsewhere. \textbf{Suitability for study:} Suitable for studying both categories of evolutionary change.

\subsubsection{Summary of selection}
The FPTC and xText projects were selected for a study of co-evolution. No appropriate projects were located for a study of synchronisation. The GMF project will not be studied immediately, but reserved as a case study for evaluating my research.



\subsection{Other examples}
As only a small number of candidate projects fulfilled all of the requirements, additional data was collected from alternative sources. Firstly, examples were sought from related domains (e.g. object-oriented systems). Secondly, examples were discovered during collaboration with colleagues on two projects, both of which are using iterative and incremental MDE.

\subsubsection{Examples of evolution from object-oriented systems}
In object-oriented programming, software is constructed by developing groups of related objects. Every object is an instance of (at least) one class. A class is a description of characteristics, which are shared by each of the class's instances (objects). A similar relationship exists between models and metamodels: metamodels comprises meta-classes, which describe the characteristics shared by each of the meta-class's instances (elements of a model). Together, model elements are used to describe one perspective (model) of a system. Therefore, studying the evolution of object-oriented systems may yield results that are relevant to evolution occurring in MDE. This section explores that argument.

\paragraph{Preliminary study of object-oriented refactorings}
\emph{Refactoring} is the process of improving the structure of existing code while maintaining its external behaviour. When used as a noun, a refactoring is one such improvement. \cite{fowler99refactoring} provides a catalogue of refactorings for object-oriented systems. For each refactoring, Fowler gives advice and instructions for its application.

Fowler's refactorings provide examples of evolutionary changes to object-oriented systems. Application of each refactoring described in \cite{fowler99refactoring} to EMF metamodels highlighted that some are irrelevant to metamodelling. The refactorings that do not apply to EMF metamodels belong to one of three categories:

\begin{enumerate}
	\item \textbf{Operational refactorings} focus on restructuring behaviour (method bodies). EMF does not support the specification of behaviour in models.
	\item \textbf{Navigational refactorings} convert, for example, between bi-directional and uni-directional associations. These changes are non-breaking in EMF, which automatically provides values for the inverse of a reference when required.
	\item \textbf{Domain-specific refactorings} manage issues specific to object-oriented programming, such as casting, defensive return values, and assertions. These issues are not relevant to metamodelling.
\end{enumerate}

The object-oriented refactorings that can be applied to metamodels provide examples of metamodel evolution. When applied, some of these refactorings potentially cause inconsistency between a metamodel and its models. By using Fowler's description of each refactoring, a migration strategy for updating (co-evolving) inconsistent models was deduced. An example of deducing a migration strategy is now presented.

Figure \ref{fig:refactoring} illustrates a refactoring that changes a reference object to a value object. Value objects are immutable, and cannot be shared (i.e. any two objects cannot refer to the same value object). Reference objects are mutable, and can be shared. Figure \ref{fig:refactoring} indicates that applying the refactoring restricts the multiplicity of the association (on the Order end) to 1 (implied by the composition); prior to the refactoring the multiplicity is many

\begin{figure}[htbp]
  \begin{center}
    \leavevmode
    \includegraphics[scale=0.5]{"4 - Analysis/exemplar_refactoring.png"}
  \end{center}
  \caption{Refactoring a reference to a value. Taken from \cite{fowler99refactoring}[pg183].}
  \label{fig:refactoring}
\end{figure}

Before applying the refactoring, each customer may be associated with more than one order. After the refactoring, each customer should be associated with only one order. Fowler indicates that every customer associated with more than one order should be duplicated, such that one customer object exists for each order. Therefore, the migration strategy in Listing \ref{lst:refactoring} is deduced.

Using this process, migration strategies were deduced for each of the refactorings that were applicable to EMF metamodels, and caused inconsistencies between a metamodel and its models. Interestingly, some of the metamodel changes deduced from Fowler's refactorings were equivalent to some of the changes observed in the metamodel changes made in existing MDE projects.

\begin{lstlisting}[caption=Migration strategy for the refactoring in pseudo code., label=lst:refactoring]
for every customer, c
  for every order, o, associated with c
    create a new customer, d
    copy the values of c's attributes into d
  next o
	
  delete c
next c
\end{lstlisting}

%Object-oriented refactorings are used to improve the maintainability of existing systems. In other words, they represent only one of the three reasons for evolutionary change defined by \cite{sjoberg93quantifying}. The two other types of changes are equally relevant to this thesis. Consequently...


\subsubsection{Research collaborations} % (fold)
\label{par:collaborations}
As well as the example data located from object-oriented system, collaboration with two colleagues using MDE provided several examples of evolution. A prototypical metamodel to standardise the way in which process-oriented programs are model was produced with Adam Sampson, a research assistant at the University of Kent, and an investigation of the feasibility of implementing a tool for generating story-worlds for interactive narratives was conducted with Heather Barber, a postdoctorate student in the department.

In both cases, a metamodel was constructed for describing concepts in the domain. The metamodels were developed incrementally and changed over time. The collaborations with Sampson and Barber did not involve constructing model-to-model transformations, but did provide data suitable for a study of co-evolution.

\subsection{Summary}
To summarise, eight existing MDE projects were located, three of which satisfied the requirements for a study of co-evolutionary changes in the context of model-driven engineering. One of the three projects, GMF, was reserved as a case study and is used for evaluation in Chapter~\ref{Evaluation}. Refactorings of object-oriented programming supplemented the data available from the existing MDE projects. Collaboration with Sampson and Barber yielded further examples of co-evolution. No appropriate examples of model synchronisation were located, and consequently this thesis focuses on model and metamodel co-evolution.





\section{Analysing Existing Techniques}
% Having described the selection of suitable data for the analysis, we will then outline the way in which we have applied existing techniques to the data, and introduce criteria against which the effectiveness of existing techniques will be measured. 


\section{Requirements Identification}
% The analysis of existing techniques will lead to requirements for our research. We will conclude the chapter by enumerating these requirements, refining the high-level research objectives from the literature review chapter into lower-level research objectives.

