%!TEX root = /Users/louis/Documents/PhD/Deliverables/Thesis/thesis.tex

\section{Future Work}
\label{sec:future_work}
In the context of a doctoral thesis, it is impossible to thoroughly investigate many of the issues raised in exploring the thesis hypothesis. Below, several potential extensions to the research presented in this thesis are identified, and any initial work in those areas is described. 

\subsection{Further Evaluation}
The extent to which the structures and processes introduced in Chapter~\ref{Implementation} increase developer productivity for co-evolution has been explored via expert evaluation and comparison to related work. Assessing the way in which the proposed structures and processes affect productivity is challenging due to the number of factors that affect productivity. In practice, for example, the proposed structures and processes would be used by developers with different expertise, and together with other tools and techniques. Evaluating the way in which software evolution is identified and managed in practice using comprehensive case and user studies would be likely to allow stronger claims to be made about the efficacy of the proposed structures and processes. Given the time constraints of a doctoral thesis, comprehensive case and user studies were not feasible, and hence are desirable extensions to the thesis research. A key first step would be to establish a common vocabulary for discussing software evolution activities in the context of MDE, which would facilitate the comparison of user experiences.

\subsection{Extensions to the Model Migration Language}
The model migration language proposed and implemented in Chapter~\ref{Implementation}, Epsilon Flock, makes idiomatic commonly occurring patterns of model migration. The evaluation presented in Chapter~\ref{Evaluation} suggested that migration strategies are often more concise when specified with Flock rather than when specified with contemporary model-to-model transformation languages. The evaluation also highlighted a limitation in the implementation of Flock and demonstrated further patterns that might be captured by model migration languages.

Addressing these issues would further improve the conciseness and re-usability of model migration strategies written in Flock and, hence, is an obvious area of future work. In particular, one language construct controls two concerns in the current implementation of Flock, and introducing separate language constructs for each concern could increase the potential for re-use between model migration rules. Applying Flock to the co-evolution examples used for evaluation highlighted further model migration patterns that might be made idiomatic in the language. For example, in situations where conservative copy can have side-effects, it may be desirable to afford more control of the copying algorithm via, for example, an \texttt{ignore} keyword that identifies values that should not be automatically copied.


\subsection{Unifying Co-evolution Approaches}
The thesis research has focused on one type of software evolution, model-metamodel co-evolution. Many further types of evolution occur in practice, including model refactoring and model synchronisation, which were discussed in Chapter~\ref{LiteratureReview}. Changes to a metamodel affect not only models but also model management operations, such as model transformations. When changes are propagated from a metamodel to a model during migration, further artefacts might be impacted as an indirect consequence of the metamodel evolution.

The use of distinct structures and processes for each type of evolution poses usability challenges relating to the interoperability of tools and increased training effort. Seeking, instead, a unified approach to managing evolution might address these challenges, and presents an interesting opportunity for future work. Integrating the thesis research with approaches for managing other types of co-evolution, such as transformation-metamodel co-evolution, is one way in which the formulation of a unified approach might proceed. To this end, an outline for integrating model-metamodel and transformation-metamodel co-evolution approaches has proposed in collaboration with Anne Etien, an Associate Professor at the Universit\'{e} Lille, and published in \cite{rose10coevolution}.


\subsection{Higher-Order Migration}
In model transformation, a higher-order transformation consumes or produces a model transformation. Higher-order transformation has been used to generate model migration strategies \cite{cicchetti08thesis,garces09managing}, and to compose and analyse transformations \cite{tisi09hot}. Similarly, higher-order migration might be used effectively for migrating model transformation specifications between similar model transformation languages, and for migrating model management operations in response to changes to their specification language. For example, higher-order migration might be applied to migrate model migration strategies between different types of transformation language, such as from a new-target to a conservative copy language.


\subsection{Genericity}
Chapter~\ref{Evaluation} identified a lack of metamodel-independent re-use as one of the primary weaknesses of Flock compared to related approaches. In Flock, model migration strategies are specified in terms of metamodel concepts, and consequently, the extent to which code can be re-used across migration strategies is reduced. By \cc mixing model management languages with ideas from generic programming, one way in which model management operations can be specified in a manner that is independent of their metamodel has been identified \cite{delara10generic}. Applying these ideas to Flock would facilitate increased re-use across model migration strategies, and address a primary weakness of Flock. 
