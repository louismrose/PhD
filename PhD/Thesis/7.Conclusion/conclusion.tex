%!TEX root = /Users/louis/Documents/PhD/Deliverables/Thesis/thesis.tex

\chapter{Conclusions}
\label{Conclusion}
% The conclusion will provide a summary of the challenges addressed by, and the objectives of, our research. We will summarise the way in which we have approached the challenges and met the objectives, concluding with a summary of the evaluation and discussion of future work.

This thesis has investigated software evolution -- a key and costly development activity in software engineering \cite{moad90maintaining} -- in the context of model-driven engineering (MDE), a state-of-the art approach to software engineering. While MDE promises increased developer productivity \cite{watson08mdahistory} and increased portability of software systems \cite{frankel02mda}, it also poses several challenges that threaten its adoption. \cite{Mens07}, for example, suggest that identifying and managing evolutionary change in the context of MDE presents many open research challenges. The thesis research has contributed to the research challenges defined in \cite{Mens07} and has explored the following research hypothesis:

\begin{quote}
\emph{In existing MDE projects, the evolution of MDE development artefacts is typically managed in an ad-hoc manner with little regard for re-use. Dedicated structures and processes for managing evolutionary change can be designed by analysing evolution in existing MDE projects. Furthermore, supporting those dedicated structures and processes in contemporary MDE environments is beneficial in terms of increased productivity for software development activities pertaining to the management of evolutionary change.}
\end{quote} 

To explore the thesis hypothesis, the following research objectives were defined.

\begin{enumerate}
	\item Identify and analyse the evolution of MDE development artefacts in existing projects.
	\item Investigate the extent to which existing structures and processes can be used to manage the evolution of MDE development artefacts. 
	\item Propose and develop new structures and processes for managing the evolution of MDE development artefacts, and integrate those structures and processes with a contemporary MDE development environment.
	\item Evaluate the proposed structures and processes for managing evolutionary change, particularly with respect to productivity.
\end{enumerate}

The remainder of this section summarises the contributions of the thesis in relation to the thesis hypothesis and research objectives, and gives a brief description of and motivation for several potential extensions to the thesis research.


%!TEX root = /Users/louis/Documents/PhD/Deliverables/Thesis/thesis.tex

\section{Research Contributions}
The primary contributions of the thesis are summarised below, and the remainder of this section discusses each contribution in turn.

\begin{itemize}
	\item Chapter~\ref{Analysis} presented an investigation of evolution in existing MDE projects, which indicated ways in which models, metamodels and model management operations evolve, highlighted model-metamodel co-evolution as a commonly-occuring software evolution activity in MDE projects and led to a categorisation of existing approaches to managing model-metamodel co-evolution.
	\item Chapter~\ref{Implementation} described the design and implementation of structures and processes for performing model-metamodel co-evolution, which included a metamodel-independent syntax for managing non-conformant models, a textual modelling notation for manually migrating models, and a model-to-model transformation language tailored for migration. The proposed structures and processes are interoperable with the Eclipse Modeling Framework \cite{steinberg09emf}, arguably the most widely-used contemporary MDE modelling framework.
	\item Chapter~\ref{Evaluation} detailed the evaluation of the proposed structures and process using quantitive measurements, expert evaluation and application to large, independent examples of co-evolution and explored the extent to which the proposed structures and processes increase developer productivity and the understandability of model migration.
\end{itemize}


\subsection{Investigation of Evolution in MDE Projects}
The way in which evolution occurs and is managed in existing MDE projects was analysed in Chapter~\ref{Analysis}. The analysis investigated two types of evolutionary change, model-metamodel co-evolution and model synchronisation, which were identified from the review presented in Chapter~\ref{LiteratureReview}. For the MDE projects considered in Chapter~\ref{Analysis}, several suitable model-metamodel co-evolution examples -- and no suitable model synchronisation examples -- were located, and consequently the remainder of the thesis focused on model-metamodel co-evolution.

The co-evolution examples were used to identify the differences between existing approaches to managing model-metamodel co-evolution, and to investigate the way in which model-metamodel co-evolution is managed in existing MDE projects. The investigation led to the definition of two distinct approaches to managing model-metamodel co-evolution in MDE projects, \emph{user-driven} and \emph{developer-driven} and to a categorisation of existing approaches, which was published in \cite{rose09analysis} and has since been used and extended in several papers, including \cite{herrmannsdoerfer10extensive,jurack10towards,mendez10towards}.


\subsection{Structures and Processes for Managing Co-evolution}
The analysis of existing MDE projects presented in Chapter~\ref{Analysis} highlighted challenges for identifying and managing co-evolution. Managing co-evolution in a user-driven manner, for instance, is particularly challenging in contemporary MDE modelling environments because conformance is enforced implicitly and models and metamodels are kept separate. Similarly, the variation in programming and transformation languages typically used to specify model migration presents a challenge for comparing existing approaches to developer-driven co-evolution approaches. Moreover, none of the languages typically used have been tailored for the specific requirements of model migration. This thesis contribute structures and processes that seek to address the challenges summarised above.

\subsubsection{Metamodel-Indepenent Syntax}
Contemporary MDE modelling frameworks cannot be used to load non-conformant models. Consequently, model migration cannot be performed using the structures typically available in contemporary MDE modelling environments, such as model editors and model management operations. The metamodel-independent syntax, introduced in Chapter~\ref{Implementation}, is a proposed extension to contemporary MDE modelling frameworks that facilitates the loading of non-conformant models and provides services for reporting conformance problems.

The metamodel-independent syntax underpins the implementation of two further structures, the textual modelling notation described below, and the automatic conformance checking service introduced in \cite{rose10concordance}. 


\subsubsection{Textual Modelling Notation}
When a small number of models are to be migrated, the effort required to specify an executable migration strategy might not be justifiable. Instead, models can be migrated by editing models by hand. The textual modelling notation, presented in Chapter~\ref{Implementation}, provides a notation for editing models in contemporary MDE development environments. The notation proposed in this thesis adopts the Human-Usable Textual Notation (HUTN) \cite{hutn}, a standard notation for textual modelling proposed by the Object Management Group (OMG) \cite{omg}. The implementation of HUTN introduced in Section~\ref{sec:notation}, Epsilon HUTN, is the sole reference implementation of the HUTN standard, and was published in \cite{rose08hutn}.

During user-driven co-evolution, model editors cannot be used for migration and editing models in their underlying storage representation, which will not have been optimised for human use, can be error-prone and time consuming. The textual modelling notation introduced in Chapter~\ref{Implementation} provides an alternative to editing models in their underlying storage representation.


\subsubsection{A Process for User-Driven Co-evolution}
The analysis of existing MDE projects highlighted several projects in which model migration was performed using user-driven co-evolution techniques, and yet no existing work sought to address the specific requirements of user-driven co-evolution. Contemporary MDE modelling environments typically enforce conformance in an implicit manner, and cannot be used to load non-conformant models. Consequently, user-driven co-evolution is an iterative, error-prone and time-consuming task, because model editors and model management operations cannot be used to assist migration.

A typical process for performing user-driven co-evolution involves performing model migration by repeatedly switching between a model editor (which reports conformance problems) a text editor (in which conformance is reconciled by the user). Chapter~\ref{Evaluation} proposes an alternative process in which conformance reporting and reconciliation occur in the same environment, using the metamodel-independent syntax and textual modelling notation described above. The alternative process was published in \cite{rose09enhanced}.


\subsubsection{Epsilon Flock: A Model Migration Language}
In addition to the structures and processes for performing user-driven co-evolution, the thesis also contributes a structure dedicated to developer-driven co-evolution, a model migration language termed \emph{Epsilon Flock}. The analysis performed in Chapter~\ref{Analysis} showed that model migration is often specified with a model-to-model transformation language or with a general purpose programming language, and that theses languages are not tailored to the specific requirements of model migration. The investigation presented in Chapter~\ref{Implementation}, highlighted that a language tailored for model migration would provide several benefits over repurposing an existing language to specify model migration. Flock was designed and implemented to explore the extent to which a language tailored for model migration might increase developer productivity and the understandability of model migration. The investigation of languages for model migration and the design and implementation of Flock were published in \cite{rose10flock}.

Flock contributes a novel mechanism for relating source and target model elements termed \emph{conservative copy}, which is a hybrid of the two existing mechanisms used to relate source and target model elements in contemporary model-to-model transformation languages. Conservative copy automatically copies to the target model every source model element that conforms to the target metamodel, and does not copy to the target model source model elements that do not conform to the target metamodel.


\subsection{Evaluation of Structures and Processes}
The structures and processes introduced in this thesis have been evaluated using a variety of techniques, including a quantitive comparison, an expert evaluation and comparison to related processes and structures. Existing work on evolutionary change in MDE has typically been evaluated with a case study (such as in \cite{sprinkle03thesis}) or by demonstration (\cite{cicchetti08thesis}), and few papers in the area report the strengths and weaknesses proposed approaches, or seek to contextualise their contentions. The evaluation presented in Chapter~\ref{Evaluation} is a contribution in its own right as it presents several alternative evaluation techniques, including the first expert evaluation of model migration tools, and seeks to identify situations in which the proposed structures and processes are both effective and ineffective.

% Achievements:
% analysed examples of co-evolution
% - didn't find much data for model-trans co-evo
% - identifed user-driven co-evo
% - understood the way in which literature addresses co-evo
% 
% built tools
% - HUTN for user-driven co-evo
% - Flock: DSL for model migration
% 
% Provided a refernce implementation of an OMG standard
% - facilitates evaluation of the standard
% - standard is centered around usability, which can now be assessed
%
% evaluated tools
% - Compared user-driven co-evo processes
% - Showed Flock to be concise, 
% - Compare Flock with other migration tools and transformation tools
% - Applied Flock to a real-world example of co-evolution


% Discuss the way in which the method differs from methods used in similar projects:
% Identifying changes from existing projects has several benefits compared to the method typically used in managing software evolution, described above. Firstly, further research requirements can be identified from the solutions currently employed by existing projects. Secondly, related work can be more rigorously analysed and compared via application to existing projects. On the other hand, evaluating work produced by identifying changes from existing projects presents a challenge: more data may be required overall, as data used in the analysis should not be used in the evaluation.
%!TEX root = /Users/louis/Documents/PhD/Deliverables/Thesis/thesis.tex

\section{Future Work}
\label{sec:future_work}
In the context of a doctoral thesis, it is impossible to thoroughly investigate many of the issues raised in exploring the thesis hypothesis. Below, several potential extensions to the research presented in this thesis are identified, and any initial work in those areas is described. 

\subsection{Further Evaluation}
The extent to which the structures and processes introduced in Chapter~\ref{Implementation} increase developer productivity for co-evolution has been explored via expert evaluation and comparison to related work. Assessing the way in which the proposed structures and processes affect productivity is challenging due to the number of factors that affect productivity. In practice, for example, the proposed structures and processes would be used by developers with different expertise, and together with other tools and techniques. Evaluating the way in which software evolution is identified and managed in practice using comprehensive case and user studies would be likely to allow stronger claims to be made about the efficacy of the proposed structures and processes. Given the time constraints of a doctoral thesis, comprehensive case and user studies were not feasible, and hence are desirable extensions to the thesis research. A key first step would be to establish a common vocabulary for discussing software evolution activities in the context of MDE, which would facilitate the comparison of user experiences.

\subsection{Extensions to the Model Migration Language}
The model migration language proposed and implemented in Chapter~\ref{Implementation}, Epsilon Flock, makes idiomatic commonly occurring patterns of model migration. The evaluation presented in Chapter~\ref{Evaluation} suggested that migration strategies are often more concise when specified with Flock rather than when specified with contemporary model-to-model transformation languages. The evaluation also highlighted a limitation in the implementation of Flock and demonstrated further patterns that might be captured by model migration languages.

Addressing these issues would further improve the conciseness and re-usability of model migration strategies written in Flock and, hence, is an obvious area of future work. In particular, one language construct controls two concerns in the current implementation of Flock, and introducing separate language constructs for each concern could increase the potential for re-use between model migration rules. Applying Flock to the co-evolution examples used for evaluation highlighted further model migration patterns that might be made idiomatic in the language. For example, in situations where conservative copy can have side-effects, it may be desirable to afford more control of the copying algorithm via, for example, an \texttt{ignore} keyword that identifies values that should not be automatically copied.


\subsection{Unifying Co-evolution Approaches}
The thesis research has focused on one type of software evolution, model-metamodel co-evolution. Many further types of evolution occur in practice, including model refactoring and model synchronisation, which were discussed in Chapter~\ref{LiteratureReview}. Changes to a metamodel affect not only models but also model management operations, such as model transformations. When changes are propagated from a metamodel to a model during migration, further artefacts might be impacted as an indirect consequence of the metamodel evolution.

The use of distinct structures and processes for each type of evolution poses usability challenges relating to the interoperability of tools and increased training effort. Seeking, instead, a unified approach to managing evolution might address these challenges, and presents an interesting opportunity for future work. Integrating the thesis research with approaches for managing other types of co-evolution, such as transformation-metamodel co-evolution, is one way in which the formulation of a unified approach might proceed. To this end, an outline for integrating model-metamodel and transformation-metamodel co-evolution approaches has proposed in collaboration with Anne Etien, an Associate Professor at the Universit\'{e} Lille, and published in \cite{rose10coevolution}.


\subsection{Higher-Order Migration}
In model transformation, a higher-order transformation consumes or produces a model transformation. Higher-order transformation has been used to generate model migration strategies \cite{cicchetti08thesis,garces09managing}, and to compose and analyse transformations \cite{tisi09hot}. Similarly, higher-order migration might be used effectively for migrating model transformation specifications between similar model transformation languages, and for migrating model management operations in response to changes to their specification language. For example, higher-order migration might be applied to migrate model migration strategies between different types of transformation language, such as from a new-target to a conservative copy language.


\subsection{Genericity}
Chapter~\ref{Evaluation} identified a lack of metamodel-independent re-use as one of the primary weaknesses of Flock compared to related approaches. In Flock, model migration strategies are specified in terms of metamodel concepts, and consequently, the extent to which code can be re-used across migration strategies is reduced. By mixing model management languages with ideas from generic programming, \cite{delara10generic} have identified one way in which model management operations can be specified in a manner that is independent of their metamodel. Applying the ideas presented in \cite{delara10generic} to Flock would facilitate increased re-use across model migration strategies, and address a primary weakness of Flock. 



\section{Coda}
Building the systems demanded by society now and in the future will require new approaches to software engineering \cite{selic03pragmatics}. MDE is a state-of-the-art, principled approach to software engineering, and promises many benefits particularly with respect to the portability and maintainability of software systems \cite{kleppe03mda,frankel02mda}. While MDE shows promise, its success is reliant on the availability of mature and powerful tools. Such tools are beginning to emerge, but typically fail to address concerns that affect their applicability to the engineering of large and complex software systems, such as scalability and the cost of systems evolution.

The work presented in this thesis has demonstrated a systematic method for identifying challenges for software evolution in typical MDE processes, proposed structures and processes for addressing those challenges, and evaluated the structures and processes by comparison to related work and by application to real-world examples of evolution.