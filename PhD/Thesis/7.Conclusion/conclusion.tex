%!TEX root = /Users/louis/Documents/PhD/Deliverables/Thesis/thesis.tex

\chapter{Conclusion}
% The conclusion will provide a summary of the challenges addressed by, and the objectives of, our research. We will summarise the way in which we have approached the challenges and met the objectives, concluding with a summary of the evaluation and discussion of future work. 


% Discuss the way in which the method differs from methods used in similar projects:
% Identifying changes from existing projects has several benefits compared to the method typically used in managing software evolution, described above. Firstly, further research requirements can be identified from the solutions currently employed by existing projects. Secondly, related work can be more rigorously analysed and compared via application to existing projects. On the other hand, evaluating work produced by identifying changes from existing projects presents a challenge: more data may be required overall, as data used in the analysis should not be used in the evaluation.

\section{Closing Remarks}
- For software evolution research, need more data from industry.
- Closer collaboration between industry and academia to study evolution in context.
- More comprehensive assessment: for example, user studies.

\section{Future Work}
\label{sec:future_work}

\subsubsection{Generalisation}
Relate to section in limitations. Two dimensions:
-- Seek a unified approach to migrating all artefacts that might be affected by metamodel changes (models, model management operations, model editors, etc).
-- Re-use concepts from model-metamodel co-evolution to manage other types of co-evolution. Summarise paper with Anne.

\subsection{Extensions to Epsilon Flock}
\begin{itemize}
	\item Summarise limitations from Evaluation chapter.
	\item Discuss extensions to Flock to reduce duplication when subtyping.
	\subitem Try applying the changes suggested here\footnote{\url{http://lmr109.basecamphq.com/projects/1508853/posts/30822433/comments}} to the co-evo examples, and investigate whether they solve the subtyping problem / introduce other issues. 
	\item Discuss extensions to conservative copy when a reference changes to a containment
\end{itemize}