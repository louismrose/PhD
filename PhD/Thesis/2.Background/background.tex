%!TEX root = /Users/louis/Documents/PhD/Deliverables/Thesis/thesis.tex

\chapter{Background}
\label{Background}
This chapter surveys literature from the area in which the thesis research was conducted, Model-Driven Engineering (MDE). MDE is a principled approach to software engineering in which models are produced and consumed throughout the engineering process. Section~\ref{sec:mde_terms} introduces the terminology and fundamental principles used in MDE. Section~\ref{sec:mde_methods} reviews guidance and three methods for performing MDE. Section~\ref{sec:mde_tools} describes contemporary MDE environments. Two areas of research relating to MDE, domain-specific languages and language-oriented programming, are discussed in Section~\ref{sec:mde_related}. Finally, the benefits of and current challenges for MDE are described in Section~\ref{sec:mde_benefits_and_challenges}.

%!TEX root = /Users/louis/Documents/PhD/Deliverables/Thesis/thesis.tex

\section{MDE Terminology and Principles}
\label{sec:mde_terms}
Software engineers using MDE construct and manipulate artefacts familiar from traditional approaches to software engineering (such as code and documentation) and, in addition, work with different types of artefact, such as \emph{models}, \emph{metamodels} and \emph{model transformations}. Furthermore, MDE involves new development activities, such as \emph{model management}. This section describes the artefacts and activities involved in MDE.

\subsection{Models}
\label{subsec:models}
Models are fundamental to MDE. \cite{kurtev04thesis} identifies many definitions of the term model, such as: ``any subject using a system A that is neither directly nor indirectly interacting with a system B to obtain information about the system B, is using A as a model for B.'' \cite{apostel60models}, ``a model is a representation of a concept. The representation is purposeful and used to abstract from reality the irrelevant details.'' \cite{starfield90model}, and ``a model is a simplification of a system written in a well-defined language.'' \cite{bezivin01definition}. 

While there are many definitions of the term model, a common notion is that a model is a representation of the real-world \cite[pg12]{kurtev04thesis}. The part of the real-world represented by a model is termed the \emph{domain}, the \emph{object system} or, simply the \emph{system}. A further commonality is noted by \cite{kolovos06eol}: a model may have either a textual or graphical representation.

\cite{ackoff62scientific} defines \emph{analogous} models as those which share some characteristics and can be used in place of their object system. An aeroplane toy that can fly is an analogous model of an aeroplane. In computer science, models can be used to construct a computer system. A model of an object system, say the lending service of a library, might be used to decide the way in which data is stored on disk, or the way in which a program is to be structured.

\cite{jackson95software} proposes that the models constructed in computer science are analogous to two systems: the object system (e.g. the library lending service in the real-world) and the computer system (e.g. the combination of software and hardware used to implement a library lending service). A model can be used to think about both the real system and the computer system. Figure~\ref{fig:jackson_model} illustrates this notion further. According to \cite{jackson95software}, a model is both the description of the domain (object system) and the machine (computer system). Computer scientists switch between \emph{designations} when using a model to think about the object system or to think about the software system.

\begin{figure}[htbp]
  \begin{center}
    \leavevmode
    \includegraphics[width=10cm]{2.Background/images/jackson_model.png}
  \end{center}
  \caption[Jackson's definition of a model]{Jackson's definition of a model, taken from \cite[pg.125]{jackson95software}.}
  \label{fig:jackson_model}
\end{figure}

Models can be unstructured (for example, sketches on a piece of paper) or structured (conform to some well-defined set of syntactic and semantic constraints). In software engineering, models are used widely to reason about object systems and computer systems. MDE recognises this, and seeks to drive the development of computer systems from structured models.

\subsection{Modelling languages}
\label{subsec:modelling_languages}
In MDE, models are structured (satisfy a well-defined set of syntactic and semantic constraints) rather than unstructured \cite{kolovos09thesis}. A \emph{modelling language} is the set of syntactic and semantic constraints used to define the structure of a group of related models. In MDE, a modelling language is often specified as a model and, hence the term \emph{metamodel} is used in place of \emph{modelling language}.

\emph{Conformance} is a relationship between a metamodel and a model. A model \emph{conforms to} a metamodel when the metamodel specifies every concept used in the model definition, and the model uses the metamodel concepts according to the rules specified by the metamodel. Conformance can be described by a set of constraints between models and metamodels \cite{paige07metamodel}. When all constraints are satisfied, a model conforms to a metamodel. For example, a conformance constraint might state that every object in the model has a corresponding non-abstract class in the metamodel.

Metamodels facilitate model interchange and, therefore, interoperability between modelling tools.  For this reason, Evans recommends that software engineers ``use a well-documented shared language that can express the necessary domain information as a common medium of communication.'' \cite[pg377]{evans04domain}. To support this recommendation, Evans discusses the way in which chemists have collaborated to define a standardised language for describing chemical structures, Chemical Markup Language (CML)\footnote{\url{http://cml.sourceforge.net/}}. The standardisation of CML has facilitated interoperability between tools for specification, analysis and simulation.

A metamodel typically comprises three categories of constraint:

\begin{itemize}
	\item \textbf{The concrete syntax} provides a notation for constructing models that conform to the language. For example, a model may be represented as a collection of boxes connected by lines. A standardised concrete syntax enables communication. Concrete syntax may be optimised for consumption by machines (e.g. XML Metadata Interchange (XMI) \cite{xmi}) or by humans (e.g. the concrete syntax of the Unified Modelling Language (UML) \cite{uml212}).
	\item \textbf{The abstract syntax} defines the concepts described by the language, such as classes, packages, datatypes. The representation for these concepts is independent of the concrete syntax. For example, the implementation of a compiler might use an abstract syntax tree to encode the abstract syntax of a program (whereas the concrete syntax for the same language may be textual or graphical).
	\item \textbf{The semantics} identifies the meaning of the modelling concepts in the particular domain of language. For example, consider a modelling language defined to describe genealogy, and another to describe flora. Although both languages may define a tree construct, the semantics of a tree in one is likely to be different from the semantics of a tree in the other. The semantics of a modelling language may be specified rigorously, by defining a reference semantics in a formal language such as Z \cite{z}, or in a semi-formal manner by employing natural language.
\end{itemize}

Concrete syntax, abstract syntax and semantics are used together to specify modelling languages. There are many other ways of defining languages, but this approach (first formalised in \cite{alvarez01mml}) is common in model-driven engineering: a metamodel is often used to define abstract syntax, a grammar or text-to-model transformation to specify concrete syntax, and code generators, annotated grammars or behavioural models to effect semantics.

\subsection{MOF: A metamodelling language}
\label{subsec:mof}
Software engineers using MDE can use existing and define new metamodels. To facilitate interoperability between MDE tools, the OMG has standardised a language for specifying metamodels, the Meta-Object Facility (MOF). Metamodels specified in MOF can be interchanged between MDE environments. Furthermore, modelling language tools are interoperable because MOF also standardises the way in which metamodels and their models are persisted to and from disk. For model and metamodel persistence, MOF prescribes XML Metadata Interchange (XMI), a dialect of XML optimised and standardised by the OMG for loading, storing and exchanging models.

Because MOF is a modelling language for describing modelling languages, it is sometimes termed a metamodelling language. Part of the UML metamodel, defined in MOF, is shown in Figure~\ref{fig:mof}. As discussed in Section~\ref{sec:mde_tools}, different kinds of concrete syntax can be used for MOF. Figure~\ref{fig:mof}, for example, uses a concrete syntax similar to that of UML class diagrams. Specifically:

\begin{itemize}
 \item Modelling constructs are drawn as boxes. The name of each modelling construct is emboldened. The name of abstract (uninstantiable) constructs are italicised.
 \item Attributes are contained within the box of their modelling construct. Each attribute has a name, a type (prefixed with a colon) and may define a default value (prefixed with an equals sign).
 \item Generalisation is represented using a line with an open arrow-head.
 \item References are specified using a line. An arrow illustrates the direction in which the reference may be traversed (no arrow indicates bi-directionality). Labels are used to name and define the multiplicity of references.
 \item Containment references are specified by including a solid diamond on the containing end.
\end{itemize}

\begin{figure}[htbp]
  \begin{center}
    \leavevmode
    \includegraphics[scale=0.33]{2.Background/images/mof.png}
  \end{center}
  \caption[A fragment of the UML metamodel defined in MOF]{A fragment of the UML metamodel defined in MOF, from \cite{uml212}.}
  \label{fig:mof}
\end{figure}

Specifying modelling languages with a common metamodelling language, such as MOF, ensures consistency in the way in which modelling constructs are specified. MOF has facilitated the construction of interoperable MDE tools that can be used with a range of modelling languages. Without a standardised metamodelling language, modelling tools were specific to one modelling language, such as UML. In contemporary MDE environments, any number of modelling languages can be used together and manipulated in a uniform manner.

Furthermore, when modelling languages are specified without using a common metamodelling language, identifying similarities between modelling languages is challenging \cite[pg97]{frankel02mda}. The sequel discusses the way in which models and metamodels are used to construct systems in MDE.

\subsection{Model Management}
\label{sec:mde}
In MDE, models are \emph{managed} to produce software. \cite{melnik04thesis} first described \emph{model management} as a collection of operators for manipulating models. \cite{kolovos09thesis} explores a means for increasing the interoperability of model management operations. This thesis uses the term \emph{model management} to refer to development activities that manipulate models for the purpose of producing software. Model management activities typical in MDE, such as model transformation and validation, are discussed in this section. Section~\ref{sec:mde_methods} discusses MDE guidelines and methods, and describes the way in which model management activities are used together to produce software in MDE.

\subsubsection{Model Transformation}
\label{subsubsec:model_transformation}
Model transformation is a development activity in which software artefacts are derived from others, according to some well-defined specification. Three different types of model transformation are described in \cite{kleppe03mda,kolovos09thesis}. Model transformations are specified between modelling languages (model-to-model transformation), between modelling languages and textual artefacts (model-to-text-transformation) and between textual artefacts and modelling languages (text-to-model transformation). Each type of transformation has unique characteristics and tools, but share some common characteristics. The remainder of this section first introduces the commonalities and then discusses each type of transformation individually.

\paragraph{Common characteristics of model transformations}
The input to a transformation is termed its \emph{source}, and the output its \emph{target}. In theory, a transformation can have more than one source and more than one target, but not all transformation languages support multiple sources and targets. Consequently, much of the model transformation literature considers single source and target transformations.

\cite{czarnecki06survey} describes a feature model for distinguishing and categorising model transformation approaches. Two of the features are relevant to the research presented in this thesis, and are now discussed. 

\subparagraph{Source-target relationship} A \emph{new-target} transformation creates target models afresh on each invocation. An \emph{existing-target} transformation is executed on existing target models. Existing target transformations are used for partial (incremental) transformation and for preserving parts of the target that are not derived from the source.

\subparagraph{Domain language} Transformations specified between a source and a target model that conform to the same metamodel are termed \emph{endogenous} or \emph{rephrasings}, while transformations specified between a source and a target model that conform to different metamodels are termed \emph{exogenous} or \emph{translations}.

Endogenous, existing-target transformations are a special case of transformation and are termed \emph{refactorings}. Refactorings have been studied in the context of software evolution and are discussed more thoroughly in Chapter~\ref{LiteratureReview}.


\paragraph{Model-to-Model (M2M) Transformation} M2M transformation is used to derive models from others. By automating the derivation of models from others, M2M transformation has the potential to reduce the cost of engineering large and complex systems that can be represented as a set of interdependent models \cite{sendall03heart}. 

M2M transformations are often specified as a set of \emph{rules} \cite{czarnecki06survey}. Each rule specifies the way in which a specific set of elements in the source model is transformed to an equivalent set of elements in the target model \cite[pg.44]{kolovos09thesis}.

Many M2M transformation languages have been proposed, such as the Atlas Transformation Language (ATL) \cite{jouault05transforming}, the Epsilon Transformation Language (ETL) \cite{kolovos08etl} and VIATRA \cite{VIATRA}. The OMG \cite{omg} provide a standardised M2M transformation language, Queries/Views/Transformations (QVT) \cite{qvt}. M2M transformation languages can be categorised according to their \emph{style}, which is either declarative, imperative or hybrid.

Declarative M2M transformation languages only provide constructs for mapping source to target model elements and, as such, are not computationally complete. Consequently, the scheduling of rules can be \emph{implicit} (determined by the execution engine of the transformation language). By contrast, imperative M2M transformation languages are computationally complete, but often require rule scheduling to be \emph{explicit} (specified by the user). Hybrid M2M transformation languages combine declarative and imperative parts, are computationally complete, and provide a mixture of implicit and explicit rule scheduling.

Because declarative M2M transformation languages cannot be used to solve some categories of transformation problem \cite{patrascoiu04embedding} and imperative M2M transformation languages are argued to be difficult to write and maintain \cite[pg.45]{kolovos09thesis}, \cite{kolovos08etl} notes that the current consensus is that hybrid languages, such as ATL are more suitable for specifying model transformation than pure imperative or declarative languages.

An exemplar M2M transformation, written in the hybrid M2M transformation language ETL, is shown in Listing~\ref{lst:exemplar_m2m}. The source of the transformation is a state machine model, conforming to the metamodel shown in Figure~\ref{fig:state_machine_mm}. The target of the transformation an object-oriented model, conforming to the metamodel shown in Figure~\ref{fig:object_oriented_mm}. The transformation in Listing~\ref{lst:exemplar_m2m} comprises two rules.

\begin{figure}[htbp]
  \begin{center}
    \leavevmode
    \includegraphics[width=5cm]{2.Background/images/StateMachines.pdf}
  \end{center}
  \caption{Exemplar State Machine metamodel.}
  \label{fig:state_machine_mm}
\end{figure}

\begin{figure}[htbp]
  \begin{center}
    \leavevmode
    \includegraphics[width=8.5cm]{2.Background/images/OO.pdf}
  \end{center}
  \caption{Exemplar Object-Oriented metamodel.}
  \label{fig:object_oriented_mm}
\end{figure}

\begin{lstlisting}[caption=Exemplar M2M transformation in the Epsilon Transformation Language \cite{kolovos08etl}, label=lst:exemplar_m2m, language=ETL]
rule Machine2Package
	transform m : StateMachine!Machine
	to        p : ObjectOriented!Package {
		
	p.name     := 'uk.ac.york.cs.' + m.id;
	p.contents := m.states.equivalent();
}

rule State2Class
	transform s : StateMachine!State
	to        c : ObjectOriented!Class 
	
	guard: not s.isFinal {
		
	c.name := s.name + 'State';
}
\end{lstlisting}

The first rule (lines 1-7) is named \texttt{Machine2Package} (line 1) and transforms \emph{Machine}s (line 2) into \emph{Package}s (line 3). The body of the first rule (lines 5-6) specifies the way in which a \texttt{Package}, \texttt{p}, can be derived from a \texttt{Machine}, \texttt{m}. Specifically, the \texttt{name} of \texttt{p} is derived from the \texttt{id} of \texttt{m} (line 5), and the \texttt{contents} of \texttt{p} are derived from the \texttt{states} of \texttt{m} (line 6). 

The second rule (lines 9-16) transforms \texttt{State}s (line 10) to \texttt{Class}es (line 11). Additionally, line 13 contains a \emph{guard} to specify that the rule is only to be applied to \texttt{State}s whose \texttt{isFinal} property is \texttt{false}.

When executed, the transformation rules will be scheduled \textbf{implicitly} by the execution engine, and invoked once for each \texttt{Machine} and \texttt{State} in the source. On line 6 of Listing~\ref{lst:exemplar_m2m}, the built-in \texttt{equivalent()} operation is used to produce a set of \texttt{Class}es from a set of \texttt{State}s by invoking the relevant transformation rule. This is an example of \textbf{explicit} rule scheduling, in which the user defines when a rule will be called.


\paragraph{Model-to-Text (M2T) Transformation} M2T transformation is used for model serialisation (enabling model interchange), code and documentation generation, and model visualisation and exploration.  In 2005, the OMG \cite{omg} recognised the lack of a standardised M2T transformation with its M2T Language Request for Proposals \footnote{\url{http://www.omg.org/docs/ad/04-04-07.pdf}}. In response, various M2T languages have been developed, including JET\footnote{\url{http://www.eclipse.org/modeling/m2t/?project=jet#jet}}, XPand\footnote{\url{http://www.eclipse.org/modeling/m2t/?project=xpand}}, MOFScript \cite{oldevik05toward} and the Epsilon Generation Language (EGL) \cite{rose08egl}.

Because M2T transformation is used to produce unstructured rather than structured artefacts, M2T transformation has different requirements to M2M transformation. For instance, M2T transformation languages often provide mechanisms for specifying sections of text that will be completed manually and must not be overwritten by the transformation engine.

\emph{Template}s are commonly used in M2T languages. Templates comprise \emph{static} and \emph{dynamic} sections. When the transformation is invoked, the contents of static sections are emitted verbatim, while dynamic sections contain logic and are executed.

An exemplar M2T transformation, written in EGL, is shown in Listing~\ref{lst:exemplar_m2t}. The source of the transformation is an object-oriented model conforming to the metamodel shown in Figure~\ref{fig:object_oriented_mm}, and the target is Java source code. The template assumes that an instance of \texttt{Class} is stored in the \texttt{class} variable.

\begin{lstlisting}[caption=Exemplar M2T transformation in the Epsilon Generation Language \cite{rose08egl}, label=lst:exemplar_m2t, language=EGL]
package [%=class.package.name];

public class [%=class.name] {
	[% for(attribute in class.attributes) { %]
	  private [%=attribute.type%] [%=attribute.name];
	[% } %]
}
\end{lstlisting}

In EGL, dynamic sections are contained within \texttt{[\%} and \texttt{\%]}. \emph{Dynamic output} sections are a specialisation of dynamic sections contained within \texttt{[\%=} and \texttt{\%]}. The result of evaluating a dynamic output section is included in the generated text. Line 1 of Listing~\ref{lst:exemplar_m2t} contains two static sections (\texttt{package}' and \texttt{;}) and a dynamic output section (\texttt{[\%=class.package.name]}), and will generate a package declaration when executed.  Similarly, line 3 will generate a class declaration. Lines 4 to 6 iterate over every \texttt{attribute} of the \texttt{class}, outputting a field declaration for each \texttt{attribute}.


\paragraph{Text-to-Model (T2M) Transformation} T2M transformation is most often implemented as a parser that generates a model rather than object code. Parser generators such as ANTLR \cite{parr07antlr} can be used to produce a structured artefact (such as an abstract syntax tree) from text. T2M tools are built atop parser generators and post-process the structured artefacts such that they conform to a metamodel specified by the user.

Xtext\footnote{\url{http://www.eclipse.org/Xtext/}} and EMFtext \cite{heidenreich09derivation} are contemporary examples of T2M tools that, given a grammar and a target metamodel, will automatically generate a parser that transforms text to a model.

An exemplar T2M transformation, written in EMFtext, is shown in Listing~\ref{lst:exemplar_t2m}. From the transformation shown in Listing~\ref{lst:exemplar_t2m}, EMFtext can be used to generate a parser that, when executed, will produce state machine models. For the input, \texttt{lift[stationary up down stopping emergency]}, the parser will produce a model containing one \texttt{Machine} with \texttt{lift} as its \texttt{id}, and five \texttt{State}s with the \texttt{name}s, \texttt{stationary}, \texttt{up}, \texttt{down}, \texttt{stopping}, and \texttt{emergency}.

\begin{lstlisting}[caption=Exemplar T2M transformation in EMFtext, label=lst:exemplar_t2m, language=EMFtext]
SYNTAXDEF statemachine
FOR <statemachine>
START Machine

TOKENS {
	DEFINE IDENTIFIER $('a'..'z'|'A'..'Z')*$;
	DEFINE LBRACKET $'['$;
	DEFINE RBRACKET $']'$;
}

RULES {
	Machine ::= id[IDENTIFIER] LBRACKET states* RBRACKET  ;
	State ::= name[IDENTIFIER] ;
}
\end{lstlisting}

Lines 1-2 of Listing~\ref{lst:exemplar_t2m} define the name of the parser and target metamodel. Line 3 indicates that parser should first seek to construct a \texttt{Machine} from the source text. Lines 5-9 define rules for the lexer, including a rule for recognising \texttt{IDENTIFIER}s (represented as alphabetic characters).  

Lines 11-14 of Listing~\ref{lst:exemplar_t2m} are key to the transformation. Line 11 specifies that a \texttt{Machine} is constructed whenever an \texttt{IDENTIFIER} is followed by a \texttt{LBRACKET} and eventually a \texttt{RBRACKET}. When constructing a \texttt{Machine}, the first time an \texttt{IDENTIFIER} is encountered, it is stored in the \texttt{id} attribute of the \texttt{Machine}. The \texttt{states*} statement on line 12 indicates that, before matching a \texttt{RBRACKET}, the parser is permitted to transform subsequent text to a \texttt{State} (according to the rule on line 13) and store the resulting \texttt{State} in the \texttt{states} reference of the \texttt{Machine}. The asterisks in \texttt{states*} indicates that any number of \texttt{State}s can be constructed and stored in the \texttt{states} reference.

\subsubsection{Model Validation}
Model validation provides a mechanism for managing the integrity of the software developed using MDE. A model that omits information is said to be \emph{incomplete}, while related models that suggest differences in the underlying phenomena are said to be \emph{contradicting} \cite{kolovos09thesis}. Incompleteness and contradiction are two examples of \emph{inconsistency}. In MDE, inconsistency is detrimental, because, when artefacts are automatically derived from each other, the inconsistency of one artefact might be propagated to others. Model validation is used to detect, report and reconcile inconsistency throughout a MDE process.

\cite{kolovos09thesis} observes that inconsistency detection is inherently pattern-based and, hence, higher-order languages are more suitable for model validation than so-called ``third-generation'' programming languages (such as Java). The Object Constraint Language (OCL) \cite{ocl2} is an OMG standard that can be used to specify consistency constraints on UML and MOF models. OCL cannot be used to specify inter-model constraints, unlike the xlinkit toolkit \cite{nentwich2003flexible} and the Epsilon Validation Language (EVL) \cite{kolovos08evl}.

An exemplar model validation constraint, written in EVL, is shown in Listing~\ref{lst:exemplar_validation}. The constraint validates state machine models that conform to the metamodel shown in Figure~\ref{fig:state_machine_mm}. The constraint shown in Listing~\ref{lst:exemplar_validation} is defined for \texttt{State}s (line 1), and checks that there exists some transition whose source or target is the current state (line 4). When the check part (line 4) is not satisfied, the message part (line 6) is displayed. When executed, the EVL constraint will be invoked once for every \texttt{State} in the model. The keyword \texttt{self} is used to refer to the particular \texttt{State} on which the constraint is currently being invoked.

\begin{lstlisting}[caption=Exemplar model validation in the Epsilon Validation Language, label=lst:exemplar_validation, language=EVL]
context State {
	constraint NoStateIsAnIsland {
		check:
			Transition.all.exists(t | t.source == self or t.target == self)
		message:
		  'The state ' + self.name + ' has no transitions.'
	}
}
\end{lstlisting}

\subsubsection{Further model management activities}
In addition to model transformation and validation, further examples of model management activities include model comparison (e.g. \cite{kolovos06ecl}), in which a \emph{trace} of similar and different elements is produced from two or more models, and model merging or weaving (e.g. \cite{kolovos07eml}), in which two or more models are combined to produce a unified model.

Further activities, such as model versioning and tracing, might be regarded as model management but, in the context of this thesis, are considered as evolutionary activities and as such are discussed in Chapter~\ref{LiteratureReview}.

\subsection{Summary}
This section has introduced the terminology and principles necessary for discussing MDE in this thesis. Models provide abstraction, capturing necessary and disregarding irrelevant details. Metamodels provide a structured mechanism for describing the syntactic and semantic rules to which a model most conform. Metamodels facilitate interoperability between modelling tools and MOF, the OMG standard metamodelling language, enables the development of tools that can be used with a range of metamodels, such as model management tools. Throughout model-driven engineering, models are manipulated to produce other development artefacts using model management activities such as model transformation and validation. Using the terms and principles described in this section, the ways in which model-driven engineering is performed in practice are now discussed.
%!TEX root = /Users/louis/Documents/PhD/Deliverables/Thesis/thesis.tex

\section{MDE Guidelines and Methods}
\label{sec:mde_methods}
For performing MDE, new engineering practices and processes have been proposed. Proponents of MDE have produced guidance and methods for MDE. This section discusses the guidance for MDE set out in the Model-Driven Architecture \cite{mda} and the methods for MDE described in \cite{stahl06mdsd,kelly08dsm,greenfield04software}. 

\subsection{The Model-Driven Architecture (MDA)}
The Model-Driven Architecture (MDA) is a software engineering framework defined by the OMG. The MDA provides guidelines for MDE. For instance, the MDA prescribes the use of a Platform Independent Model (PIM) and one or more Platform Specific Models (PSMs).

A PIM provides an abstract, implementation-agnostic view of a system. Successive PSMs provide increasingly more implementation detail. Inter-model mappings are used to forward- and reverse-engineer these models, as depicted in
Figure \ref{fig:mda}.

\begin{figure}[htbp]
  \begin{center}
    \leavevmode
    \includegraphics[scale=0.5]{2.Background/images/PIMs_and_PSMs.png}
  \end{center}
  \caption{Interactions between a PIM and several PSMs.}
  \label{fig:mda}
\end{figure}

A key difference between the MDA and related approaches, such as round-trip engineering (in which models and code are co-evolved to develop a system), is that the MDA prescribes automated transformations between PIM and PSMs, whereas other approaches use some manual transformations.

With MDA, the OMG sought to communicate and encourage the adoption of MDE principles, and to provide standards for building interoperable MDE platforms. Arguably, some parts of MDA have been widely adopted. For example, the metamodelling language provided by EMF is heavily based on MOF (Section~\ref{subsec:mof}), one of the key standards prescribed by MDA. However, it is difficult to assess the extent to which the principles advocated by MDA have been adopted. Empirical analysis is needed to determine the way in which MDE is performed in practice, and to drive changes to MDA and the modelling standards provided by the OMG.

\subsubsection{Standards for the MDA}
As part of the guidelines for MDE, the OMG prescribes a set of standards for the MDA. The standards are allocated to one of four tiers, and each tier represents a different level of model abstraction. Members of one tier conform to a member of the tier above. The four tiers described in the MDA are shown in Figure \ref{fig:mda-pyramid}, and a short discussion based on \cite[Section 8.2]{kleppe03mda} follows.

\begin{figure}[htbp]
  \begin{center}
    \leavevmode
    \includegraphics[scale=0.5]{2.Background/images/mda-pyramid.png}
  \end{center}
  \caption{The tiers of standards used as part of MDA.}
  \label{fig:mda-pyramid}
\end{figure}

The base of the pyramid, tier M0, contains the domain (real-world). When modelling a business, this tier is used to describe items of the business itself, such as a real customer or an invoice. When modelling software, M0 instances describe the software representation of such items. M1 contains models (Section~\ref{subsec:models}) of the concepts in M0, for example a customer may be represented as a class with attributes. The M2 tier contains the modelling languages (metamodels, Section~\ref{subsec:modelling_languages}) used to describe the contents of the M1 tier. For example, if UML \cite{uml212} models were used to describe concepts as classes in the M1 tier, M2 would contain the UML metamodel. Finally, M3 contains a metamodelling language (metametamodel, Section~\ref{subsec:mof}) which describes the modelling languages in the M2 tier. As discussed in Section~\ref{subsec:mof}, the M3 tier facilitates tool standardisation and interoperability. The MDA specifies the Meta-Object Facility (MOF) \cite{mof} as the only member of the M3 tier.

\subsubsection{Interpretations of the MDA}
\cite{mcneile03mda} identifies two ways in which engineers have interpreted the MDA. Both interpretations begin with a PIM, but the way in which executable code is produced differs:

\begin{itemize}
 \item \textbf{Translationist}: The PIM is used to generate code directly using a sophisticated code generator. Any intermediate PSMs are internal to the code generator. No generated artefacts are edited manually.
 \item \textbf{Elaborationist}: Any generated artefacts (such as PSMs, code and documentation) can be augmented with further details of the application. To ensure that all models and code are synchronised, tools must allow bi-directional transformations.
\end{itemize}

Translationists must encode behaviour in their PIMs \cite{mellor02executable}, whereas elaborationists have a choice, electing to specify behaviour in PSMs or in code \cite{kleppe03mda}.

The difference between translationist and elaborationist approaches to MDE is related to a difference in the way in which models are viewed in traditional approaches to software engineering. For example, \cite{evans04domain} proposes the use of models throughout the development process, and the way in which code is structured is driven by the model. By contrast, \cite[ch. 14]{martin06agile} prescribes modelling only for communicating and reasoning about a design, and not ``as a long-term replacement for real, working software''. Rather \cite{martin06agile} advocate using models to quickly compare different ways in which a system might be structured and then disregarding those models in favour of working code.

% \subsection{Models in Software Engineering}
% 
% \emph{\textbf{Richard, Fiona}: This section is the result of reconciling an argument  I was having with myself! I agree with most of the principles in \cite{martin06agile}, but disagree that code should always be preferred to models. It took me quite a while to come to the conclusions reached in this section, but I'm not sure if the section adds enough value to warrant inclusion in the thesis.} \vspace{4mm}
% 
% The models that \cite{jackson95software} describes are used widely in software engineering. This section considers two software engineering approaches, each with  different opinions of the role of models in software engineering. \cite{evans04domain} advocates modelling throughout software development to better understand the object system, and prescribes structuring software based on the terms and concepts identified in the object system. \cite{martin06agile} advocates that the primary product of software engineering is code, and as such, models should be used to explore ideas, but then disregarded in favour of code.
% 
% \subsubsection{In Favour of Modelling}
% \cite{evans04domain} proposes the use of models throughout the development process to capture and communicate knowledge of the object system and to shape the structure of the resulting software system. Wherever possible, the way in which code is structured is driven by the model. Throughout, \cite{evans04domain} emphasises the importance of modelling and a process, which he terms \emph{refactoring to deeper insight}, that seeks incremental improvements to models.
% 
% \begin{quote}
% Distillation is the process of separating the components of a mixture to extract the essence in a form that makes it more valuable and useful. A model is a distillation of knowledge. With every refactoring to deeper insight, we abstract some crucial aspect of domain knowledge and priorities. \cite[pg397]{evans04domain}
% \end{quote}
% 
% After each refactoring to deeper insight, the model should more closely represent the object system, and the computer system is changed to reflect the newfound knowledge. 
% 
% \subsubsection{In Favour of Coding}
% By contrast, \cite[ch. 14]{martin06agile} prescribes modelling only for communicating and reasoning about a design, and not ``as a long-term replacement for real, working software''. Rather \cite{martin06agile} advocates using models to quickly compare different ways in which a system might be structured and then to disregard those models in favour of working code.
% 
% \cite{martin06agile} motivates this position by noting that, in some engineering disciplines, models are used to reduce risk. Structural engineers build models of bridges. Aerospace engineers build models of aircraft. In these disciplines, a model is used to determine the efficacy of the object system and, moreover, is cheaper to build and test than the object system, often by a huge factor. The produce of many engineering disciplines is physical and the manufacturing process costly. By contrast, \cite{martin06agile} argues that software models are often not much cheaper to build and test than the software they represent. Consequently, \cite{martin06agile} proposes that working code is to be favoured over models.
% 
% In this justification, it is clear that \cite{martin06agile} favours code over \emph{models of computer systems}, but makes no argument for favouring code over \emph{models of object systems}. In fact, it seems that for \cite{martin06agile}, the code is the model of the object system.
% 
% \subsubsection{Models in MDE}
% As the discussion above demonstrates, the way in which models are used and regarded varies in software engineering. In \cite{evans04domain}, models are key -- they are used to shape the solution's design, to define a common vocabulary for communication between team members, and to distinguish interesting and uninteresting elements of the object system. In \cite{martin06agile}, code is the primary artefact used to describe the object system and the computer system, and can be regarded as a model.
% 
% MDE recognises that models -- albeit in different forms -- are used throughout software engineering. Furthermore, MDE seeks to capture the way in which models can be used to develop software, as discussed in Section~\ref{sec:mde}. To facilitate this, models are structured and, hence, are amenable to manipulation. The sequel describes modelling languages, which are used to describe the way in which groups of related models are structured.


\subsection{Methods for MDE}
Several methods for MDE are prevalent today. In this section, three of the most established MDE methods are discussed: Architecture-Centric Model-Driven Software Development \cite{stahl06mdsd}, Domain-Specific Modelling \cite{kelly08dsm} and Microsoft's Software Factories \cite{greenfield04software}. All three methods have been defined by MDE practitioners, and have been used repeatedly to solve problems in industry. The methods vary in the extent to which they follow the guidelines set out by the MDA.

\subsubsection{Architecture-Centric Model-Driven Software Development}
Model-Driven Software Development is the term given to MDE by \cite{stahl06mdsd}. The style of MDE that \cite{stahl06mdsd} describes, \textit{architecture-centric model-driven software development} (AC-MDSD), focuses on generating the infrastructure of large-scale applications. For example, a typical J2EE application contains concepts (such as EJBs, descriptors, home and remote interfaces) that ``admittedly contain domain-related information such as method signatures, but which also exhibit a high degree of redundancy'' \cite{stahl06mdsd}. Using code generation, AC-MDSD seeks to eliminate this kind of redundancy. Domain-related information is specified in a single source (typically in a model). The single source is used as input to code generators, which automatically produce the implementation concepts.

AC-MDSD applies more of the MDA guidelines than the other methods discussed below. For instance, AC-MDSD supports the use of a general-purpose modelling language for specifying models. \cite{stahl06mdsd} utilise UML in many of their examples, which demonstrate how AC-MDSD may be used to enhance the productivity, efficiency and understandability of software development. In these examples, models are annotated using UML profiles to describe domain-specific concepts.


\subsubsection{Domain-Specific Modelling}
\cite{kelly08dsm} present Domain-Specific Modelling (DSM), a collection of principles, practices and advice for constructing systems using MDE. DSM is based on the translationist interpretation of the MDA: domain models are transformed directly to code. In motivating the need for DSM, Kelly and Tolvanen state that large productivity gains were made when third-generation programming languages were used in place of assembler, and that no paradigm shift has since been able to replicate this degree of improvement. Tolvanen\footnote{Tutorial on Domain Specific Modelling for Full Code Generation at the Fourth European Conference on Model Driven Architecture (ECMDA), June 2008, Berlin, Germany.} notes that DSM focuses on increasing the productivity of software engineering by allowing developers to specify solutions by using models that describe the application domain.

To perform DSM, expert developers define:

\begin{itemize}
 \item \textbf{A domain-specific modelling language}: allowing domain experts to encode solutions to their problems.
 \item \textbf{A code generator}: that translates the domain-specific models to executable code in an existing programming language.
 \item \textbf{Framework code}: that encapsulates the common areas of all applications in this domain.
\end{itemize}

As the development of these three artefacts requires significant effort from expert developers, Tolvanen\footnotemark[\value{footnote}] states that DSM should only be applied if more than three problems specific to the same domain are to be solved.

Tools for defining domain-specific modelling languages, editors and code generators enable DSM \cite{kelly08dsm}. Reducing the effort required to specify these artefacts is key to the success of DSM. In this respect, DSM resembles a programming paradigm termed \textit{language-oriented programming} (LOP), which also requires tools to simplify the specification of new languages. LOP is discussed further in Section \ref{sec:mde_related}.

Throughout \cite{kelly08dsm}, examples from industrial partners are used to argue that DSM can improve developer productivity. Unlike the MDA, DSM appears to be optimised for increasing productivity, and less concerned with portability or maintainability. Therefore, DSM is less suitable for engineering applications that frequently interoperate with -- and are underpinned by -- changing technologies.

\subsubsection{Microsoft Software Factories}
\cite[pg159]{greenfield04software} state that industrialisation of the automobile industry has addressed problems with economies of scale (mass production) and scope (product variation). Software Factories, a software engineering method developed at Microsoft, seeks to address problems with economies of scope in software engineering by borrowing concepts from product-line engineering. \cite{greenfield04software} argue that, unlike many other engineering disciplines, software development requires considerably more development effort than production effort in that scaling software development to account for scope is significantly more complicated than mass production of the same software system.

The Software Factories method \cite{greenfield04software} prescribes a bottom-up approach to abstraction and re-use. Development begins by producing prototypical applications. The common elements of these applications are identified and abstracted into a product-line. When instantiating a product, models are used to choose values for the variation points in the product. To simplify the creation of these models, Software Factories propose model creation wizards. \cite[pg179]{greenfield04software} state that ``moving from totally-open ended hand-coding to more constrained forms of specification [such as wizard-based feature selection] are the key to accelerating software development.'' By providing explanations that assist in making decisions, the wizards used in Software Factories guide users towards best practices for customising a product.

Compared to DSM, the Software Factories method appears to provide more support for addressing portability problems. The latter provides \textit{viewpoints} into the product-line (essentially different ways of presenting and aggregating data from development artefacts), which allow decoupling of concerns (e.g. between logical, conceptual and physical layers). Viewpoints provide a mechanism for abstracting over different layers of platform independence, adhering more closely than DSM to the guidelines provided in the MDA. Unlike the guidelines provided in the MDA, the Software Factories method does not insist that development artefacts be derived automatically where possible.

Finally, the Software Factories method prescribes the use of domain-specific languages (discussed in Section~\ref{subsec:dsls}) for describing models in conjunction with Software Factories, rather than general-purpose modelling languages, as the authors of Software Factories state that the latter often have imprecise semantics \cite{greenfield04software}.

\subsection{Summary}
This section has discussed the ways in which process and practices for MDE have been captured. Guidance for MDE has been set out in the MDA standard, which seeks to use MDE to produce adaptable software in a productive and maintainable manner. Three methods for performing MDE have been discussed.
 
The methods discussed share some characteristics. They all require a set of exemplar applications, which are examined by MDE experts. Analysis of the exemplar applications identifies the way in which software development may be decomposed. A modelling language for the problem domain is constructed, and instances are used to generate future applications. Code common to all applications in the problem domain is encapsulated in a framework.

Each method has a different focus. AC-MDSD seeks to eliminate duplication of information from the problem domain via automatic code generation, and targets enterprise applications. The Software Factories method concentrates on providing different viewpoints into the system, and facilitating collaborative specification of a system. DSM aims to improve reusability between solutions to problems in the same problem domain, and hence improve developer productivity.

Perhaps unsurprisingly, the proponents of each method for MDE recommend a single tool (such as MetaCase for DSM). Alternative tools are available from open-source modelling communities, including the Eclipse Modelling Project, which provides -- among other MDE tools -- arguably the most widely used MDE modelling framework. Two MDE tools are reviewed in the sequel.
%!TEX root = /Users/louis/Documents/PhD/Deliverables/Thesis/thesis.tex

\section{Tools for MDE}
\label{sec:mde_tools}
Mature and powerful tools and languages for many common MDE activities are available today. This section discusses two MDE tools that are well-suited for MDE research and that are used in the remainder of the thesis.

Section~\ref{subsec:emf} provides an overview of the Eclipse Modelling Framework (EMF) \cite{steinberg09emf}, which implements MOF and underpins many contemporary MDE tools and languages, facilitating their interoperability. Section~\ref{subsec:epsilon} discusses Epsilon \cite{kolovos09thesis}, an extensible platform for the specification of model management languages. The highly extensible nature of Epsilon (which is described below) makes it an ideal host for the rapid prototyping of languages and exploring research hypotheses.  

The purpose of this section is to review EMF and Epsilon, and not to provide a thorough review of all MDE tools. There are many other MDE tools and environments that this section does not discuss, such as ATL and VIATRA for M2M transformation (Section~\ref{subsubsec:model_transformation}), oAW\footnote{\url{http://www.eclipse.org/workinggroups/oaw/}} for model transformation and validation, and the AMMA platform\footnote{\url{http://wiki.eclipse.org/AMMA}} for large-scale modelling, model weaving and software modernisation.

\subsection{Eclipse Modelling Framework}
\label{subsec:emf}
The Eclipse Foundation\footnote{\url{http://www.eclipse.org}} is an open-source community seeking to build an extensible development platform. The Eclipse Modelling Framework (EMF) project \cite{steinberg09emf} provides support for MDE within Eclipse. EMF provides code generation facilities, and a meta-modelling language, Ecore, that implements the MOF 2.0 standard \cite{mof}. EMF is arguably the most widely-used contemporary MDE modelling framework.

EMF is used to generate metamodel-specific editors for loading, storing and constructing models. EMF model editors comprise a navigation view for specifying the elements of the model, and a properties view for specifying the features of model elements. Figure~\ref{fig:emf_model_editor} shows an EMF model editor for a simple state machine language. The navigation (or tree) view is shown in the top pane, while the properties view is shown in the bottom pane.

\begin{figure}[htbp]
  \begin{center}
    \leavevmode
    \includegraphics[width=10cm]{2.Background/images/emf_model_editor.png}
  \end{center}
  \caption{An EMF model editor for state machines.}
  \label{fig:emf_model_editor}
\end{figure}

Users of EMF can define their own metamodels in Ecore, the metamodelling language and MOF implementation of EMF. EMF provides two metamodel editors, tree-based and graphical. Figure~\ref{fig:emf_metamodel_editor_tree} shows the metamodel of a simple state machine language in the tree-based metamodel editor. Figure~\ref{fig:emf_metamodel_editor_diagrammatic} shows the same metamodel in the graphical metamodel editor. Like MOF, the graphical metamodel editor uses concrete syntax similar to that of UML class diagrams. Emfatic\footnote{\url{http://www.alphaworks.ibm.com/tech/emfatic}} provides a further, textual metamodel editor for EMF, and is shown in Figure~\ref{fig:emf_metamodel_editor_textual}. The editors shown in Figure~\ref{fig:emf_metamodel_editor_tree}, \ref{fig:emf_metamodel_editor_diagrammatic} and \ref{fig:emf_metamodel_editor_textual} are used to manipulate the same underlying metamodel, but using different syntaxes. A change to the metamodel in one editor can be propagated automatically to the other two.

\begin{figure}[htbp]
  \begin{center}
    \leavevmode
    \includegraphics[width=10cm]{2.Background/images/emf_metamodel_tree.png}
  \end{center}
  \caption{EMF's tree-based metamodel editor.}
  \label{fig:emf_metamodel_editor_tree}
\end{figure}

\begin{figure}[htbp]
  \begin{center}
    \leavevmode
    \includegraphics[width=10cm]{2.Background/images/emf_metamodel_diagrammatic.png}
  \end{center}
  \caption{EMF's graphical metamodel editor.}
  \label{fig:emf_metamodel_editor_diagrammatic}
\end{figure}

\begin{figure}[htbp]
  \begin{center}
    \leavevmode
    \includegraphics[width=10cm]{2.Background/images/emf_metamodel_textual.png}
  \end{center}
  \caption{The Emfatic textual metamodel editor for EMF.}
  \label{fig:emf_metamodel_editor_textual}
\end{figure}

From a metamodel, EMF can generate an editor for models that conform to that metamodel. For example, the simple state machine metamodel specified in Figures~\ref{fig:emf_metamodel_editor_tree}, \ref{fig:emf_metamodel_editor_diagrammatic} and \ref{fig:emf_metamodel_editor_textual} was used to generate the code for the model editor shown in Figure~\ref{fig:emf_model_editor}. The model editors generated by EMF incorporate mechanisms for loading and saving models. As prescribed by MOF, EMF typically generates code that stores models using XMI \cite{xmi}, a dialect of XML optimised for model interchange.

The Graphical Modeling Framework (GMF) \cite{gronback09emp} is used to create graphical model editors from metamodels defined with EMF. Figure~\ref{fig:gmf_model_editor} shows a model editor produced with GMF for the simple state machine language described above. GMF itself uses a model-driven approach: users specify several models, which are combined, transformed and then used to generate code for the resulting graphical editor. 

\begin{figure}[htbp]
  \begin{center}
    \leavevmode
    \includegraphics[width=10cm]{2.Background/images/gmf_model_editor.png}
  \end{center}
  \caption{GMF state machine model editor.}
  \label{fig:gmf_model_editor}
\end{figure}

Many MDE tools are interoperable with EMF, enriching its functionality. The remainder of this section discusses one tool that is interoperable with EMF, Epsilon.

\subsection{Epsilon}
\label{subsec:epsilon}
The Extensible Platform for Specification of Integrated Languages for mOdel maNagement (Epsilon) \cite{kolovos09thesis} is a suite of tools and domain-specific languages for MDE. Epsilon comprises several integrated model management languages -- built on \changed{``atop'' to ``on''} a common infrastructure -- for performing tasks such as model transformation, model validation and model merging \cite{kolovos09thesis}. Figure \ref{fig:epsilon} illustrates the various components of Epsilon.

Whilst many model management languages are bound to a particular subset of modelling technologies, limiting their applicability, Epsilon is metamodel-agnostic: models written in any modelling language can be manipulated by Epsilon's model management languages \cite{kolovos06eol}. Currently, Epsilon supports models implemented using EMF, MOF 1.4, XML, or Community Z Tools (CZT)\footnote{\url{http://czt.sourceforge.net/}}. Interoperability with further modelling technologies can be achieved by extension of the Epsilon Model Connectivity (EMC) layer. 

\begin{figure}[htbp]
  \begin{center}
    \leavevmode
    \includegraphics[scale=0.6]{2.Background/images/epsilon.png}
  \end{center}
  \caption[The architecture of Epsilon]{The architecture of Epsilon, taken from \cite{rose08egl}.}
  \label{fig:epsilon}
\end{figure}

The architecture of Epsilon promotes reuse when building task-specific model management languages and tools. Each Epsilon language can be reused wholesale in the production of new languages. Ideally, the developer of a new language only has to design language concepts and logic that do not already exist in Epsilon languages. As such, new task-specific languages can be implemented in a minimalistic fashion. This \cc claim has been demonstrated by the style of implementation used to construct the Epsilon Generation Language (EGL) \cite{rose08egl}.

The Epsilon Object Language (EOL) \cite{kolovos06eol} is the core of the platform and provides functionality similar to that of OCL \cite{ocl2}. However, EOL provides an extended feature set, which includes the ability to update models, access to multiple models, conditional and loop statements, statement sequencing, and provision of standard output and error streams.

As shown in Figure \ref{fig:epsilon}, every Epsilon language re-uses EOL, so improvements to EOL enhance the entire platform. EOL also allows developers to delegate computationally intensive tasks to extension points, where the task can be authored in Java.

Epsilon is a member of the Eclipse GMT\footnote{\url{http://www.eclipse.org/gmt}} project, a research incubator for the top-level modelling technology project. Epsilon provides a lightweight means for defining new experimental languages for MDE. For these reasons, Epsilon is uniquely positioned as an ideal host for the rapid prototyping of languages for model management, and hence has been used extensively for the work described in Chapter~\ref{Implementation}. 

\subsection{Summary}
This section has introduced the MDE tools used throughout the remainder of the thesis. The Eclipse Modeling Framework (EMF) provides an implementation of MOF, Ecore, for defining metamodels. From metamodels defined in Ecore, EMF can generate code for metamodel-specific editors and for persisting models to disk. EMF is arguably the most widely used contemporary MDE modelling framework and its functionality is enhanced by numerous tools, such as the Graphical Modeling Framework (GMF) and Epsilon. GMF allows metamodel developers to specify a graphical concrete syntax for metamodels, and can be used to generate graphical model editors. Epsilon is an extensible platform for defining and executing model management languages, provides a high degree of re-use for defining new model management languages and can be used with a range of modelling frameworks, including EMF.
%!TEX root = ../thesis.tex

\section{Research Relating to MDE}
\label{sec:mde_related}
MDE is closely related to several other fields of software engineering. This section discusses two of those fields, Domain-Specific Languages (DSLs) and Language-Oriented Programming (LOP). A further related area, Grammarware, is discussed in the context of software evolution in Section~\ref{subsec:grammar_evolution}. DSLs and LOP are closely related to the research central to this thesis. Other areas relating to MDE but less relevant to this thesis, such as formal methods, are not considered here.

\subsection{Domain-Specific Languages}
\label{subsec:dsls}
For a set of related problems, a specific, tailored approach is likely to provide better results than instantiating a generic approach for each problem \cite{deursen00dslbib}. The set of problems for which the specific approach outperforms the generic approach is termed the \emph{domain}. A \emph{domain-specific programming language} (often called a \emph{domain-specific language} or \emph{DSL}) enables the encoding of solutions for a particular domain.

Like modelling languages, DSLs describe abstract syntax. Furthermore, a common language can be used to define DSLs (e.g. EBNF \cite{ebnf}), like the use of MOF for defining modelling languages. In addition to abstract syntax, DSLs typically define a textual concrete syntax but, like modelling languages, can utilise a graphical concrete syntax.

Cobol, Fortran and Lisp first existed as DSLs for solving problems in the domains of business processing, numeric computation and symbolic processing respectively, and evolved to become general-purpose programming languages \cite{deursen00dslbib}. SQL, on the other hand, is an example of a DSL that, despite undergoing much change, has not grown into a general-purpose language. Unlike a general-purpose language, a single DSL cannot be used to program an entire application. DSLs are often small languages at inception, but can grow to become complicated (such as SQL). Within their domain, DSLs should be easy to read, understand and edit \cite{fowler10dsls}.

There are two ways in which DSLs are typically implemented. An \emph{internal} DSL uses constructs from a general-purpose language (the \emph{host}) to describe the domain \cite{fowler10dsls}. Examples of internal DSLs include the libraries of abstract data types that are part of many programming languages (e.g. STL for C++, the Collections API for Java). Some languages are better than others for hosting internal DSLs. For \cc example, Ruby has been proposed as a suitable host for DSLs due to its unintrusive syntax and flexible runtime evaluation \cite[ch. 4]{fowler10dsls}. In Lisp, internal DSLs can be implemented by using macros to translate domain-specific concepts to Lisp abstractions \cite{graham93lisp}.

When the gap between domain and programming concepts is large, constructing an internal DSL can require a lot of programming effort. Consequently, \cc \emph{translating} DSL programs into code written in a general-purpose language has been recommended \cite{parr07antlr}. The \cc term \emph{external} is sometimes used for this style of DSL implementation \cite{fowler10dsls}. Programs written in simple DSLs are often easy to translate to programs in an existing general-purpose language \cite{parr07antlr}. Approaches to translation include preprocessing; building or generating an interpreter or compiler; or extending an existing compiler or interpreter \cite{fowler10dsls}.

The construction of an external DSL can be achieved using many of the principles, practices and tools used in MDE. Parsers can be generated using text-to-model transformation; syntactic constraints can be specified with model validation; and translation can be specified using model-to-model and model-to-text transformation. MDE tools are used to implement two external DSLs in Chapter~\ref{Implementation}.

Internal \cc and external DSLs have been successfully used as part of application development in many domains \cite{deursen00dslbib}. They have been used in conjunction with general-purpose languages to build systems rapidly and to improve productivity in the development process (such as automation of system deployment and configuration). More recently, some developers are building complete applications by combining DSLs, in a style of development called Language-Oriented Programming. 

\subsection{Language-Oriented Programming}
DSLs \cc are central to LOP, a style of software development \cite{ward94lop}. Firstly, a very high-level language to encode problem domains is developed. Simultaneously, a compiler is developed to translate programs written in the high-level language to an existing programming language. Ward describes how this approach to programming can enhance the productivity of development and the understandability of a system. Additionally, Ward mentions the way in which multiple very high-level languages could be layered to separate domains.

Combining \cc DSLs to solve a problem is not a new technique \cite{fowler10dsls}. Traditionally, UNIX has encouraged developers to combine programs written in small (domain-specific) languages (such as awk, make, sed, lex and yac) to solve problems. Lisp, Smalltalk and Ruby programmers often construct domain-specific languages when developing programs \cite{graham93lisp}.

To fully realise the benefits of LOP, the development effort required to construct DSLs must be minimised. Two approaches for constructing DSLs seem to be prevalent for LOP. The first advocates using a highly dynamic, reflexive and extensible programming language to specify DSLs. This category of language has been termed a \emph{superlanguage} \cite{clark08superlanguages}. The superlanguage permits new DSLs to re-use constructs from existing DSLs, which simplifies development.

A \textit{language workbench} \cite[ch. 9]{fowler10dsls} is an alternative means for simplifying DSL development. Language workbenches provide tools, wizards and DSLs for defining abstract and concrete syntax, for constructing editors and for specifying code generators.

For defining DSLs, the main difference between using a language workbench or a superlanguage is the way in which semantics of language concepts are encoded. In a language workbench, a typical approach is to write a generator for each DSL \cite{fowler10dsls}, whereas a superlanguage often requires that semantics be encoded in the definition of language constructs \cite{clark08superlanguages}.

Like MDE, LOP requires mature and powerful tools and languages to be applicable in the large, and to complex systems. Unlike MDE, LOP tools typically combine concrete and abstract syntax. The emphasis for LOP is in defining a single, textual concrete syntax for a language. MDE tools might provide more than one concrete syntax for a single modelling language. For example, two distinct concrete syntaxes are used for the tree-based and graphical editors of the simple state-machine language shown in Figures~\ref{fig:emf_model_editor} and~\ref{fig:gmf_model_editor}.

Some of the key concerns for MDE are also important to the success of LOP. For example, tools for performing LOP and MDE need to be as usable as those available for traditional development, which often include support for code-completion, automated refactoring and debugging. Presently, these features are often lacking in tools that support LOP or MDE.

In summary, LOP addresses many of the same issues with traditional development as MDE, but requires a different style of tool. LOP focuses more on the integration of distinct DSLs, and providing editors and code generators for them. Compared to LOP, MDE typically provides more separation between concrete and abstract syntax, and concentrates more on model management.

\subsection{Summary}
This section has described two areas of research related to MDE, domain-specific languages (DSLs) and language-oriented programming (LOP). DSLs facilitate the encoding of solutions for a particular problem domain. For solving problems in their domain, DSLs can be easier to read, use and edit than general-purpose programming languages \cite{deursen00dslbib,fowler10dsls}. During MDE, one or more DSLs may be used to model the domain, and the tools and techniques for implementing DSLs can be used for MDE.

LOP is an approach to software development that seeks to specify complete systems using a combination of DSLs. Contemporary LOP seeks to minimise the effort required to specify and use DSLs. Like MDE, LOP requires mature and powerful tools, but, unlike MDE, LOP does not separate concrete and abstract syntax, and does not focus on model management, which is a key development activity in MDE.

%!TEX root = /Users/louis/Documents/PhD/Deliverables/Thesis/thesis.tex

\section{Benefits of and Current Challenges for MDE}
\label{sec:mde_benefits_and_challenges}

\section{Chapter Summary}
This chapter has discussed Model-Driven Engineering (MDE), a state-of-the-art and principled approach to software engineering. The terminology, development activities and tools used in a typical MDE process were introduced. Two areas relating to MDE, language-oriented programming and domain-specific languages, were discussed, and three methods for performing MDE were reviewed.

Traditional approaches to software engineering do not treat modelling artefacts -- such as model, metamodels and model management operations -- as first-class citizens, and they are typically represented in an unstructured manner, if at all. MDE involves creating, manipulating and managing changes to modelling artefacts and therefore modelling artefacts are represented in a structured manner. This chapter has demonstrated that contemporary MDE tools, such as EMF and Epsilon, provide structures and processes for creating and manipulating modelling artefacts, but not for managing evolutionary change. Chapters~\ref{LiteratureReview} and~\ref{Analysis} review, explore and investigate structures and processes for managing the evolution of modelling artefacts.