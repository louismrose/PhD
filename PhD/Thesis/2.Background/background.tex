%!TEX root = /Users/louis/Documents/PhD/Deliverables/Thesis/thesis.tex

\chapter{Background}
\label{Background}

\section{Modelling}

\subsection{Models}

Evans in Domain-Driven Design, pg397: ``Distillation is the process of separating the components of a mixture to extract the essence in a form that makes it more valuable and useful. A model is a distillation of knowledge. With every refactoring to deeper insight, we abstract some crucial aspect of domain knowledge and priorities.''

\subsection{Modelling Languages}

Evans in Domain-Driven Design, pg377: ``Use a well-documented shared language that can express the necessary domain information as a common medium of communication.'' He uses the example of Chemical Markup Language (CML), which has been standardised and facilitates the creation of tools (such as JUMBO Browser, which creates graphical views of chemical structures) that would probably never have been developed for in-house, ad-hoc domain models of chemicals.


\section{Model-Driven Engineering}

\subsection{Tools}

\subsubsection{Epsilon}
\label{subsubsec:epsilon}

%The background chapter will serve two purposes: firstly, to introduce areas of computer science that are related to our research, and secondly to introduce two categories of evolution observed in model-driven engineering. These two categories were described in the progress report.


\section{Related Areas}
%Several subsections will be used, one per related area. Topics are likely to include domain-specific languages and language-oriented programming; refactoring and design patterns; and iterative and incremental approaches to software engineering. We will discuss the applicability and relationship of each area to software evolution in the context of model-driven engineering.

\subsection{Grammarware}
\label{subsec:grammarware}
- Relationship between grammars and metamodels (as discussed in ``Toward an Engineering Discipline for Grammarware'', Klint, particularly section 2).

\section{Categories of Evolution in MDE}
%This section will discuss model and metamodel co-evolution and model synchronisation, two categories of evolution observed in MDE. These categories were introduced in Section~\ref{sec:introduction} of this thesis outline.