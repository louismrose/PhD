%!TEX root = /Users/louis/Documents/PhD/ProgressReport/pr.tex
\section{Introduction}
Model-Driven Architecture (MDA) is a software engineering framework defined by the Object Management
Group (OMG) \cite{omg}. MDA provides a set of guidelines for developing computer systems in a model-centric, or \textit{model-driven}, fashion. In \textit{model-driven engineering} (MDE), models are utilised as the primary software development artefact. Several approaches to MDE are prevalent today, such as \cite{stahl06mdsd}, \cite{kelly08dsm} and \cite{greenfield04software}. The approaches vary in the extent to which they follow the guidelines set out by MDA.

In MDE, the primary software development artefacts are \textit{models}. When used here, the term model has the same meaning as given in \cite{kolovos06eol}: a model is a description of a phenomenon of interest, and may have either a textual or graphical representation. A model provides an abstraction over a real-world object, which enables engineers of differing disciplines to reason about that object.

MDA prescribes automated - rather than (partially) manual - transformations between models; a crucial difference between MDE and similar approaches to software development. Employing the MDA guidelines has been shown to significantly improve developer productivity and the correctness of software \cite{watson08mdahistory}.

\subsection{Software Evolution in Model-Driven Engineering}
\label{Intro:MigrationProblem}
Studies \cite{erlikh00leveraging,moad90maintaining} suggest that the evolution of software can account for as much as 90\% of a development budget. \cite{sjoberg93quantifying} identifies reasons for software evolution, which include addressing changing requirements, adapting to new technologies, and architectural restructuring.

Modern software development often involves constructing a system by combining numerous types of artefact (such as source and object code, build scripts, documentation and configuration settings). Artefacts depend on each other. Some examples of these dependencies from traditional development include: compiling object code from source code, generating documentation from source code, and deploying object code using a build script. When one artefact is changed, the development team updates the other artefacts accordingly. Here, this activity is termed \textit{migration}. Some literature refers to it as \textit{co-evolution}.

MDE prescribes automated transformation from models to code. Transformation may be partial or complete; and may be direct to code or via intermediate models \cite{kleppe03mda}. Any code or intermediate models generated by these transformations are dependent on other development artefacts -- e.g. a change to a model may have an impact on other models \cite{deursen07mdse}.

In traditional software development, some migration activities can be automated (e.g. background incremental compilation of source code to object code), while some must be performed manually (e.g. updating design documents after adding a new feature). MDA seeks to reduce the amount of manual migration required to develop software, but presently no tools for MDE fully support automated migration.