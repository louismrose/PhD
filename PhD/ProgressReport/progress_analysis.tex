%!TEX root = /Users/louis/Documents/PhD/Deliverables/ProgressReport/pr.tex
\subsection{Analysis of Existing Co-evolution Research}
\label{sub:analysis_of_existing_techniques}
Several approaches to automating co-evolution have been proposed. \cite{sprinkle04domain} were the first to describe co-evolution as distinct from the more general activity of model-to-model transformation. \cite{gruschko07towards} suggests inferring migration strategies from metamodel changes. \cite{wachsmuth07metamodel} classifies metamodel changes, formally defining their impact on models. Wachsmuth was the first to employ higher-order model transformation\footnote{A model-to-model transformation that consumes or produces a model-to-model transformation is termed a higher-order model transformation.} to generate a model-to-model transformation for performing model migration. \cite{cicchetti08automating} proposes using higher-order transformations to infer migration strategies from the change history of a metamodel. \cite{herrmannsdoerfer08cope} describes a co-evolution language, COPE, that allows for the decomposition of a migration into modular changes. COPE enforces model and metamodel consistency constraints only at the start and end of each modular change; whereas most model-to-model transformation languages enforce this consistency after every statement.

% Firstly, very little research has investigated the co-evolution occurring in existing MDE projects. (fold)

%\subsubsection{Investigating Data from Existing MDE Projects}
%\cite{dig06automatic} describes a tool for automatically updating programs in response to a change made to a programming interface. To derive requirements for the tool, Dig performed a study of the evolution of existing programming interfaces in \cite{dig06apis}. The study showed that over 80\% of the changes made to programming interfaces could be categorised as refactorings (changes that did not alter the external behaviour of the programming interface).

%Most co-evolution research has not engineered requirements by analysing the co-evolution occurring in existing MDE projects. The exception is \cite{herrmannsdoerfer08automatability}, which details a study of the evolution of two metamodels. The study concentrates on partitioning the metamodel changes into three categories: metamodel-independent (observed in all metamodels), metamodel-specific (observed in only one metamodel) and model-specific (observed only in one instance of only one metamodel). Herrmannsdoerfer et al. find that over two-thirds of metamodel changes are metamodel-independent and, consequently, may be re-used across domains. Hence, \cite{herrmannsdoerfer08automatability} identifies re-use as a key requirement for a co-evolution language.

%There are several problems with the study presented in \cite{herrmannsdoerfer08automatability}. Firstly, neither of the metamodels studied are in the public domain. Therefore, the study cannot be recreated. Secondly, both metamodels describe aspects of human machine interfaces (HMIs). One metamodel formalises the specification of HMIs, while the other describes the concepts used to test HMIs. Due to the overlapping domains of the two metamodels, I believe this may have lead to some metamodel changes to have been incorrectly categorised as metamodel-independent.  (end)


\subsubsection{Analysis of Existing Co-evolution Approaches}
Automating co-evolution remains an open research challenge. There are problems with existing co-evolution research, which need to be addressed. Crucially, no co-evolution research includes a thorough evaluation. In particular, application to existing MDE projects is overlooked. Consequently, it is difficult to assess the extent to which existing research may be applied to automate co-evolution. 

I have started to analyse existing approaches to co-evolution. I will not analyse \cite{wachsmuth07metamodel} and \cite{gruschko07towards}, as they present only theoretical frameworks for performing co-evolution. \cite{sprinkle04domain} provides an implementation, but it uses outdated technology and significant development effort would be required to update it. Instead, my analysis will focus on the work described in \cite{herrmannsdoerfer08cope} and \cite{cicchetti08automating}. Both provide implementations for their co-evolution techniques.

\subsubsection{Initial Investigation}
\label{subsubs:intial_investigation}
Model migration is often encoded as a model-to-model transformation, but there are problems with this approach. I want to assess to what extent co-evolution tools provide improvements over using model-to-model transformation for model migration. To formulate a process for performing the analysis of existing co-evolution techniques, I selected a set of evolutionary changes from the FPTC project. Users of the FPTC project had a requirement for automatically migrating their models from one version of the FPTC metamodel to another.

I began by encoding the migration with the Epsilon Transformation Language (ETL) \cite{kolovos08etl}, a model-to-model transformation language. An excerpt from the transformation is presented in Listing \ref{lst:evo_trans}. 

\begin{lstlisting}[caption=Excerpt from the transformation used to migrate FPTC models., label=lst:evo_trans,language=etl]
rule Block2Block
  transform old : Old!Block
  to        b   : New!Block  {
  
	b.name := old.name;
	b.system := old.system.equivalent();
	b.faultBehaviour := old.faultBehaviour.equivalent();
	
	for (successor in old.successors) {
		var connection := new New!Connection;
		connection.source := b;
		connection.target := successor.equivalent();
		b.system.connections.add(connection);
	}
}
\end{lstlisting}

I identified three main problems with encoding the migration strategy as a model-to-model transformation. Firstly, I had to manually formulate and codify the entire transformation, which was time consuming. Secondly, the language lacked facilities for encoding this kind of transformation: for example, explicit copying of data (as seen on lines 5-7) occurred often and was tedious to write. Thirdly, I had to augment the model-to-model transformation engine to allow batch processing, required for the simultaneous migration of all instances of the old metamodel.


\subsubsection{Analysis Process}
I will use the MDE projects identified in Section \ref{sub:examples} to analyse existing co-evolution tools. So far, I have concentrated on devising a process for the analysis. The following steps for each evolutionary change will be carried out in the co-evolution tool under analysis:

\begin{enumerate}
	\item Recreate the metamodel as it was before any evolutionary change.
	\item Adapt the metamodel according to the version history in its (real-world) project.
	\item Migrate each instance of the original metamodel to be consistent with the adapted metamodel.
\end{enumerate}

To determine in what sense each co-evolution tool provides advantages over using a model-to-model transformation for model migration, I will ask the following questions:

\begin{itemize}
	\item Are all of the migrated instances consistent with the evolved metamodel?
	\item To what extent was the tool able to automate the formulation of the intended migration strategy? To what extent could the generated migration strategy be customised?
	\item Did the tool provide increased expressiveness over a model-to-model transformation for model migration?  
	\item Was the tool capable of applying the migration strategy to more than one model at a time?
	\item Did use of the tool alter my development process? 
\end{itemize}


\subsubsection{Analysis Findings}
I have begun to analyse COPE \cite{herrmannsdoerfer08cope} with data from the FPTC project.  Analysis has been impeded by several bugs in, and a lack of documentation for, COPE. COPE's author has been working to resolve my issues with the tool. My findings so far are summarised in this section.

COPE provides reusable co-evolution operators. These operators specify both a metamodel adaptation and a corresponding model migration. Using the operators greatly reduces the effort required to migrate models from one metamodel version to another. However, not all of the evolutionary changes made to the FPTC metamodel could be represented using operators alone. Instead, I had to adapt the metamodel in the COPE editor and manually encode a migration strategy. COPE provides an imperative scripting language for writing migration strategies. COPE's migration language provided a major benefit over using a model-to-model transformation language: there is no need to copy existing data from one version of a model to another; unless overwritten, data is automatically copied from the old to the new version.

COPE makes assumptions on the development process. Firstly, COPE requires that the metamodel adaptation be available as a sequence of primitive changes. Consequently, COPE provides a metamodel editor that forces adaptation to be performed as a sequence of primitive, atomic changes. However, some developers edit their metamodels using a textual syntax. Freeform text editing does not produce a sequence of atomic metamodel changes. Secondly, COPE assumes that metamodel history will be stored as a model on the developer's hard disk. This leads to problems with source code management systems (SCMs) (used to record historical versions of software, and manage collaborative development). SCMs cannot be reasonably expected to respect the semantics of the COPE history model. Consequently, COPE complicates sharing and versioning metamodels.

Other problems with COPE relate to its co-evolution operators. Firstly, it is difficult to ascertain the effects of the operators. COPE only provides a name and the metamodel construct (e.g. class, attribute) to which the operator may be applied. From this information, I had to guess the effects of the metamodel adaptation and the model migration strategy. Frequently, they were not the effects that I expected. Secondly, extending COPE with new co-evolution operators requires the definition of an Eclipse plug-in, which is a non-trivial activity.

\subsubsection{Summary and Future Analysis}
I have selected two existing co-evolution tools for analysis. As discussed in Section \ref{sub:examples}, I will use data from existing MDE projects to analyse the tools. I have investigated using a model-to-model transformation for performing model migration, and highlighted problems with this approach. I have defined a process for analysing co-evolution tools and a set of questions for comparing approaches to co-evolution. Finally, I have presented findings after using one of the co-evolution tools, COPE.

I will continue my analysis by using COPE to recreate other evolutionary changes occurring in the FPTC project, and to recreate the changes seen in the other examples identified in Section~\ref{sub:examples}. I will also analyse the tool described in \cite{cicchetti08automating}.