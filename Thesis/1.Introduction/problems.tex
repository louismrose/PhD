%!TEX root = /Users/louis/Documents/PhD/Deliverables/Thesis/thesis.tex

\section{Motivation: Software Evolution in MDE}
Proponents of MDE suggest that, compared to traditional approaches to software engineering, application of MDE leads to systems that better support evolutionary change \cite{kleppe03mda}. Large-scale \cc systems developed with traditional approaches to software engineering have been described as examples of a modern-day Sisyphus\footnote{In Greek mythology, Sisyphus was condemned to an eternity of repeatedly rolling a boulder to the top of a mountain, only to see it return to the mountain's base.}, whose developers must constantly perform evolution to support conformance to changing standards and interoperability with external systems \cite{frankel02mda}. Some \cc proponents suggest that MDE can be used to reduce the cost of software evolution \cite{frankel02mda}, while others report that MDE introduces additional challenges for managing software evolution \cite{Mens07}.

In particular, the evolution of models, modelling languages and other MDE development artefacts must be managed in MDE. Contemporary development environments provide some assistance for performing software evolution activities (by, for example, providing transformations that automatically restructure code). However, there is little support for software evolution activities that involve models and modelling languages. Chapters~\ref{LiteratureReview} and~\ref{Analysis} review, analyse and motivate improvements to the way in which software evolution is identified and managed in contemporary MDE development environments. Chapters~\ref{Implementation} and~\ref{Evaluation} explore the extent to which the productivity of identifying and  managing evolutionary change can be increased by extending contemporary MDE development environments with additional, dedicated structures and processes.
